\chapter*{\centering Résumé}
\addcontentsline{toc}{chapter}{Résumé}
%Encabezado
%\lhead[Trabajo Final de Máster]{Resumen}
%\rhead[Resumen]{Trabajo Final de Máster}

Le travail qu'on se présente par la suite porte sur l'ensaignement de l'unité didactique correspondant à la ressemblance qui se donne dans les cours de la ESO. On compte faire un analyse en comparant la situation entre la France et l'Espagne, grâce à l'article de \cite{Horoks}, en tenant en compte la configuration de ce sujet ci dans les programmes scolaires français et espagnols. Pour ce faire, on a choisit deux livres (un pour chaque pays) à partir desquels on va se détailler les structures des unités didactiques, de même que l'étude de certains exercices exposés dans eux pour développer cet analyse.

%El trabajo que se presenta a continuación versa sobre la docencia de la unidad didáctica correspondiente a la semejanza que se imparte en los cursos de la ESO. Basándonos en el artículo \cite{Horoks} y desde la perspectiva de la configuración de dicho tema, tanto en los programas escolares en el ámbito español como en el francés, se pretende llevar a cabo un análisis comparando la situación en ambos países. Para ello se han seleccionado dos libros como referente (uno para cada país) a partir de los cuales se detallarán las estructuras didácticas de las unidades, así como el estudio de algunos de los ejercicios expuestos en ellos para desarrollar dicho análisis. %Se pretende destacar cuáles son las mayores dificultades que tienen los alumnos con estas nociones y si, en cada país, se resuelven las del otro y las propias.

\underline{\textbf{Mots clés}}: Triangles semblables, France-Espagne, Niveau de Mise en Fonctionnement, Tâches, Practiques Enseignantes.