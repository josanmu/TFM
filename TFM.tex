\documentclass[12pt,a4paper]{book}
\usepackage{setspace}
\usepackage[utf8]{inputenc}
\usepackage[spanish]{babel}
\decimalpoint %punto decimal
\usepackage{amsmath}   %paquetes matematicas necesarios
\usepackage{amsfonts}
\usepackage{amssymb}
\usepackage{amsthm}
\usepackage{esvect}
\usepackage{mathtools}
\usepackage{multirow}
\usepackage{caption}
\usepackage{slashbox} 
\usepackage{cancel}  %paquete para tachar (cancelar) en ecuaciones y entornos math ($bla bla bla$)
\usepackage{color}   %poner color al texto \textcolor{"color"}{texto que se quiera poner en color}
\usepackage{colortbl} %poner color a las tablas
\usepackage{graphicx}  %entorno para poner imágenes
\usepackage{epstopdf}
\usepackage{animate}	%añadir gifs
%\graphicspath{ {images/} }    %para indicar el directorio de las imagenes
\usepackage[all]{xy}  %paquete para poner esquemas (flechas y eso)
\usepackage[left=2.54cm,right=2.54cm,top=2.54cm,bottom=2.54cm]{geometry}%[a4paper,textwidth=80ex,bottom=2.5cm]{geometry}%[left=2.5cm,right=2.5cm,top=2.5cm,bottom=2.5cm]{geometry}   %paquete 													para margenes en la pagina
\usepackage{titlesec}
%\titleformat{\chapter}[hang]
%    {\titlerule[2pt]\Large\scshape\raggedright}{}{0em}{Capítulo \thechapter. }[{\titlerule[2pt]}]
\usepackage{pdfpages} %Inclusión de pdfs a página completa, para incrustados usar figure
\usepackage{url}
\usepackage[breaklinks=true]{hyperref}
\hypersetup{
    colorlinks = true,
    %citebordercolor=false,
    linkcolor=black,
    urlcolor=blue,
    urlbordercolor={1 0 0},
    citecolor=black
}
\usepackage[nottoc]{tocbibind} %añadir bibliografía a índice
\usepackage{pifont} %fuente de simbolos
\usepackage{multirow}
	%\multicolumn{no de columnas}{alineacion}{texto}
	%\multirow{no de filas}{alineacion}{texto}
	%(si no queremos especificar el ancho ponemos *).
%\usepackage[none]{hyphenat}%\sloppy     %no cortar palabras
\usepackage{booktabs}% http://ctan.org/pkg/booktabs
\usepackage{multicol}
	\setlength{\columnsep}{0cm}
\usepackage{array}
\usepackage{pgf,tikz}
\usepackage{mathrsfs}
\usetikzlibrary{arrows}
%\usepackage{ornament}

%\usepackage{apacite}
\usepackage{flexbib}
	\renewcommand{\bbletal}{et al.}
	\renewcommand{\bibnamefont}[1]{#1}
	\renewcommand{\bibfnamefont}[1]{#1}
	\renewcommand{\nameseparator}{,}
%\addto\captionsspanish{\renewcommand{\bibliographyname}{Bibliografía}}


%%%%%%%%%%%%%%%%%%%%%%%%%%%%%%%%%%%%%%%%%%%%%%%%%%%%%%%%%%%%%%%%%%%%%%%%%%%%%%%%%%%%%%%%%%%%%%%%%%%%%%%%%%
%%																										%%
%% --------------------------------------- MODIFICACIONES --------------------------------------------- %%
%%																										%%
%%%%%%%%%%%%%%%%%%%%%%%%%%%%%%%%%%%%%%%%%%%%%%%%%%%%%%%%%%%%%%%%%%%%%%%%%%%%%%%%%%%%%%%%%%%%%%%%%%%%%%%%%%
\widowpenalty=10000 %lineas viudas
\clubpenalty=10000 %lineas huerfanas


\newtheorem{defi}{Definición}[section]
  \newtheorem{teo}{Teorema}[section]
  \newtheorem{ej}{Ejemplo}[section]
\newtheorem{prop}{Proposición}[section]
\renewcommand{\theprop}{\arabic{chapter}.\arabic{section}.\arabic{prop}}
\newtheorem{cor}{Corolario}[section]
\newtheorem{lema}{Lema}[section]

\newcommand{\f}[1]{{\textsc\sffamily #1}}
\newcommand{\mt}[1]{\mathbb{#1}}
\newcommand{\tu}[1]{\textup{#1}}
\newcommand{\cl}[1]{\mathcal{#1}}
\newcommand{\sm}{\sum\limits}
\newcommand{\Figura}[1]{\textbf{Figura \ref{#1}}}
\newcommand{\cod}[1]{{\small\ttfamily\fcolorbox{white}{gray90}{\textbf{#1}}}}
\newcommand{\tc}[2]{\textcolor{#1}{#2}}
\renewcommand{\sin}{\operatorname{\sen}}
\newcommand{\tabitem}{~~\llap{\textbullet}~~}

%Cita
\newenvironment{dedication}
	{%\clearpage	%we want a new page 		%%	comented this
		\thispagestyle{empty} %no header an footer
		%\vspace*{\stretch{1}} %some space at the top
		\itshape			  %the text is in italic
		\raggedleft			  %flush to the right margin
	}
	{	\par %end the paragraph
		%\vspace{\stretch{3}} %space at the bottom is three times that at the top
		%\clearpage %finish off the page
	}

%%%%%%%%%%%%%%%%%%%%%%%%%%%%%%%%%%%%%%%%%%%%%%%%%%%%%%%%%%%%%%%%%%%%%%%%%%%%%%%%%%%%%%%%%%%%%%%%%%%%%%%%%%	
%---------------------------------------------------------------------------------------------------------------
%%%%%%%%%%%%%%%%%%%%%%%%%%%%%%%%%%%%%%%%%%%%%%%%%%%%%%%%%%%%%%%%%%%%%%%%%%%%%%%%%%%%%%%%%%%%%%%%%%%%%%%%%%


%%%%%%%%%%%%%%%%%%%%%%%%%%%%%%%%%%%%%%%%%%%%%%%%%%%%%%%%%%%%%%%%%%%%%%%%%%%%%%%%%%%%%%%%%%%%%%%%%%%%%%%%%%
%%																										%%
%% ------------------------------------------- ENCABEZADO --------------------------------------------- %%
%%																										%%
%%%%%%%%%%%%%%%%%%%%%%%%%%%%%%%%%%%%%%%%%%%%%%%%%%%%%%%%%%%%%%%%%%%%%%%%%%%%%%%%%%%%%%%%%%%%%%%%%%%%%%%%%%
\usepackage{fancyhdr}
%
%% aqui definimos el encabezado de las paginas pares e impares. define el texto a la izquierda (l = left, c = center, r = right) de los encabezados
%\lhead[\thepage]{}
%\chead[j]{m}
%\rhead[]{\thepage}
\renewcommand{\headrulewidth}{0.5pt}%define el grosor de la línea
%
%% aqui definimos el pie de pagina de las paginas pares e impares.
\lfoot[\thepage]{José Andrés Muñoz}
\cfoot[]{}
\rfoot[José Andrés Muñoz]{\thepage}
\renewcommand{\footrulewidth}{0.5pt}
%
%% aqui definimos el encabezado y pie de pagina de la pagina inicial de un capitulo. define el texto de los encabezados de la primera página de un capítulo
\fancypagestyle{plain}{
\fancyhead[L]{}
\fancyhead[C]{}
\fancyhead[R]{}
\fancyfoot[L]{}
\fancyfoot[C]{}
\fancyfoot[R]{}
\renewcommand{\headrulewidth}{0pt}
\renewcommand{\footrulewidth}{0pt}
}
%
%\pagestyle{fancy}

%%%%%%%%%%%%%%%%%%%%%%%%%%%%%%%%%%%%%%%%%%%%%%%%%%%%%%%%%%%%%%%%%%%%%%%%%%%%%%%%%%%%%%%%%%%%%%%%%%%%%%%%%%
%---------------------------------------------------------------------------------------------------------------- 
%%%%%%%%%%%%%%%%%%%%%%%%%%%%%%%%%%%%%%%%%%%%%%%%%%%%%%%%%%%%%%%%%%%%%%%%%%%%%%%%%%%%%%%%%%%%%%%%%%%%%%%%%%

\usepackage{makeidx} %añade indice terminologico
\def\indexname{Índice Terminológico}
%\makeatletter
%\renewcommand*{\cleardoublepage}{\clearpage\if@twoside
%\ifodd\c@page\else
%\hbox{}\thispagestyle{empty}\newpage
%\if@twocolumn\hbox{}\newpage\fi\fi\fi}
%\makeatother
%\makeindex


%%%%%%%%%%%%%%%%%%%%%%%%%%%%%%%%%%%%%%%%%%%%%%%%%%%%%%%%%%%%%%%%%%%%%%%%%%%%%%%%%%%%%%%%%%%%%%%%%%%%%%%%%%
%%																										%%
%% ------------------------------------------- CUERPO ------------------------------------------------- %%
%%																										%%
%%%%%%%%%%%%%%%%%%%%%%%%%%%%%%%%%%%%%%%%%%%%%%%%%%%%%%%%%%%%%%%%%%%%%%%%%%%%%%%%%%%%%%%%%%%%%%%%%%%%%%%%%%

\includeonly{
	DocsPortada/Portada,
	%DocsPortada/Portada1,
	DeclaracionOriginalidad,
	Agradecimientos,
	Resumen,
	Introduccion,
	SintesisArticulo,
	Analisis,
	Reflexiones,
	Anexos}
\begin{document}
\parskip=10pt plus 1pt minus 3pt %interlineado
\renewcommand{\proofname}{\underline{\textit{Solución}}}
\renewcommand{\tablename}{Tabla}

%Portada:
	
	\pagestyle{empty}
	\pagenumbering{Roman} 
	\include{DocsPortada/Portada}
	%\include{DocsPortada/Portada1}
	
\onehalfspace

%Documentos Previos
	\chapter*{Declaración de Originalidad}

	\begin{center}
	%\begin{large}
		\textbf{ASIGNATURA TRABAJO FIN DE MÁSTER}
	\end{center}
	\begin{center}
		\textbf{MÁSTER UNIVERSITARIO EN FORMACIÓN DEL PROFESORADO DE EDUCACIÓN SECUNDARIA Y BACHILLERATO, FORMACIÓN PROFESIONAL, ENSEÑANZAS DE IDIOMAS Y ENSEÑANZAS ARTÍSTICAS}\\
	\end{center}
	\begin{center}
		\textbf{DECLARACIÓN DE AUTORÍA DEL TRABAJO}
	%\end{large}
	\end{center}
	
	\textit{D./D.ª} José Andrés Muñoz, \textit{con DNI} 48647967-T, declaro que el Trabajo Fin de Máster presentado en la asignatura del mismo nombre conducente a obtener el Título de Máster Universitario en Formación del Profesorado de Educación Secundaria Obligatoria y Bachillerato, Formación Profesional, Enseñanzas de Idiomas y Enseñanzas Artísticas es original del estudiante que lo firma, así como que su elaboración es consecuencia de mi trabajo personal.
	
	Para que conste a efectos de la evaluación de mi Trabajo Fin de Máster, firmo el presente documento en
	
	\begin{center}
		Murcia, a 21 de junio de 2017
	\end{center}
	
	%\vspace{0.45cm}
	
	\begin{center}
		\includegraphics[scale=0.7]{Firma.jpg}\\
%	\end{center}
%	\begin{center}
		Fdo: José Andrés Muñoz
	\end{center}
	
	
	
	

%	JOSÉ ANDRÉS MUÑOZ, autor del TFG 
%	
%	\vspace{1cm}
%	
%	\begin{center}
%		\begin{huge} ``Matemáticas Aplicadas a la Realidad Virtual''\end{huge}
%	\end{center}
%	
%	\vspace{1cm}
%	
%	bajo la tutela del profesor ELISEO CHACÓN VERA, declara que el trabajo que presenta es original, en el sentido de que ha puesto el mayor empeño en citar debidamente todas las fuentes utilizadas.
%	
%	En Murcia, a 5 de septiembre de 2016.
%	
%	\vspace{3cm}
%	
%	(Nota: En la Secretaría de la Facultad de Matemáticas se ha presentado una copia firmada de esta declaración).
	
	\chapter*{Agradecimientos}

	\textit{Ante todo quiero dedicar las primeras palabras de agradecimiento a mi familia. Gracias a ella, su apoyo, su constancia y su ánimo he conseguido la fuerza para lograr todo aquello que me he propuesto. Con este trabajo se cierra una etapa de mi vida que no hubiera sido capaz de lograr sin ellos. Para ellos, que siempre creyeron en mí: Gracias, espero estar a la altura.}
	
	\textit{También quiero dar un especial agradecimiento a mi tutora, Dolores, por su dedicación y toda su ayuda y guía en la elaboración de este trabajo, pues siempre tuvo la puerta de su despacho abierta cuando lo necesité, aconsejándome sobre el más mínimo detalle que se presentase. Asimismo, agradecer también a los profesores Rafael, Ascensión y Juan Diego, del IES ALfonso X ``El Sabio'' de Murcia, toda su ayuda y colaboración con este trabajo. Al igual que mi más sincero y enorme agradecimiento a Alba, quien facilitó el libro francés que aquí se considera y, sin el cuál, este trabajo no habría sido posible. Y no puedo tampoco olvidar a mi gran profesor Salvador, por su ayuda también en la composición de este trabajo y que, además, como sabe, ha marcado siempre en mí frente y guía como profesor, demostrando lo que significa la implicación que ha de tener uno en la docencia, con el alumno y con la clase, que velar y cuidar por esa buena relación también es una labor de aquél que quiera ser buen profesor.}
	
	\textit{Y, finalmente, dado que con este trabajo me sitúo un paso más cerca del que es mi sueño: llegar un día a ser profesor, quiero agradecer a todos aquellos que me marcaron, con el paso de los años, ese hilo de ilusión que me hizo una vez plantearme cómo sería estar, no detrás del pupitre, sino delante, en la pizarra, y que acabó por convertirse en mi principal objetivo profesional. En especial a Virtudes, que al igual que Salvador, me mostró que un profesor no deja de estar ahí cuando se acaba la clase, ser profesor no es algo que se active o desactive cuando se pretenda, sino que significa mucho más, tanto para sí mismo, como para el alumno.}








%	\textit{Agradecer en primer lugar a mi familia pues, si hoy estoy donde estoy, no es gracias sino a ellos que por su gran esfuerzo, en todos los sentidos, me han apoyado y animado a seguir cada día, motivándome para poder conseguir mi sueño, que hoy estoy cumpliendo. Y junto a ellos, todas las personas que, estando a mi lado, saben el sudor que ésto me ha conllevado.}
%	
%	\textit{Agradecer, por supuesto, a mi tutor Eliseo, por haberme dado la oportunidad de haber podido realizar este trabajo y atenderme siempre con la paciencia que lo ha hecho. Dar también, cómo no, mi más sincero agradecimiento a mis dos profesores ya que, sin tener por qué, ahí estuvieron siempre con su mano tendida para este trabajo: Gregorio, por la paciencia y la gran ayuda desinteresada ofrecida en los incontables y extensísimos correos que intercambiábamos sobre la programación; y Salvador, por tener siempre la puerta abierta a cualquier ayuda que pudiera dar, tanto académica para la propia redacción de este trabajo, como personal. Y, finalmente, no quisiera terminar sin dar una mención especial de agradecimiento a mi gran estimado Alberto del Valle, por su gran dedicación tanto a la Facultad como a sus alumnos, y quien siempre ha tenido abierta la puerta de su despacho, en particular, con un cariño especial para algunos de nosotros. Y hacer, de igual modo, una mención especial, en este trabajo, a mi antigua profesora, y amiga, Virtudes, por su gran granito de arena y su siempre inestimable ayuda a pesar de los años que van pasando.}
%	
	\begin{flushright}
		\textit{Gracias a todos.}
	\end{flushright}
	
	
	
	%Y, finalmente, no quisiera terminar sin dar una mención especial de agradecimiento a mi gran estimado Alberto del Valle, por su gran dedicación tanto a la Facultad como a sus alumnos, y quien, con un cariño especial, siempre ha tenido abierta la puerta de su despacho para algunos de nosotros.

%Indice:
	\tableofcontents 
	%Encabezado
	\lhead[Trabajo Final de Máster]{Índice}
	\rhead[Índice]{Trabajo Final de Máster}
	\pagestyle{fancy}
	
%Introduccion/Resumen:
\chapter*{\centering Resumen}
\phantomsection\addcontentsline{toc}{chapter}{Resumen}
%Encabezado
%\lhead[Trabajo Final de Máster]{Resumen}
%\rhead[Resumen]{Trabajo Final de Máster}

El trabajo que se presenta a continuación versa sobre la docencia de la unidad didáctica correspondiente a la semejanza que se imparte en los cursos de la ESO. Basándonos en el artículo de \citet{Horoks} y desde la perspectiva de la configuración de dicho tema, tanto en los programas escolares en el ámbito español como en el francés, se pretende llevar a cabo un análisis comparando la situación en ambos países. Para ello se han seleccionado dos libros como referente, uno para cada país, a partir de los cuales se detallarán las estructuras didácticas de las unidades, así como el estudio de algunos de los ejercicios expuestos en ellos para desarrollar dicho análisis. %Se pretende destacar cuáles son las mayores dificultades que tienen los alumnos con estas nociones y si, en cada país, se resuelven las del otro y las propias.

\underline{\textbf{Palabras clave}}: Semejanza de Triángulos, Francia-España, Nivel de Puesta en Funcionamiento, Tareas, Prácticas Docentes.%Conocimientos particulares de la semejanza.

\vspace{1.5cm}

\begin{center}
	\begin{Huge}
		\textbf{Résumé}
	\end{Huge}
\end{center}

\vspace{1cm}

Le travail qu'on se présente par la suite porte sur l'ensaignement de l'unité didactique correspondant à la ressemblance qui se donne dans les cours de la ESO. On compte faire un analyse en comparant la situation entre la France et l'Espagne, grâce à l'article de \cite{Horoks}, en tenant en compte la configuration de ce sujet ci dans les programmes scolaires français et espagnols. Pour ce faire, on a choisit deux livres (un pour chaque pays) à partir desquels on va se détailler les structures des unités didactiques, de même que l'étude de certains exercices exposés dans eux pour développer cet analyse.

\underline{\textbf{Mots clés}}: Triangles semblables, France-Espagne, Niveau de Mise en Fonctionnement, Tâches, Practiques Enseignantes.

%\chapter*{\centering Résumé}
%\addcontentsline{toc}{chapter}{Résumé}
%
%Le travail qu'on se présente par la suite porte sur l'ensaignement de l'unité didactique correspondant à la ressemblance qui se donne dans les cours de la ESO. On compte faire un analyse en comparant la situation entre la France et l'Espagne, grâce à l'article de \cite{Horoks}, en tenant en compte la configuration de ce sujet ci dans les programmes scolaires français et espagnols. Pour ce faire, on a choisit deux livres (un pour chaque pays) à partir desquels on va se détailler les structures des unités didactiques, de même que l'étude de certains exercices exposés dans eux pour développer cet analyse.
%
%\underline{\textbf{Mots clés}}: Triangles semblables, France, Espagne, Niveau de Mise en Fonctionnement, Tâches.
	
	

%\begin{itemize}
%	\item Semejanza de Triángulos
%	\item Francia
%	\item España
%	\item Nivel de Puesta en Funcionamiento
%	\item Conocimientos particulares de la semejanza
%\end{itemize}
\pagestyle{empty}
%\chapter*{\centering Résumé}
\addcontentsline{toc}{chapter}{Résumé}
%Encabezado
%\lhead[Trabajo Final de Máster]{Resumen}
%\rhead[Resumen]{Trabajo Final de Máster}

Le travail qu'on se présente par la suite porte sur l'ensaignement de l'unité didactique correspondant à la ressemblance qui se donne dans les cours de la ESO. On compte faire un analyse en comparant la situation entre la France et l'Espagne, grâce à l'article de \cite{Horoks}, en tenant en compte la configuration de ce sujet ci dans les programmes scolaires français et espagnols. Pour ce faire, on a choisit deux livres (un pour chaque pays) à partir desquels on va se détailler les structures des unités didactiques, de même que l'étude de certains exercices exposés dans eux pour développer cet analyse.

%El trabajo que se presenta a continuación versa sobre la docencia de la unidad didáctica correspondiente a la semejanza que se imparte en los cursos de la ESO. Basándonos en el artículo \cite{Horoks} y desde la perspectiva de la configuración de dicho tema, tanto en los programas escolares en el ámbito español como en el francés, se pretende llevar a cabo un análisis comparando la situación en ambos países. Para ello se han seleccionado dos libros como referente (uno para cada país) a partir de los cuales se detallarán las estructuras didácticas de las unidades, así como el estudio de algunos de los ejercicios expuestos en ellos para desarrollar dicho análisis. %Se pretende destacar cuáles son las mayores dificultades que tienen los alumnos con estas nociones y si, en cada país, se resuelven las del otro y las propias.

\underline{\textbf{Mots clés}}: Triangles semblables, France-Espagne, Niveau de Mise en Fonctionnement, Tâches, Practiques Enseignantes.
\titleformat{\chapter}[hang]
    %{\titlerule[2pt]\Large\scshape\raggedright}{}{0em}{}[{\titlerule[2pt]}]
    {\titlerule\bf\Huge}{}{0em}{}[{\titlerule}]
\chapter*{Introducción}
\phantomsection\addcontentsline{toc}{chapter}{Introducción}
\pagenumbering{arabic} 
\setcounter{page}{1}
\lhead[Trabajo Final de Máster]{Introducción}
\rhead[Introducción]{Trabajo Final de Máster}
\lfoot[\thepage]{José Andrés Muñoz}
\rfoot[José Andrés Muñoz]{\thepage}
\begin{dedication} 
\large ``La agudeza consiste en saber la semejanza de las cosas diferentes,\\ y la diferencia de las cosas semejantes.''
	\rightline{{\rm --- Germaine De Staël (1766-1817)}}
\end{dedication}
\pagestyle{fancy}
%\begin{dedication} 
%\large ``Las matemáticas, bien vistas,\\ no sólo poseen la verdad.''
%	\rightline{{\rm --- Bertrand Russell (1872-1970)}}
%\end{dedication}

	La Geometría, vista desde los ojos de un matemático, no es únicamente una de las ramas básicas del estudio de esta maravillosa ciencia, sino que es la forma que tenemos de estructurar, analizar y explicar todo aquello que nos rodea. Categorizamos la realidad desde las figuras geométricas más simples como los cuadrados, los triángulos o los polígonos regulares; hasta los más enrevesados fractales que son capaces de componer un copo de nieve, estructurar el crecimiento de algunas plantas o describir comportamientos en el universo.
	
	En este contexto de armonías y figuras que nos permite desarrollar la geometría, se sitúa el objeto de estudio en el que se centrará nuestro tema: la Semejanza. Más concretamente nos centraremos en la semejanza de triángulos y las propiedades que de ella se derivan. Aludiendo a las palabras del famoso matemático y escritor Eric Temple Bell (1883-1960): ``Ningún tema pierde tanto cuando se le divorcia de su historia como las Matemáticas'' \cite[p. 17]{Suma45}, y haciendo honor a ellas merece la pena comentar brevemente el contexto histórico en el que se desarrolla esta noción.
	
	Como se especifica en \citet{Suma58}, se tiene constancia de que ya en la antigua mesopotamia se precisaba de la resolución de problemas geométricos para el cálculo de figuras y superficies agrarias en los que la situación se describía mediante triángulos semejantes. Claro está, no con la notación actual, pero sí dando lugar a las primeras concepciones de esta noción. Más tarde, en el siglo VII a.C., Thales de Mileto dio los primeros avances en la proporción y la semejanza de triángulos mediante los teoremas de semejanza que se le atribuyen. Sin embargo, no fue hasta el 300 a.C. con la obra de Euclides, \textit{Los Elementos}, más concretamente el libro VI, cuando se establecen las definiciones y proposiciones asociadas a esta noción.
	
	Visto ahora que matemáticos de la talla de Thales o Euclides estudiaron esta noción y que, además, es uno de los temas que se tratan en los programas escolares, cabe preguntarse cómo se enfocará la didáctica de este tema para que pueda adaptarse a los distintos niveles de enseñanza. Con este objetivo, la doctora en matemáticas Julie Horoks observó que, en Francia, el número de alumnos que acuden a clases particulares para poder superar la asignatura de matemáticas está creciendo a medida que pasan los años. Ello suscitó el desarrollo de su tesis y de un artículo sobre este tema, particularizando en los contenidos y prácticas de enseñanza que se aplican en \textit{seconde}, concretamente, en la semejanza de triángulos.
	
	Así, en este trabajo nos dedicaremos a intentar analizar la investigación que Horoks llevó a cabo. Para ello se estructurará en tres partes bien diferenciadas:
	
	En el primer capítulo nos centraremos en el análisis del propio artículo de Horoks, que complementaremos en ciertas ocasiones con algunas alusiones hacia su tesis, donde se definirán los parámetros principales del análisis: los niveles de puesta en funcionamiento de los alumnos a la hora de enfrentarse a los ejercicios propuestos, y los conocimientos básicos que se habrán de adquirir una vez superada la noción. También se comentará algún ejemplo de cómo la autora lleva a cabo su investigación en relación al análisis de los ejercicios, terminando con las conclusiones a las que le llevó su estudio.
	
	El segundo capítulo tiene el objetivo de adaptar la investigación de Horoks a dos casos particulares de la docencia de la semejanza: uno en España y otro en Francia, a través de dos libros de texto de uso actual en los centros. Se analizarán, de este modo, algunos ejercicios dispuestos en ellos a partir del sistema \{tarea, desarrollo\} que propone la autora. Tras ello se dedica una sección para tratar las conclusiones a las que se ha llegado una vez desarrollado el grueso del trabajo, comparando con los resultados obtenidos por la propia autora en su investigación.
	
	Para terminar, se concluirá el trabajo con un último capítulo destinado a unas reflexiones más globales que versarán sobre el propio artículo, el trabajo llevado a cabo y algunas reflexiones personales que ha suscitado el trabajo a lo largo de su desarrollo y que se ha considerado relevante destacar.
	
	En definitiva, con este trabajo se pretende dar una visión reflexionada sobre la importancia de las prácticas docentes y el impacto que pueden llegar a ejercer sobre los alumnos. La trascendencia del papel del profesor que, sujeto en cierto modo a unas limitaciones impuestas por los tiempos y los programas académicos, puede ser notable sobre ellos.
	
	
	
	
	
	
	
	

%Capitulos:
%\setcounter{chapter}{0}
\titleformat{\chapter}[hang]
    {\titlerule\bf\Huge}{}{0em}{\chaptertitlename \ \thechapter. }[{\titlerule}]
    \pagestyle{empty}
\chapter[Investigación sobre la semejanza de Triángulos]{Investigación sobre la \\semejanza de Triángulos}
%\markboth{\MakeUppercase{Resumen}}{\MakeUppercase{Resumen}}
%Encabezado
\lhead[Trabajo Final de Máster]{Investigación sobre la semejanza de Triángulos}
\rhead[Investigación sobre la semejanza de Triángulos]{Trabajo Final de Máster}

\begin{dedication} 
 ``La enseñanza de las matemáticas es mucho más complicada de lo que esperabas,\\ a pesar de que ya esperases que fuera más complicada de lo que esperabas.''
	\rightline{{\rm --- Edward Griffith Begle (1914-1978)}}
\end{dedication}
\pagestyle{fancy}

	El artículo de ``Los triángulos semejantes en clase de \textit{seconde}: la enseñanza de los aprendizajes'' de \citet{Horoks} se trabajan los contenidos y desarrollos de la enseñanza sobre la semejanza de triángulos en el curso de \textit{seconde} (15-16 años)\footnote{En el Anexo A se ha elaborado una tabla descriptiva de la estructura de los cursos en Francia para facilitar la compresión y asociación a la estructura de nuestro país.}, basado en la Tesis Doctoral de la misma autora autora \citep{TH}. En ella intenta estudiar el porqué de la gran cantidad de alumnos de matemáticas que asisten a clases particulares fuera de la escuela, y cómo se vincula este hecho a la forma en la que los diferentes profesores imparten sus clases, analizando las influencias que se ejerce en los alumnos\footnote{Se interpreta que esto es debido a que se considera que los aprendizajes se evalúan a través del éxito o el fracaso en los exámenes, cuestión que, en ocasiones, lleva al alumno con resultados negativos a acudir a las clases particulares. Razón por la cuál analiza la influencia que tiene el profesor y su práctica docente en dicho aprendizaje.}. A continuación comentaremos cada uno de los aspectos de dicho artículo. %Así, el estudio se centra en qué ocurre en clase, al tiempo que se trata de aprender las carencias o vacíos que tengan ciertos alumnos en términos de las tareas propuestas.  interpretación
	
\section*{Inicio del trabajo de investigación}
\phantomsection\addcontentsline{toc}{section}{Inicio del trabajo de investigación}
\lfoot[\thepage]{Inicio del trabajo de investigación}
\rfoot[Inicio del trabajo de investigación]{\thepage}
%Cuadro teórico
	En el artículo se adopta la perspectiva según la cual los aprendizajes se hacen a través de \textbf{actividades} propuestas en clase por el profesor por medio de \textbf{tareas}. Estos términos tienen significados concretos: \textit{actividades}\footnote{Ligadas a lo que el profesor manda, sus intervenciones y la propia metodología de la que haga uso -(Robert, 2003) citado por \citet{Horoks}.} va referido a lo que ``el alumno hace o no''; la \textit{tarea}, a la ``descripción del trabajo matemático'' (que puede dar lugar a \textit{sub-tareas} debidas a las modificaciones de la original); y las \textit{prácticas} a  ``todo aquello que el profesor hace antes, durante y después de la clase''. 
	
	Para analizar las actividades y valorar su complejidad se contemplan cuatro tipos de adaptaciones posibles sobre los procedimientos de resolución de las tareas, denotadas por \textit{nivel de puesta en funcionamiento} (en adelante NPF), que abarcan las distintas tácticas o técnicas que pueden seguir los alumnos a la hora de enfrentarse a sus ejercicios:
	\begin{itemize} \label{NPF}
		\item Reconocimiento de modalidades de aplicación de un teorema;
		\item Necesidad de hacer cálculos intermedios;
		\item Necesidad de hacer elecciones;
		\item Necesidad de introducir etapas.
	\end{itemize}
	
	Otras variables consideradas fueron los conocimientos antiguos que puedan intervenir y la configuración geométrica en la que se sitúa el ejercicio.		
	
	Se establece que el aprendizaje parte de lo que realiza el profesor y su colaboración en la clase, junto con el correspondiente desarrollo que el alumno llevará a cabo en casa. Se tiene en cuenta, por tanto, el análisis de las tareas atendiendo al tiempo en el que trascurre cada una, las formas de trabajo adoptadas y las ayudas aportadas por el profesor con las modificaciones que éste lleve a cabo. A partir de estos resultados se pretende prever qué rendimiento tendrán los alumnos fuera de clase así como una idea aproximada de sus actividades potenciales.

%Elección de la noción	
	Para poder realizar una comparación en varias clases la autora buscó una noción a observar que tuviera un número razonable de sesiones y que minimizara el aporte de conocimientos anteriores (a fin de distinguir los aprendizajes ligados a las nuevas propiedades). La noción elegida fue la ``Semejanza de triángulos''. Esta noción desaparece del currículo en 1970 y reaparece en el 2000, %(poco antes de la tesina) 
	lo que le llevó a pensar que esto daría lugar a una nueva gestión de las clases en las que sería interesante estudiar la evolución de las enseñanzas en los nuevos programas. Además, no exime de una posible generalización de los resultados a otras nociones.
	
%Problemática
	La autora defiende la diferencia entre el trabajo realizado en clase y las actividades que realmente lleva a cabo el alumno en casa. Sin embargo, el desarrollo de las tareas en clase aporta una aproximación bastante fiable de las posibles actividades potenciales del alumno. Para observar dicho desarrollo se atiende a las variables fijadas por el marco teórico antes definido, a través de las cuales se medirá el impacto sobre el resultado de los alumnos con respecto a la noción trabajada. Gracias a este análisis se pretende determinar, para cada conocimiento evaluado, la totalidad del trabajo previo propuesto así como sus condiciones.
	
%Límites del cuadro
	Se destaca que los aprendizajes no se adquieren únicamente de aquello que ocurre en clase sino que necesitan tiempo. Se recalca que, de haberse podido tener en cuenta, habría sido muy interesante poder analizar este factor pero que, dadas las características y circunstancias de la experiencia, no ha sido posible. Al igual que éste, muchos otros factores externos no se han podido medir ni analizar, sin embargo, la elección de la ``Semejanza de triángulos'' permite acotar al máximo los conocimientos enseñados en las clases observadas.
	

\section*{Un breve análisis de la noción de triángulos\\ semejantes}
\phantomsection\addcontentsline{toc}{section}{Un breve análisis de la noción de triángulos semejantes}
\lfoot[\thepage]{Un breve análisis de la noción de triángulos semejantes}
\rfoot[Un breve análisis de la noción de triángulos semejantes]{\thepage}
%Análisis del programa de matemáticas 2000 para la clase de \textit{seconde}
	El artículo señala que en Francia, en el escalón previo a la secundaria -el \textit{collège} (11-14 años)- dentro de la noción de ``Triángulos semejantes'', se tratan la simetría axial y central, y las rotaciones y  traslaciones; sin abordar otras trasformaciones entre figuras semejantes que podrían ser útiles más adelante. En el \textit{lycée} (15-17 años), en cambio, dicha noción se estudia junto al teorema de Thales y solamente en \textit{seconde}. Este trabajo tan aislado lleva a que algunos profesores no lo aborden debido a su reducida utilidad.
	
	Atendiendo a lo que indica el programa de 2.º, se destaca que la definición dada no es la euclídea que, según \citet[p. 55]{Elementos}, cita como sigue\footnote{O, según la autora en su artículo, modernizando la definición: ``\textit{Dos triángulos son semejantes si sus ángulos son respectivamente iguales y sus lados proporcionales}''.}:
	
	\begin{quote}\small
		``1. Figuras rectilíneas semejantes son las que tienen los ángulos iguales uno a uno y proporcionales los lados que comprenden los ángulos iguales.\\
		2. (Dos) figuras están inversamente relacionadas cuando en cada una de las figuras hay razones antecedentes y consecuentes\footnote{Los términos antecedente y consecuente aluden al dividendo y al divisor, respectivamente, de una determinada razón de semejanza.}.''
	\end{quote}
	
	 Sino que se utiliza una condición de semejanza, ya que sólo se necesita de la igualdad de dos ángulos para obtener la de sus correspondientes triángulos. Así mismo, el artículo refleja las siguientes tres caracterizaciones de los Triángulos Semejantes que se trabajan en la docencia de esta noción:
	\begin{itemize}
		\item[C1:] Dos triángulos que tienen sus ángulos iguales, respectivamente, son semejantes.\label{C1}
		\item[C2:] Dos triángulos que tienen un ángulo idéntico comprendido entre lados respectivamente proporcionales son semejantes.
		\item[C3:] Dos triángulos que tienen sus lados respectivamente proporcionales son semejantes.
	\end{itemize}

	A estas caracterizaciones se añaden: el recíproco de esta tercera propiedad (P) y la razón entre áreas de triángulos semejantes (Área) que, junto con las anteriores, conforman los aprendizajes que se analizan en el artículo.
	
%Algunos comentarios sobre los documentos que acompañan los programas
	Los documentos que acompañan los programas asocian la semejanza de triángulos con los triángulos isométricos o iguales. Se cuida el estatus del enunciado, los métodos lógicos y una continuidad de lo adquirido en el \textit{collège} (cuyos conocimientos \textit{movilizables} han de convertirse en \textit{disponibles}). Puesto que no se estudian las transformaciones no isométricas, esto lleva a que la única introducción posible de este tema sea mediante la semejanza euclídea. Sin embargo, dicha ausencia dará pie a problemas y obstáculos tales como la localización de los vértices homólogos.
	
%El problema de la localización de vértices homólogos
	Esta dificultad no se encuentra reflejada en los programas escolares, lo cual no es siempre evidente, sobre todo cuando los vértices no vienen dados en orden en el enunciado o cuando un triángulo viene encajado en otro (Figura \ref{Figura1Horoks}). Analizando los libros y manuales se puede observar que no se mencionan teoremas o propiedades, aparentemente triviales, como que ``el lado más largo siempre se sitúa de forma opuesta al ángulo más grande''; sino que se opta por clasificar lados y ángulos en función de su tamaño y, después, deducir los homólogos. Son cuestiones simples pero, si no son tratadas explícitamente, los estudiantes no tienen por qué ser conscientes de ellas no permitiendo, además, el uso de técnicas que permitan una explicación provechosa por parte de los profesores. Esto lleva a la autora a preguntarse sobre la relación entre el déficit que puede aportar este hecho, junto con la novedad de la noción, al aprendizaje del alumno.
	
	\begin{figure}[h!]
		\centering
		\includegraphics[scale=0.7]{Figura1Horoks.png}
		\caption{Ejemplo  de triángulos semejantes encajados, extraído de (Horoks, 2008, p. 392).}
		\label{Figura1Horoks}
	\end{figure}
	
%Análisis de los ejercicios
	Por otro lado, el artículo utiliza once libros ofertados a los alumnos con el fin de observar las propiedades más trabajadas, caracterizando la dificultad de los ejercicios propuestos en relación a ellas. Se reflexiona sobre si la falta de ciertas tareas sobre algunos conceptos, frente a otras más trabajadas, conlleva unas dificultades mayores en casa, encontrando, en esto, un objeto de estudio interesante. Con ello se ha podido detectar el problema de las transformaciones que, estando presentes en la noción de triángulos semejantes, se encuentran ausentes en los programas escolares desembocando, con ello, en un problema dentro las clases observadas.
	

\section*{Algunos elementos de metodología y ejemplos de\\ \mbox{análisis}}
\phantomsection\addcontentsline{toc}{section}{Algunos elementos de metodología y ejemplos de análisis}
\lfoot[\thepage]{Algunos elementos de metodología y ejemplos de análisis}
\rfoot[Algunos elementos de metodología y ejemplos de análisis]{\thepage}

%Los datos recogidos y su tratamiento
	Para proceder a este estudio, la autora se basó en la observación de tres profesoras: las señoras B., P. y F. durante sus clases en \textit{seconde}. No en todas las aulas ha estado presente un observador, sino que algunas se realizaron únicamente a través de grabaciones de vídeo. Aunque no se tuvo en cuenta el perfil de las profesoras como docentes, sí se tuvo en cuenta el ``nivel'' del centro.  Gracias a los vídeos se clasificaron las variables antes definidas (NPF), las tareas y las actividades potenciales. Con ello fue posible comparar los ejercicios realizados en clase frente a los propuestos en los controles.
	
%Análisis del sistema \{tarea, desarrollo\} a partir de un ejemplo
	Con el fin de sintetizar los datos recopilados Horoks elabora tablas de \{tarea, desarrollo\} que permiten analizar el sistema. Esto es debido a que, para analizar el trabajo que se realiza en una clase, se considera que no basta con conocer las tareas que se proponen sino que el comprobar cómo se desarrollan éstas en el aula también es fundamental para el estudio que se pretende llevar a cabo. De esta manera, indicando el \textbf{ejercicio}, la \textbf{tarea} y el \textbf{desarrollo}, es posible desglosar: la \textit{configuración}, los \textit{conocimientos} (antiguos y nuevos) y el \textit{NPF} para la tarea, así como los \textit{tiempos de ``silencio''} y las \textit{ayudas} para el desarrollo de la cuestión. Esta organización del trabajo por parte del profesor será clave, ya que determinará las iniciativas propuestas a los alumnos y las posibles modificaciones de los ejercicios que puedan ser de interés. A través de este método se han analizado las clases de las tres profesoras, obteniendo así un balance del conjunto de las sesiones en las que se trataba la semejanza de triángulos.
	
%Análisis de las tareas del control y puesta en paralelo con lo que ocurre en clase
	Del mismo modo, también se realiza un análisis de las tareas y ejercicios que se disponen en los controles y se comparan con los realizados en clase%\footnote{\textit{Posibilidad de escribir como anexo los ejemplos a los que hace referencia el artículo, con la resolución pertinente de cada uno de ellos: Figura6,7; y quizá algunos cuadros}}
	, ya que en ellos no se recibe ninguna ayuda por parte del profesor. A través de tablas %\footnote{\textit{como la 7 y 8, añadir si es preciso}}
	 se pueden observar las similitudes entre los ejercicios, en particular aquellos cuyas variables son más próximas y que pueden prepararles mejor para los del control. A continuación se muestra un ejemplo de cómo se realiza dicho análisis para un ejercicio concreto (obtenido de \citet[p. 398-401]{Horoks} y \citet[p. 110]{TH}):%\footnote{Se ha constatado, antes del estudio, que las prácticas de los profesores eran estables y coherentes y que el desarrollo durante el capítulo era el mismo para cada profesor}.
	 
	 \begin{quote}\small
		``Ejercicio 9:\\
		$(C)$ es un círculo de centro $O$ y de radio $r$, $[AB]$ es un diámetro de $(C)$ y $P$ es el punto de $[AB]$ tal que $AP=1/3r$.\\
		Una recta $d$ distinta de la recta $(AB)$ pasa por $P$ y corta al círculo en dos puntos $M$ y $N$.\\
		1) Demostrar que los triángulos $APM$ y $NPB$ son triángulos semejantes.\\
		2) Deducir que $PM\times PN=5/9r^2$\\
		\begin{center}
			\includegraphics[scale=0.35]{Exercice9.PNG}
		\end{center}
		Solución:\\
		1) Se demuestra que los dos triángulos tienen dos ángulos iguales con la ayuda de los ángulos opuestos por el vértice y el teorema del ángulo inscrito. Así son, por tanto, semejantes.\\
		2) Dos triángulos semejantes tienen sus lados respectivamente proporcionales, por lo que aquí, tras la localización de los vértices homólogos, se obtiene: $MP/BP=PA/PN$. Obteniendo la igualdad buscada.
		\begin{center}
		\scalebox{0.84}[1]{
		\begin{tabular}{|c||l|l|l|l||l|l|l|l|}
\hline	\cellcolor[gray]{0.8}& \multicolumn{4}{|>{\cellcolor[cmyk]{0.9,0.3,0.2,0.1}}c||}{\textbf{Tarea}} & \multicolumn{4}{|>{\cellcolor[cmyk]{0.9,0.3,0.2,0.1}}c|}{\textbf{Desarrollo}}\\ \cline{2-9}
%
	\cellcolor[gray]{0.8} & \cellcolor[gray]{0.95}	&\multicolumn{2}{|c|}{\cellcolor[gray]{0.95}Conocimiento}	& \cellcolor[gray]{0.95} &	\cellcolor[gray]{0.95} & \multicolumn{3}{|c|}{\cellcolor[gray]{0.95}Ayudas}	\\ \cline{3-4}\cline{7-9}
%
	\multirow{-3}{*}{\cellcolor[gray]{0.8}ej} &	\multirow{-2}{*}{\cellcolor[gray]{0.95}Configuración} &	\cellcolor[gray]{0.95}Antiguo	&\cellcolor[gray]{0.95}Nuevo	&\multirow{-2}{*}{\cellcolor[gray]{0.95}NPF}&	\multirow{-2}{*}{\cellcolor[gray]{0.95}\begin{minipage}[l]{0.2cm}\noindent Tiempo\\ Silencio\end{minipage}}	& \cellcolor[gray]{0.95}Naturaleza & \cellcolor[gray]{0.95}Momento & \cellcolor[gray]{0.95}Forma  \\
%
\hline		\cellcolor[gray]{0.8}\begin{minipage}[l]{0.2cm}9\\ 1)\end{minipage}&  círculo	& \begin{minipage}[l]{1.28cm}\scriptsize\noindent Cálculo literal\end{minipage}& C1 &\begin{minipage}[l]{1.31cm}\scriptsize\noindent Cálculos\\ Intermedios \end{minipage}&	\begin{minipage}[l]{1.31cm}\scriptsize\noindent 29 mint\\ al ppio. \end{minipage}	&	Método	&\begin{minipage}[l]{0.2cm}\scriptsize\noindent Tras\\ Investigación\end{minipage} &\scriptsize Individual	\\
%		
\hline		\cellcolor[gray]{0.8}\begin{minipage}[l]{0.2cm}9\\ 2)\end{minipage}&  círculo	& \begin{minipage}[l]{1.28cm}\scriptsize\noindent Ángulo inscrito\end{minipage}& P &\begin{minipage}[l]{1.31cm}\scriptsize\noindent Cálculos\\ Intermedios \end{minipage}&	\begin{minipage}[l]{1.31cm}\scriptsize\noindent 15 mint\\ al ppio. \end{minipage}	&	Método	&\begin{minipage}[l]{0.2cm}\scriptsize\noindent Tras\\ Investigación\end{minipage} &\scriptsize Individual	\\ \hline
		\end{tabular}''
		}
		\end{center}
	 \end{quote}

%Interpretación de los resultados de los alumnos
	A partir de las respuestas obtenidas en el control, sus resultados durante el año y las propias apreciaciones de los profesores se clasificó a los alumnos en dos categorías: ``buenos'' y ``malos'' con el fin de comparar las influencias previas que puede tener, en los resultados del control\footnote{Ante esta categorización, el artículo es prudente puesto que no se sabe si son los buenos alumnos los que se aprovechan de ciertas elecciones del profesor, o si son los alumnos que se aprovechan de las opciones del profesor los que se convierten en buenos alumnos.}, todo aquello que se realiza en clase y fuera de ella. Se obtuvo, lógicamente, que si se utiliza un \textit{NPF} más elevado en el control que el trabajado en clase se vuelve un obstáculo para el alumno. %Parece ser que un \textit{NPF} más elevado en el examen que en el trabajo de clase se vuelve un obstáculo para gran parte del alumnado (sólo diez alumnos de los ``buenos'' han tenido éxito en una de estas cuestiones requeridas), mientras que un \textit{NPF} más simple en el control desemboca en unos resultados notablemente superiores\footnote{\textit{tablas 9,10,11 y 12 interesantes como resumen visual}}. 
 Otra constatación importante es la dificultad de la conexión entre la relación de dos triángulos semejantes cuando uno se encuentra contenido en otro  en una posición diferente a la de Thales, junto con la proporcionalidad de sus lados \footnote{Ejemplo Figura \ref{ExerciceExamen} obtenido de \citet[p. 402]{Horoks} y \citet[p. 131]{TH} en la que se ha de comprobar la semejanza entre los triángulos MNE y END.}, ya que supone localizar los vértices homólogos en ambos triángulos. Esto ofrece la ocasión de comprobar si los alumnos han asimilado esta tarea fundamental, dado que algunas soluciones pueden haberse obtenido tras la aplicación de técnicas que el profesor utilizaba en clase, tomándolas como ``recetas'', no porque se haya comprendido la noción.
 
 \begin{figure}
 	\centering
 	\includegraphics[scale=0.35]{ExerciceExamen.PNG}
 	\caption{Figura asociada a uno de los ejercicios de los controles analizados en el artículo.}
 	\label{ExerciceExamen}
 \end{figure}

	De este modo, la distribución del \textit{NPF} es una variable relevante del análisis de las tareas para evaluar los aprendizajes de los alumnos: periodos amplios de tiempo de investigación personal, sin intervenciones del profesor ni puestas en común, beneficia únicamente a los ``buenos'' alumnos. 
	
\section*{Algunos resultados}
\phantomsection\addcontentsline{toc}{section}{Algunos resultados}
\lfoot[\thepage]{Algunos resultados}
\rfoot[Algunos resultados]{\thepage}

%Los aprendizajes: las estrechas relaciones con lo que pasa en clase
	Obtenidos los resultados tras los diferentes controles, en lugar de centrar la atención en la clase que mejor resolvió las pruebas, el artículo reconoce una mayor importancia en la resolución de los propios ejercicios: cuáles han sido superados por la mayor parte de los alumnos o, por el contrario, cuál ha sido el que ha supuesto un mayor fracaso. Esto permitirá deducir qué tipo de propuestas, aportadas por los profesores, beneficiará más a los alumnos. 
	
	El ``ejercicio mejor resuelto'' está asociado a una preparación previa y concienzuda en el aula, con ejercicios de mayor o igual dificultad tanto en clase como en el control, y para los cuales se ha tenido tiempo de indagación y trabajo personal. En todos los casos se trata de ejercicios que precisan la caraterización C1, y parece indicarse que los conocimientos de los alumnos no son transferibles a \textit{NPF} más elevados\footnote{Crahay(2000) citado por \citet{Horoks} - ``Los alumnos, finalmente, aprenden... lo que se les enseña. Pero no es tan sencillo.''.}.
	
	En lo referente al ``ejercicio peor resuelto'' hay mayor diferencia entre las clases observadas. Se trata de preguntas que requerían una localización de los vértices homólogos no trivial. Esta dificultad no está ligada únicamente a la organización del trabajo en clase, sino que es un elemento importante de las decisiones del profesor en el aula ya que, en los casos observados, es éste el que normalmente se encarga de dirigirla e incidir en las dificultades. De hecho, en el caso de las dos profesoras que en clase sólo habían hecho ejercicios con las letras de los vértices homólogos de forma ordenada, se detectó dificultades en la resolución del ejercicio en el que, por primera vez, se encontraban desordenadas. 
	
	Por último, lo que caracteriza a los ejercicios del control donde se obtuvo los ``resultados menos homogéneos'' concierne a aquellos en los que se utilizaba una situación más adidáctica, dejando la mayor autonomía al alumno sin intervención del profesor durante o después de la tarea. Esto parece indicar, como se señaló antes, que el trabajo autónomo no beneficia de la misma forma a todos los alumnos.
	
	De este modo el artículo concluye que las decisiones del profesor pueden tener ciertas influencias sobre los aprendizajes de los alumnos, aunque hay que señalar que, en muchas ocasiones, estas decisiones vienen impuestas por los programas escolares, factor que es necesario tener en cuenta. De hecho, Horoks se sorprende ante la falta de comentarios o instrucciones sobre esta dificultad en dichos programas.
	
%Algunas observaciones sobre las prácticas de las enseñanzas en este capítulo
	Con respecto a las metodologías adoptadas por cada una de las profesoras, la autora concluye que la Sra. B realiza una simplificación de los ejercicios generalmente demasiado pronto, de forma que los alumnos acaban por realizar tareas muy dirigidas y aisladas. La Sra. P interviene en los ejercicios después de un tiempo de estudio autónomo por parte de los alumnos, dejándoles, potencialmente, la carga de las tareas complejas y elevando así, notablemente, el \textit{NPF} respecto de las otras clases. En último lugar, la Sra F. simplifica rápidamente las tareas complejas que se presentan en clase, sin que los alumnos puedan beneficiarse de un tiempo de estudio de cara a las preguntas más difíciles que trabajarán más tarde en casa.
	
	\begin{table}[h!]
		\centering
		\scalebox{0.9}[1]{
		\begin{tabular}{|>{\columncolor[gray]{0.7}}c|c|c|c|}
	\hline \cellcolor[gray]{0.6}\backslashbox{metodología}{profesora}& 	\cellcolor[gray]{0.8}Sra. B	 &	\cellcolor[gray]{0.8}Sra. P	 &	 \cellcolor[gray]{0.8}Sra. F\\
	\hline nivel e preparación  &	bajo	 & muy bueno & muy bueno\\
	\hline tareas de clase		& simples	 & complejas & complejas\\
	\hline carga de los alumnos
			en clase			& tareas simples & tareas complejas & tareas simples\\
	\hline en casa				& simples	 & simples o complejas  & complejas\\
	\hline en el control 		& complejas  & complejas 			& simples\\%\footnote{Razón de buenos resultados}\\
	\hline resultados de los
			alumnos				& medios y heterogéneos	 & buenos	& buenos\\
	\hline
		\end{tabular}
		}
		\caption{Metodología y tareas de las diferentes profesoras}
		\label{metodologias}
	\end{table}
	
	Como resumen ante estas metodologías podemos observar la Tabla \ref{metodologias}, que nos permite comprender el porqué de los buenos resultados obtenidos en la clase de la Sra. F, a pesar de que su metodología de trabajo no sea eficaz, como es el caso de la Sra. P. que, aplicando una dinámica de trabajo que introduce más tareas complejas, obtiene también resultados positivos. Cabe destacar que la diferencia entre lo que se propone y lo que es potencialmente realizado por los alumnos en la clase tiene, ciertamente, una influencia sobre la aptitud para resolver tareas complejas. Sin embargo, el artículo señala que este hecho no puede ser evaluado por el control propuesto, de manera que conforma un límite en su estudio. Es interesante constatar, aún así, que las prácticas son estables y coherentes a lo largo del capítulo, que la novedad de esta noción en los programas es un elemento a tener en cuenta y que otras investigaciones, como la llevada a cabo por Roditi (2005) (citado por \citet{Horoks}), obtuvieron resultados similares. 
	
\section*{Conclusión}
\phantomsection\addcontentsline{toc}{section}{Conclusión}
\lfoot[\thepage]{Conclusión}
\rfoot[Conclusión]{\thepage}

	Para concluir el artículo se vuelve a la preocupación de partida: determinar los déficits que pudieran tener ciertos alumnos en clase en lo referente a la noción de triángulos semejantes.
	
%Límites y/o faltas en la enseñanza de la noción
	La primera y clara carencia, que ha quedado manifiesta a lo largo del artículo, es la referente a la localización de los vértices homólogos. Ésta queda ligada al vacío en los programas educativos que no facilitan al profesor el poder ofrecer un método sistemático o una justificación matemática de su situación. Como cita \citet[p. 341]{TH}: ``A pesar de la libertad de elección de sus estrategias de enseñanza - libertad que ejercen en este capítulo\footnote{Haciendo referencia al capítulo de triángulos semejantes.}, con estrategias muy diferentes - la restricción del programa y los horarios son muy fuertes''. %A coalición con ello, en la tesis se destacan la gran cantidad de errores en cuanto a las propiedades concernientes a las longitudes de los lados o la confusión entre triángulos isométricos y semejantes. Además destaca el uso del álgebra
	
	Otra de las limitaciones destacables es la poca variedad de tareas propuestas, trabajando realmente poco ciertas aplicaciones. En cuanto a las ``actividades posibles`'' de los alumnos cabe remarcar que no siempre reflejan las tareas propuestas y que, el trabajo en pequeños grupos, puede frenar el aprendizaje de aquellos que no son demasiado buenos. 
	
%Vuelta a las clases particulares: un medio para ir más lejos en esta investigación
	Finalmente, el artículo advierte de la importancia que suscita el fenómeno de las clases particulares. Las cifras de estudiantes que acuden a este tipo de cursos en Francia aumenta vertiginosamente. El ser capaces de poder estudiar y analizar este tipo de clases hubiera sido un valioso recurso para completar la investigación y las reflexiones del artículo pero, desgraciadamente, señala la gran dificultad de poder llevarlo a cabo. Es un tema delicado para ciertos profesores y también alumnos, que no siempre admiten tener este tipo de ayudas.
	













%	Esquema de resumen
%	\begin{itemize}
%		\item Introducción
%		\item Puesta en marcha del trabajo de búsqueda
%		\begin{enumerate}
%			\item Cuadro teórico
%			\item Elección de la noción
%			\item Problemática
%			\item Límites del cuadro
%		\end{enumerate}
%		\item Un breve análisis de la noción de triángulos semejantes
%		\begin{enumerate}
%			\item Análisis del programa de matemáticas 2000 para la clase de \textit{seconde}
%			\item Algunos comentarios sobre los documentos que acompañan los programas
%			\item El problema de la localización de los vértices homólogos
%			\item Análisis de los ejercicios de los manuales escolares
%		\end{enumerate}
%		\item Algunos elementos de metodología y ejemplos de análisis
%		\begin{enumerate}
%			\item Los datos recogidos y su tratamiento
%			\item Análisis del sistema \{tarea, desarrollo\} a partir de un ejemplo
%			\item Análisis de las tareas del control y puesta en paralelo con lo que ocurre en clase
%			\item Interpretación de los resultados de los alumnos
%		\end{enumerate}			
%		\item Algunos resultados
%		\begin{enumerate}
%			\item Los aprendizajes: las estrechas relaciones con lo que pasa en clase
%			\item Algunas observaciones sobre las prácticas de las enseñanzas en este capítulo
%		\end{enumerate}
%		\item Conclusiones
%		\begin{enumerate}
%			\item Límites y/o faltas en la enseñanza de la noción
%			\item Vuelta a las clases particulares: un medio para ir más lejos en esta investigación
%		\end{enumerate}
%	\end{itemize}
	
	
	
	
%	
%	
%\chapter*{Cuestiones Relevantes}
%
%	A continuación se destacan algunas de las cuestiones más relevantes y que me han llamado la atención a lo largo del capítulo:
%
%\begin{itemize}
%	\item Vértices homólogos, ejemplos y problemáticas de ejemplos propuestos
%	\item Ejercicios y análisis de los programas educativos (libros de texto español/francés)
%	\item Niveles de puesta en funcionamiento (página \pageref{NPF})
%	\item Ventajas de la noción de triángulos semejantes
%	\item Límites del artículo (enmarcación, duración, imposibilidad de poner ejercicios en controles, imposibilidad de comprobar las clases particulares)
%	\item Sistema \{tarea, desarrollo\} empleado para recoger los datos
%	\item Relación: aquello que ocurre en clase $\longleftrightarrow$ aprendizajes del alumno y superación del control
%	\item Sistemas metodológicos de las profesoras (\ref{metodologias})
%\end{itemize}





\chapter{Análisis Francia-España}
%\phantomsection\addcontentsline{toc}{chapter}{Análisis Francia-España}
%Encabezado
\lhead[Trabajo Final de Máster]{Análisis Francia-España}
\rhead[Análisis Francia-España]{Trabajo Final de Máster}
\begin{dedication} 
\large ``Toma lo que hace falta, opera como debes\\ y obtendrás lo que deseas.''
	\rightline{{\rm --- Gottfried Leibniz (1646-1716)}}
\end{dedication}

	El artículo de Horoks realiza un buen análisis sobre el tema de la semejanza de triángulos y cuáles son sus características respecto a la enseñanza en Francia. El tema de la semejanza de triángulos es un concepto que se enseña tanto en Francia como en España y que, en ambos países, es una noción algo desubicada en los programas educativos. Parece interesante comprobar, pues, cuáles son las diferencias fundamentales del aprendizaje de esta noción en relación a ellos. De esta manera, se seleccionó un libro correspondiente al \textit{lycée} francés y otro al instituto español para poder compararlos.
	
\section{Libros y estructura}
\lfoot[\thepage]{Libros y estructura}
\rfoot[Libros y estructura]{\thepage}

	En este apartado realizaremos un análisis del tema de semejanza de cada uno de los libros que se han elegido para llevar a cabo la comparación:
	
	Por un lado tenemos el libro español: \citealp*{spa}\footnote{Desarrollo del tema en Anexo B}. Como se especifica en el título, este libro corresponde al cuarto curso de la ESO donde, según la actual Ley Orgánica 8/2013, de 9 de diciembre, para la mejora de la calidad educativa (LOMCE), se tiene estipulado que se estudien las nociones de semejanza pertinentes para nuestro estudio. Por otro lado, disponemos del libro francés: \citet{fr}\footnote{Desarrollo del tema en Anexo C}. Este libro, en cambio, pertenece al curso de $3^{\text{ème}}$ que se asemejaría al curso de $3.º$ de la ESO en España\footnote{Estructura de los cursos esquematizada en Anexo A.}. Aunque los libros no se ajustan al mismo nivel de enseñanza, hemos visto conveniente su comparación y estudio puesto que los niveles son los más cercanos posibles y el estudio de las nociones se corresponde. Hay que tener en cuenta que los desarrollos teóricos son diferentes según el país y que, a pesar de que las edades no sean las mismas, los contenidos que se abordan sí son lo más similares posible.
	
	Al mismo tiempo que se describen los temas, se señalarán las correspondencias con los cinco conocimientos de la semejanza de triángulos sobre los que versa el artículo de Horoks, así como alguna notación nueva que nos servirá para los análisis posteriores.
	
\subsection{Libro Español}\label{DefConN}

	El libro de \citet{spa} estructura el tema con puntos bien diferenciados: da una breve introducción histórica de la semejanza, explica el concepto de semejanza, trata la semejanza de triángulos, particulariza en los triángulos rectángulos, explica algunas aplicaciones interesantes y termina la teoría con la semejanza de rectángulos y sus peculiaridades, aludiendo al número áureo. Finalmente, propone una serie de ejercicios: primero tres problemas resueltos y, tras ellos, una lista de cincuenta y cuatro ejercicios para practicar divididos por sus nociones. A modo de cierre del capítulo, se proponen ejercicios para razonar a partir de lo estudiado y se termina con una autoevaluación de siete ejercicios. 
	
	El esquema del tema es el siguiente:
	\begin{itemize}
		\item[] \underline{6. Semejanza. Aplicaciones}
		\begin{itemize}
			\item[\ding{43}] \textbf{Breve contexto histórico}
			\begin{itemize}
				\item[\textbullet] Thales de Mileto y teorema de Thales
				\item[\textbullet]Eratóstenes y su predicción del radio de la Tierra
				\item[\textbullet] Semejanza en la antigua China
			\end{itemize}
			\item[1.] \textbf{Semejanza}
			\begin{itemize}
				\item[\textbullet] Definición de Semejanza%: ``misma forma'' - ángulos correspondientes iguales y longitudes proporcionales. Razón de semejanza.
%				\item[\ding{43}] Semejanza en la vida cotidiana
				\item[\textbullet] Definición de Escala%: cociente entre longitud de reproducción y correspondiente longitud de la realidad. Escala $=$ razón de semejanza.
%				\begin{itemize}
%					\item[-] Objetos grandes: 1:200 $\Rightarrow$ 1$cm$ reproducción $\rightarrow$ 200$cm$ reales.
%					\item[-] Objetos pequeños: 100:1 $\Rightarrow$ 100$cm$ reproducción $\rightarrow$ 1$cm$ real.
%				\end{itemize}
				\item[\textbullet] Relación entre áreas y volúmenes semejantes (conocimiento: Área)%: áreas $\rightarrow \ k^2$, volúmenes $\rightarrow \ k^3$ 
%				\item[\ding{43}] Ejercicios resueltos y propuestos
			\end{itemize}
			\item[2.] \textbf{Semejanza de Triángulos}
			\begin{itemize}
%				\item[\ding{43}] Estudio de triángulos semejantes $\Longrightarrow$ Fundamental el Teorema de \mbox{Thales}
				\item[\textbullet] Teorema de Thales%: Si $a$, $b$ y $c$ son rectas paralelas cortando a $r$ y $s$, entonces los segmentos que determinan son semejantes.
				\item[\textbullet] Triángulos semejantes
				\item[\textbullet] Triángulos en posición de Thales \textit{(conocimiento que denotaremos \mbox{por \textup{T})}}
				\item[\textbullet] Criterios de Semejanza de Triángulos
				\begin{itemize}
					\item Primer criterio: $\overset{\triangle}{ABC}\sim\overset{\triangle}{A'B'C'}$ \ si \ $\hat{A}=\hat{A'}$ y $\hat{B}=\hat{B'}$ \\(conocimiento: C1)
					\item Segundo criterio: $\overset{\triangle}{ABC}\sim\overset{\triangle}{A'B'C'}$ \ si \ $\dfrac{a'}{a}=\dfrac{b'}{b}=\dfrac{c'}{c}$ \\(conocimiento: C3)
					\item Tercer criterio: $\overset{\triangle}{ABC}\sim\overset{\triangle}{A'B'C'}$ \ si \ $\hat{A}=\hat{A'}$ y $\dfrac{b'}{b}=\dfrac{c'}{c}$ \\(conocimiento: C2)
				\end{itemize}
%				\item[\ding{43}] Ejercicios propuestos
			\end{itemize}
			\item[3.] \textbf{Semejanza de Triángulos Rectángulos}
			\begin{itemize}
				\item[\textbullet] Criterio de Semejanza de Triángulos Rectángulos y consecuencias
				\item[\textbullet] Teorema del Cateto \textit{(conocimiento que denotaremos \mbox{por \textup{TC})}}
				\item[\textbullet] Teorema de la Altura \textit{(conocimiento que denotaremos \mbox{por \textup{TA})}}
%				\item[\ding{43}] Ejercicios resueltos y propuestos
			\end{itemize}
			\item[4.] \textbf{Aplicación de la Semejanza}
			\begin{itemize}
				\item[\textsquare] Ejercicio resuelto: Tronco de Cono
				\item[\textsquare] Ejercicio resuelto: Superficie visible de la Tierra a una altura dada
%				\item[\ding{43}] Ejercicios propuestos
			\end{itemize}
			\item[5.] \textbf{Semejanza de Rectángulos. Aplicaciones}
			\begin{itemize}
				\item[\textbullet] Caracterización de la semejanza
				\item[\textbullet] Una hoja de papel A4 y el Rectángulo Áureo
%				\item[\ding{43}] Ejercicios propuestos
			\end{itemize}
			\item[\ding{43}] Ejercicios y problemas resueltos
			\item[\ding{43}] Ejercicios y problemas
			\item[\ding{43}] Aprende y reflexiona y autoevaluación
		\end{itemize}
	\end{itemize}
	
	Destacamos la definición de algunos conocimientos nuevos: T, TC y TA que añadimos a los descritos por Horoks en su artículo (C1, C2, C3, P y Área) ya que serán de utilidad más adelante en este trabajo.
	
\subsection{Libro Francés}
	
	El libro de \citet{fr} también organiza los puntos del capítulo de forma bien diferenciada, pero con una estructura distinta al modelo español: introduce el tema con un ejercicio tipo, define la semejanza y el vocabulario que se empleará, establece propiedades sobre los ángulos y las longitudes, relaciona las áreas de triángulos semejantes y, finalmente, establece la solución del ejercicio tipo de la introducción y propone diez ejercicios con sus soluciones.
	
	Esquemáticamente, el capítulo se estructura de la siguiente forma:
	\begin{itemize}
		\item[10.] \underline{Triángulos Semejantes}
		\begin{itemize}
			\item[\ding{43}] Introducción con ejercicio tipo
			\item[1.] Definición y vocabulario
			\begin{itemize}
				\item[\textbullet] Definición de Semejanza
				\item[\textbullet] Definición de lados, ángulos y vértices homólogos
			\end{itemize}
			\item[2.] Propiedades sobre los ángulos
			\begin{itemize}
				\item[\textbullet] Propiedad 1: $\overset{\triangle}{ABC}\sim\overset{\triangle}{A'B'C'}$ \ si \ $\hat{A}=\hat{A'}$, $\hat{B}=\hat{B'}$ y $\hat{C}=\hat{C'}$ \\ (conocimiento: C1)
				\item[\textbullet] Propiedad 2: recíproca de la Propiedad 1 
			\end{itemize}
			\item[3.] Propiedades sobre las longitudes
			\begin{itemize}
				\item[\textbullet] Propiedad 3: Si $\overset{\triangle}{ABC}\sim\overset{\triangle}{A'B'C'} \ \Longrightarrow \ \dfrac{a'}{a}=\dfrac{b'}{b}=\dfrac{c'}{c}$ \\(conocimiento: P)
				\item[\textbullet] Propiedad 4: recíproca de la propiedad 3 (conocimiento: C3)
			\end{itemize}
			\item[4.] Relación entre las áreas
			\begin{itemize}
				\item[\textbullet] Propiedad 5: Si $\overset{\triangle}{ABC}\sim\overset{\triangle}{A'B'C'}$ tal que $\dfrac{A'B'}{AB}=k \ \Longrightarrow \ \dfrac{\text{área}(A'B'C')}{\text{área}(ABC)}=k^2$ (conocimiento: Área)
				\item[\textbullet] Anécdota: Teorema de Thales
			\end{itemize}
			\item[\ding{43}] Solución del ejercicio tipo: Propiedades 1 y 2
			\item[\ding{43}] Ejercicios y soluciones
		\end{itemize}
	\end{itemize}
	
\section{Estudio}
	
	Analizados ambos libros hemos podido comprobar que hay grandes diferencias entre uno y otro, sobre todo en cuanto a contenido en ambos temas:
	
	En el libro español podemos observar que al tratar la semejanza no se centran únicamente en la semejanza de triángulos, sino que la definen en general a partir del concepto de ``escala''. Hecho esto, particularizan en la semejanza de triángulos y, más concretamente, en la de los triángulos rectángulos; y, finalmente, termina con la semejanza en rectángulos, aportando curiosas aplicaciones de su estudio. 
	
	El libro francés, por el contrario, únicamente trata la semejanza de triángulos centrándose en sus propiedades principales (las caracterizaciones en las que se centra el artículo de Horoks: C1, C2, C3, P y Área) para resolver los problemas ``tipo'' de esta clase de noción.
	
	Un detalle importante de este desarrollo en el libro español es que, al trabajar con la semejanza de triángulos rectángulos más detenidamente, utiliza este concepto como demostración del Teorema del Cateto y el Teorema de la Altura. Considero que la importancia de dar estas dos nociones es substancial ya que, con un procedimiento simple y sencillo, se es capaz de probar dos resultados altamente útiles y efectivos -como se puede comprobar en los ejercicios del capítulo- y que están al alcance de los alumnos. Les permite comenzar a ser conscientes de la importancia de una prueba, que no por ser eficaz ha de ser compleja.
	
	Otro detalle destacable es la gran diferencia que podemos encontrar en cuanto al número de ejercicios propuestos y su contenido:
	
	En el libro español hay un total de cincuenta y cuatro ejercicios propuestos al final del tema, a los que hay que sumar los ejercicios planteados durante el capítulo, más los ejercicios de la reflexión y la autoevaluación. Esto hace un total de ochenta y ocho ejercicios, entre propuestos y resueltos, para que el alumno pueda realizar. Dada la extensión del tema y el tiempo que se tiene para impartirlo, es difícil que todos esos ejercicios puedan llevarse a cabo por algún alumno. Por añadidura, si algún alumno llegara a intentarlo, acabaría aburrido y desmotivado pues los ejercicios son altamente repetitivos: de los cincuenta y cuatro ejercicios propuestos al final del tema, diecisiete son aplicaciones del Teorema de Thales; once, aplicaciones del Teorema del Cateto y de la Altura y trece, aplicación directa de la escala y la razón de semejanza. Es improbable que a un estudiante le pueda ser eficaz intentar, siquiera, realizar todos estos ejercicios. Sin embargo, y por otro lado, el no enfocar tal cantidad de ejercicios como ``obligatorios para poder aprobar el examen'' sino como un listado variado donde el estudiante pueda elegir, tenga opción al error y una diversidad de ejercicios que le permitan practicar y subsanar dichos errores, pueda ser positivo.
	
	Por su parte, el libro francés plantea únicamente diez ejercicios, más el ejercicio tipo que resuelve al final. Si nos centramos en los diez ejercicios propuestos tras la resolución anterior: los dos primeros tratan la teoría preguntando sobre las nociones de forma teórica (verdadero/falso y elección de respuesta) y, tras ellos, se piden comprobaciones de semejanza de triángulos en diferentes contextos geométricos (circunferencias, paralelogramos, a través de una bisectriz, por Thales...) terminando con un problema de investigación un poco más complejo sobre semejanza. Estructurando así los ejercicios sí que es cierto que los alumnos no tienen tanta variedad como en el caso del libro español pero, siendo pocos y pudiéndose trabajar incluso en clase, es más motivador para el alumnos el llevarlos a cabo. Pese a ello, considero que sería propicio plantear más ejercicios, no llegando quizás a la cantidad que se presentan en \citet{spa}, pero sí los suficientes como para que el alumno pueda poner en práctica lo aprendido y, como se ha especificado antes, tenga opción a equivocarse y poder rectificar con problemas diferentes. Citando a \citet[p. 243]{Chevallard}: ``Estudiar problemas es un medio que permite crear y poner en marcha una técnica relativa a los problemas del mismo tipo, técnica que será a continuación el medio para resolver de manera casi rutinaria los problemas de este tipo''.
	
\section{Sistemas \{tarea, desarrollo\}}

	Para poder observar qué tipos de conocimiento entran en juego de cara a la resolución de los distintos tipos de problemas propuestos en ambos libros, analizaremos algunos de los ejercicios que en ellos aparecen. Se llevará a cabo un estudio por medio del sistema \{tarea, desarrollo\} propuesto en \citet{Horoks} para dos clases de ejercicio: uno ``tipo'' y otro ``más geométrico'' destinado a presentar aplicaciones de la semejanza en otro tipo de figuras geométricas. Dicha distinción es interesante ya que, como determina Horoks en su estudio, la configuración geométrica en la que se presente el ejercicio, junto con los conocimientos previos asociados, es una variable destacable en el análisis que se realice.
	
	Para realizar el análisis nos basaremos en los NPF, los conocimientos que desarrolla el artículo (C1, C2, C3, P y Área) y los que se definieron en la sección \ref{DefConN} (T, TC y TA) en lo referente al apartado de ``tarea''. La parte de desarrollo, en cambio, ha conllevado un límite en nuestro marco de trabajo ya que no ha sido posible llevar dicho factor al aula. Por ello, el análisis que se expone a continuación se ocupará únicamente de la parte correspondiente a la tarea, tal y como lo lleva a cabo Horoks siguiendo el ejemplo que se especificó al comienzo de este trabajo: Enunciado, solución y tabla de análisis. % puesto que no hemos podido poner en práctica dichos ejercicios en el entorno de un aula se propondrá un desarrollo que se considere adecuado y eficiente, basándonos en la complejidad de la resolución de los ejercicios que llevemos a cabo, en las observaciones de Horoks sobre las prácticas de las tres profesoras que analiza ella en su artículo, así como de las guías de algunos de los profesores de matemáticas del IES Alfonso X ``El Sabio'' de Murcia, con los que he podido colaborar.
	
\subsection{Ejercicios tipo}

\subsubsection*{Libro Español}
	A continuación se muestra un ejercicio tipo del libro de texto \citet{spa}, así como una breve solución. En concreto, presentamos el ejercicio 26, que podemos encontrar en la página 137 de dicho libro:%, reflejado en la Tabla \ref{ej26p137}. El ejercicio se propone de la siguiente manera:
	
	\begin{quote}\small
	\begin{multicols}{2}
		\begin{minipage}{7cm}
		\begin{flushleft}
			Para medir la altura de la casa, Álvaro, de $165cm$ de altura, se situó a $1,5cm$ de la verja y tomó las medidas indicadas. ¿Cuánto mide la casa?
		\end{flushleft}
		\end{minipage}
		
	\columnbreak
	
	\begin{flushright}
		\includegraphics[scale=0.25]{Ej26p137.png}
	\end{flushright}
	\end{multicols}
	\end{quote}
	
\begin{proof} Para resolver este ejercicio será necesario realizar un pequeño dibujo esquemático de la situación, como el que observamos más abajo, de manera que tenemos dos triángulos en posición de Thales (conocimiento T). Así, podemos aplicar la igualdad $$\dfrac{25+1,5}{1,5}=\dfrac ha$$ donde, calculando $a$ queda:
	
	\begin{center}
\begin{tabular}{>{\centering\arraybackslash}c>{\centering\arraybackslash}c}
		\multirow{5}{*}{\includegraphics[scale=0.42]{Ej26p137s111.png}} & 
			$a=3,5-1,65=1,85m$\\&\\
			& $\dfrac{25+1,5}{1,5}=\dfrac{h}{1,85}\rightarrow h=\dfrac{26,5\cdot1,85}{1,5}=32,68m$\\ &\\
			&$\text{Altura de la casa: }32,68+1,65=34,33m$
	\end{tabular}
	\end{center}
	
\end{proof}
	
	Una vez resuelto el ejercicio podemos observar los resultados del análisis resumidos en la Tabla \ref{ej26p137}.
	
	\begin{table}[h!]
	\centering
	\scalebox{1}[1]{
		\begin{tabular}{|c||l|c|l|c|}
\hline	\cellcolor[gray]{0.8}& \multicolumn{4}{|>{\cellcolor[cmyk]{0.9,0.3,0.2,0.1}}c|}{\textbf{Tarea}} \\ \cline{2-5}
%
	\cellcolor[gray]{0.8} & \cellcolor[gray]{0.95}	&\multicolumn{2}{|c|}{\cellcolor[gray]{0.95}Conocimiento}	& \cellcolor[gray]{0.95} 	\\ \cline{3-4}
%
	\multirow{-3}{*}{\cellcolor[gray]{0.8}ej} &	\multirow{-2}{*}{\cellcolor[gray]{0.95}Configuración} &	\cellcolor[gray]{0.95}Antiguo	&\cellcolor[gray]{0.95}Nuevo	&\multirow{-2}{*}{\cellcolor[gray]{0.95}NPF}\\
%
\hline		\cellcolor[gray]{0.8}26	&  \begin{minipage}[l]{0.2cm}\small Dibujo\end{minipage}	& \begin{minipage}[l]{3cm}\small\noindent Descomposición triángulos rectángulos\end{minipage}&\begin{minipage}[l]{0.2cm}\small\noindent\textbullet T\\ \noindent\textbullet C3\end{minipage}&\begin{minipage}[l]{3.5cm}\small\noindent\textbullet Cálculos\\ Intermedios\\ \noindent\textbullet Etapas\end{minipage}\\ \hline		
		\end{tabular}
		}
		\caption{Tabla Sistema \{tarea,desarrollo\} ejercicio 26, p.137}\label{ej26p137}
		\end{table}
%	\begin{table}[h!]
%	\scalebox{0.92}[1]{
%		\begin{tabular}{|c||l|l|l|l||l|l|l|l|}
%\hline	\cellcolor[gray]{0.8}& \multicolumn{4}{|>{\cellcolor[cmyk]{0.9,0.3,0.2,0.1}}c||}{\textbf{Tarea}} & \multicolumn{4}{|>{\cellcolor[cmyk]{0.9,0.3,0.2,0.1}}c|}{\textbf{Desarrollo}}\\ \cline{2-9}
%%
%	\cellcolor[gray]{0.8} & \cellcolor[gray]{0.95}	&\multicolumn{2}{|c|}{\cellcolor[gray]{0.95}Conocimiento}	& \cellcolor[gray]{0.95} &	\cellcolor[gray]{0.95} & \multicolumn{3}{|c|}{\cellcolor[gray]{0.95}Ayudas}	\\ \cline{3-4}\cline{7-9}
%%
%	\multirow{-3}{*}{\cellcolor[gray]{0.8}ej} &	\multirow{-2}{*}{\cellcolor[gray]{0.95}Configuración} &	\cellcolor[gray]{0.95}Antiguo	&\cellcolor[gray]{0.95}Nuevo	&\multirow{-2}{*}{\cellcolor[gray]{0.95}NPF}&	\multirow{-2}{*}{\cellcolor[gray]{0.95}\begin{minipage}[l]{0.2cm}\noindent Tiempo\\ Silencio\end{minipage}}	& \cellcolor[gray]{0.95}Naturaleza & \cellcolor[gray]{0.95}Momento & \cellcolor[gray]{0.95}Forma  \\
%%
%\hline		\cellcolor[gray]{0.8}26	&  \begin{minipage}[l]{0.2cm}\scriptsize\noindent\textbullet Thales\\ \noindent\textbullet Esquema/\\ dibujo\end{minipage}	& \begin{minipage}[l]{1.5cm}\scriptsize\noindent Descomp.  triángulos rectángulos\end{minipage}&\begin{minipage}[l]{0.2cm}\scriptsize\noindent\textbullet T\\ \noindent\textbullet C3\end{minipage}&\begin{minipage}[l]{1.31cm}\scriptsize\noindent\textbullet Cálculos\\ Intermedios\\ \noindent\textbullet Etapas\end{minipage}&	5 mints		&	Método	&\begin{minipage}[l]{0.2cm}\scriptsize\noindent Tras\\ Investigación\end{minipage} & \scriptsize Individual	\\ \hline		
%		\end{tabular}
%		}
%		\caption{Tabla Sistema \{tarea,desarrollo\} ejercicio 26, p.137}\label{ej26p137}
%		\end{table}
		
		Como podemos observar, la tarea se muestra con una breve descripción de un dibujo con dos triángulos en posición de Thales, lo cual lleva a la resolución que hemos marcado como ``tipo''. Su resolución, por tanto, debería ser sencilla, sin embargo, la forma en la que está expresado el ejercicio, junto con el dibujo que aporta detalles demasiado superfluos, puede llevar al alumno a confusión ya que interpretar el ejercicio para llegar al esquema que se muestra en la solución no es sencillo y, además, se ha de tener un nivel elevado de manejo de este tipo de situaciones. Debido a ello, como conocimiento previo será necesario reconocer esta estructura, descomponiendo la visualización de la imagen en los diferentes triángulos de la solución. En este punto se podrán emplear los nuevos conocimientos - C1 y P - resolviendo el ejercicio por etapas y con pequeños cálculos intermedios.
		
		%En lo referente al desarrollo para este ejercicio, puesto que se trata de un ejercicio tipo y, por tanto, sencillo, su resolución no será compleja cuando los alumnos hayan realizado ya algunos parecidos. Por ello, bastará con un periodo de cinco minutos de trabajo del alumno, tras los cuales, si es necesario, se guiará a aquellos que tengan algún problema centrándonos en el método de resolución; pero de forma individualizada, pues se espera que el conjunto de la clase no tenga problemas.
	

\subsubsection*{Libro Francés}

	Veamos ahora un ejemplo de \citet{fr}. 
	
	\paragraph{\underline{5. Cálculos de longitudes.}}Centrémonos en el ejercicio 5 que podemos encontrar en la página 261 del citado libro. Se tiene lo siguiente:
	\vspace{0.5cm}
	\begin{multicols}{2}
		\begin{minipage}{9cm}
	
	Sobre la figura siguiente, los triángulos $ABC$ y $BA'C'$ son semejantes. Con la ayuda de los datos de esta figura, responde a las siguientes cuestiones:
	\begin{enumerate}
		\item Calcular la longitud de $BA'$
		\item Calcular la longitud de $BC$
	\end{enumerate}
	\end{minipage}
		
	\columnbreak
	
%	\begin{center}
%		\includegraphics[scale=1]{Ej5p261.png}
%	\end{center}
\definecolor{zzttqq}{rgb}{0.6,0.2,0.}
\begin{tikzpicture}[line cap=round,line join=round,>=triangle 45,x=0.9cm,y=0.9cm]
\clip(-0.4208460561053181,-0.6538734257527625) rectangle (12.562054335131556,5.634366155046311);
\fill[color=zzttqq,fill=zzttqq,fill opacity=0.10000000149011612] (3.,0.) -- (7.,0.) -- (3.9794875000000003,3.046408416044006) -- cycle;
\fill[color=zzttqq,fill=zzttqq,fill opacity=0.10000000149011612] (3.9794875000000003,3.046408416044006) -- (6.009334715321547,2.099138506012362) -- (6.35575378388172,4.877626193395177) -- cycle;
\draw [color=zzttqq] (3.,0.)-- (7.,0.);
\draw [color=zzttqq] (7.,0.)-- (3.9794875000000003,3.046408416044006);
\draw [color=zzttqq] (3.9794875000000003,3.046408416044006)-- (3.,0.);
\draw [color=zzttqq] (3.9794875000000003,3.046408416044006)-- (6.009334715321547,2.099138506012362);
\draw [color=zzttqq] (6.009334715321547,2.099138506012362)-- (6.35575378388172,4.877626193395177);
\draw [color=zzttqq] (6.35575378388172,4.877626193395177)-- (3.9794875000000003,3.046408416044006);
\draw [shift={(3.,0.)}] plot[domain=0.:1.259713317181554,variable=\t]({1.*0.47402406126574403*cos(\t r)+0.*0.47402406126574403*sin(\t r)},{0.*0.47402406126574403*cos(\t r)+1.*0.47402406126574403*sin(\t r)});
\draw [shift={(3.9794875000000003,3.046408416044006)}] plot[domain=4.401305970771347:5.493518790998627,variable=\t]({1.*0.30375904219655825*cos(\t r)+0.*0.30375904219655825*sin(\t r)},{0.*0.30375904219655825*cos(\t r)+1.*0.30375904219655825*sin(\t r)});
\draw [shift={(3.9794875000000003,3.046408416044006)}] plot[domain=4.401305970771347:5.4935187909986265,variable=\t]({1.*0.3859035249165582*cos(\t r)+0.*0.3859035249165582*sin(\t r)},{0.*0.3859035249165582*cos(\t r)+1.*0.3859035249165582*sin(\t r)});
\draw [shift={(3.9794875000000003,3.046408416044006)}] plot[domain=-0.43663034610513485:0.6565729649128689,variable=\t]({1.*0.25857945415581607*cos(\t r)+0.*0.25857945415581607*sin(\t r)},{0.*0.25857945415581607*cos(\t r)+1.*0.25857945415581607*sin(\t r)});
\draw [shift={(3.9794875000000003,3.046408416044006)}] plot[domain=-0.43663034610513485:0.6565729649128711,variable=\t]({1.*0.32288132591701085*cos(\t r)+0.*0.32288132591701085*sin(\t r)},{0.*0.32288132591701085*cos(\t r)+1.*0.32288132591701085*sin(\t r)});
\draw [shift={(6.009334715321547,2.099138506012362)}] plot[domain=1.4467574054890744:2.7049623074846583,variable=\t]({1.*0.42349128661882796*cos(\t r)+0.*0.42349128661882796*sin(\t r)},{0.*0.42349128661882796*cos(\t r)+1.*0.42349128661882796*sin(\t r)});
\draw (2.604734208982927,0.12510059772144944) node[anchor=north west] {$\mathbf{A}$};
\draw (3.428866146861442,3.658707399858092) node[anchor=north west] {$\mathbf{B}$};
\draw (7.064078256407766,0.19283746932790266) node[anchor=north west] {$\mathbf{C}$};
\draw (5.946419874901288,2.24752257472365) node[anchor=north west] {$\mathbf{A'}$};
\draw (6.307683190135705,5.340839711418346) node[anchor=north west] {$\mathbf{C'}$};
\draw (2.582155251780776,1.931417173893535) node[anchor=north west] {$3,2$};
\draw (4.896498365001262,0.03478476891284516) node[anchor=north west] {$4$};
\draw (4.896498365001262,4.539286730741984) node[anchor=north west] {$3$};
\draw (6.206077882726025,3.760312707267772) node[anchor=north west] {$2,8$};
\end{tikzpicture}
	\end{multicols}

\begin{proof} Como se puede comprobar en la página 266 del mismo libro, la solución a este ejercicio se basa en la aplicación de la propiedad P. Además, se especifica que la necesidad de conocer los vértices homólogos para poder aplicarla, resaltando la importancia de este hecho. Resolviendo el ejercicio tenemos que:

	Basándonos en los datos del enunciado, $\hat{BAC}=\hat{BA'C'}$ y $\hat{ABC}=\hat{A'BC'}$, por lo que los lados homólogos son: $AB$ y $A'B$, $AC$ y $A'C'$, y $BC$ y $BC'$. Podemos, por tanto, escribir $$\dfrac{A'B}{AB}=\dfrac{C'B}{BC}=\dfrac{A'C'}{AC} \ \Longrightarrow \ \dfrac{A'B}{3,2}=\dfrac{3}{BC}=\dfrac{2,8}{4}$$
	
	\begin{enumerate}
		\item Para determinar la longitud $BA'$ tomamos $\dfrac{A'B}{3,2}=\dfrac{2,8}{4}$ con lo que $$A'B=\dfrac{2,8\times3,2}{4}=2,24$$
		\item Para determinar la longitud $BA'$ tomamos $\dfrac{3}{BC}=\dfrac{2,8}{4}$ con lo que $$BC=\dfrac{3\times4}{2,8}=4,29$$
	\end{enumerate}	
\end{proof}	
	
	Aludiendo de nuevo a las variables que conforman la tarea propuesta, analizamos el ejercicio con la Tabla \ref{ej5p261}.
	
	\begin{table}[h!]
	\centering
	\scalebox{1}[1]{
		\begin{tabular}{|c||l|c|l|c|}
\hline	\cellcolor[gray]{0.8}& \multicolumn{4}{|>{\cellcolor[cmyk]{0.9,0.3,0.2,0.1}}c|}{\textbf{Tarea}} \\ \cline{2-5}
%
	\cellcolor[gray]{0.8} & \cellcolor[gray]{0.95}	&\multicolumn{2}{|c|}{\cellcolor[gray]{0.95}Conocimiento}	& \cellcolor[gray]{0.95} 	\\ \cline{3-4}
%
	\multirow{-3}{*}{\cellcolor[gray]{0.8}ej} &	\multirow{-2}{*}{\cellcolor[gray]{0.95}Configuración} &	\cellcolor[gray]{0.95}Antiguo	&\cellcolor[gray]{0.95}Nuevo	&\multirow{-2}{*}{\cellcolor[gray]{0.95}NPF}\\
%
\hline		\cellcolor[gray]{0.8}5	&  \begin{minipage}[l]{0.2cm}\small Esquema/\\ dibujo\end{minipage}	& \begin{minipage}[l]{3cm}\small\noindent Cálculo literal\end{minipage}&\begin{minipage}[l]{0.2cm}\small P\end{minipage}&\begin{minipage}[l]{3.5cm}\small\noindent\textbullet Reconocer teorema\\ \noindent\textbullet Necesidad elección\end{minipage}\\ \hline		
		\end{tabular}
		}
		\caption{Tabla Sistema \{tarea,desarrollo\} ejercicio 5, p. 261}\label{ej5p261}
		\end{table}
%	\begin{table}[h!]
%	\scalebox{0.92}[1]{
%		\begin{tabular}{|c||l|l|l|l||l|l|l|l|}
%\hline	\cellcolor[gray]{0.8}& \multicolumn{4}{|>{\cellcolor[cmyk]{0.9,0.3,0.2,0.1}}c||}{\textbf{Tarea}} & \multicolumn{4}{|>{\cellcolor[cmyk]{0.9,0.3,0.2,0.1}}c|}{\textbf{Desarrollo}}\\ \cline{2-9}
%%
%	\cellcolor[gray]{0.8} & \cellcolor[gray]{0.95}	&\multicolumn{2}{|c|}{\cellcolor[gray]{0.95}Conocimiento}	& \cellcolor[gray]{0.95} &	\cellcolor[gray]{0.95} & \multicolumn{3}{|c|}{\cellcolor[gray]{0.95}Ayudas}	\\ \cline{3-4}\cline{7-9}
%%
%	\multirow{-3}{*}{\cellcolor[gray]{0.8}ej} &	\multirow{-2}{*}{\cellcolor[gray]{0.95}Configuración} &	\cellcolor[gray]{0.95}Antiguo	&\cellcolor[gray]{0.95}Nuevo	&\multirow{-2}{*}{\cellcolor[gray]{0.95}NPF}&	\multirow{-2}{*}{\cellcolor[gray]{0.95}\begin{minipage}[l]{0.2cm}\noindent Tiempo\\ Silencio\end{minipage}}	& \cellcolor[gray]{0.95}Naturaleza & \cellcolor[gray]{0.95}Momento & \cellcolor[gray]{0.95}Forma  \\
%%
%\hline		\cellcolor[gray]{0.8}5	&  \begin{minipage}[l]{0.2cm}\scriptsize Esquema/\\ dibujo\end{minipage}	& \begin{minipage}[l]{1.28cm}\scriptsize\noindent Cálculo literal\end{minipage}&\begin{minipage}[l]{0.2cm}\scriptsize\noindent P\end{minipage}&\begin{minipage}[l]{1.31cm}\scriptsize\noindent\textbullet Reconocer\\ teorema\\ \noindent\textbullet Necesidad\\ elección\end{minipage}&	5 mints		&	Método	&\begin{minipage}[l]{0.2cm}\scriptsize\noindent Tras\\ Investigación\end{minipage} &\scriptsize Colectivo	\\ \hline		
%		\end{tabular}
%		}
%		\caption{Tabla Sistema \{tarea,desarrollo\} ejercicio 5, p. 266}\label{ej5p266}
%		\end{table}
		
	En este ejemplo, la tarea describe la situación mediante un dibujo esquemático que representa el enunciado. Para su resolución podemos ver que sólo es necesario realizar los cálculos literales consecuentes de la aplicación de la propiedad P, una vez se elijan adecuadamente los lados correspondientes y se reconozca el teorema subyacente a esta propiedad. El ejercicio corresponde, por tanto, a lo que hemos categorizado como ``tipo'' debido tanto a su presentación como a su solución, ya que la tarea se presenta de forma clara y alude a un nivel muy simple, con el fin de que el alumno ponga en práctica lo aprendido, sin interés de ir más allá de la propia aplicación del conocimiento señalado.
	
	%Con respecto al desarrollo, nos encontramos ante un nuevo ejercicio tipo, de forma que no se esperan grandes problemáticas por parte del alumnado y bastará con un tiempo de trabajo propio de cinco minutos. Una vez trascurrido este tiempo, a modo de corrección se darán pautas colectivas sobre el método de resolución de este tipo de ejercicios.
	
	
\subsection{Aplicaciones a otras figuras geométricas}

	Los dos ejercicios que acabamos de ver representan los ejercicios más típicos que se requieren a los alumnos. Sin embargo, estas nociones permiten desarrollar algunos más complejos en los que poder aplicar la semejanza de triángulos y que están contenidos dentro de un marco de figuras geométricas, donde la utilidad de la semejanza se hace indispensable. Como bien expresa la propia autora en su tesis:
	
	\begin{quote}\small
		``Los ejercicios donde los triángulos semejantes son un útil para demostrar propiedades - generalmente sobre las longitudes y las áreas - son bastante raros [...] podrían, eventualmente, presentar un sustituto válido para reemplazar las trasformaciones ausentes, y volver a dar una dimensión intuitiva de esta localización\footnote{Aludiendo a las dificultades de la localización de los vértices homólogos.}, abstracta para los alumnos.''\cite[p. 340]{TH}.
	\end{quote}
	
	Por ello, se ha analizado también este tipo de ejercicios a fin de poder comprobar qué conocimientos se requieren en cada país ya que, como veremos, en España se hace gran uso de las propiedades T, TC y TA en dichos contextos. Veamos un ejemplo de este supuesto en cuanto a lo que abarcan los contenidos de la semejanza.
	
\subsubsection{Libro Español}

	En el caso del libro Español hay dos ejercicios basados en los teoremas del Cateto y de la Altura cuya demostración se basa, precisamente, en la semejanza de los triángulos rectángulos construidos a partir de un triángulo rectángulo mayor que los contiene. Estos ejercicios muestran muy bien la aplicación de la semejanza en cuerpos geométricos de muy distinta índole.
	
	\paragraph{\underline{Teorema del Cateto}} Para ejemplificar un caso de este tipo de ejercicios analizaremos el ejercicio 40 de la página 138 de \citet{spa}.
	
	\begin{multicols}{2}
		
		\begin{minipage}{7cm}
			Sobre una esfera de $20cm$ de radio se encaja un cono de $30cm$ de altura.
Halla el área del casquete esférico que determina el cono.
		\end{minipage}
		
	\columnbreak
	
	\begin{center}
		\includegraphics[scale=0.3]{Ej40p138.png}
	\end{center}
	\end{multicols}
	
\begin{proof} En primer lugar realizamos un pequeño dibujo esquemático del ejercicio para determinar cómo proceder, observando que podemos construir un triángulo rectángulo $ABC$ dividido en otros dos, también rectángulos, donde poder aplicar el Teorema del Cateto:

\begin{multicols}{2}
		
		\begin{center}
		\includegraphics[scale=0.35]{Ej40p138s1.png}
		\vspace{-2cm}
		\begin{align*}
				\hspace{6.85cm}x^2&+30x-400=0 \ \rightarrow \ x=\dfrac{-30\pm50}{2}=\begin{cases}-40cm\\ 10cm\end{cases}\\
				%\hspace{0}\text{Altura casquete}&=20-10=10cm\\
				%\text{Área casquete}&=2\pi Rh=400\pi cm^2
			\end{align*}
	\end{center}
				
	\columnbreak
	
	\begin{minipage}{7cm}
			Para calcular $x$ bastará aplicar el Teorema del Cateto para el cateto del rectángulo $ABC$, el cual coincide, precisamente, con el radio de la esfera:
			\begin{align*}
				20^2&=(30+x)\cdot x \ \rightarrow \ 400=30x+x^2\\
				%x^2&+30x-400=0 \ \rightarrow \ x=\dfrac{-30\pm50}{2}=\begin{cases}-40cm\\ 10cm\end{cases}
			\end{align*}	
			
%			Como estamos tratando con distancias, la solución $-40cm$ no es válida, por lo que $x=10$. Así:
%			\begin{align*}
%				%\text{Altura casquete}&=20-10=10cm\\
%				\text{Área casquete}&=2\pi Rh=400\pi cm^2
%			\end{align*}	 
		\end{minipage}
	
	\end{multicols}
	Como estamos tratando con distancias, la solución $-40cm$ no es válida, por lo que $x=10$. Así:
			\begin{align*}
				\text{Altura casquete}&=20-10=10cm\\
				\text{Área casquete}&=2\pi Rh=400\pi cm^2
			\end{align*}	
	%\vspace{-1.2cm}
\end{proof}

	El análisis de este ejercicio, según los conocimientos que emplea, queda reflejado en la Tabla \ref{ej40p138}.
	
	\begin{table}[h!]
	\centering
	\scalebox{1}[1]{
		\begin{tabular}{|c||l|c|l|c|}
\hline	\cellcolor[gray]{0.8}& \multicolumn{4}{|>{\cellcolor[cmyk]{0.9,0.3,0.2,0.1}}c|}{\textbf{Tarea}} \\ \cline{2-5}
%
	\cellcolor[gray]{0.8} & \cellcolor[gray]{0.95}	&\multicolumn{2}{|c|}{\cellcolor[gray]{0.95}Conocimiento}	& \cellcolor[gray]{0.95} 	\\ \cline{3-4}
%
	\multirow{-3}{*}{\cellcolor[gray]{0.8}ej} &	\multirow{-2}{*}{\cellcolor[gray]{0.95}Configuración} &	\cellcolor[gray]{0.95}Antiguo	&\cellcolor[gray]{0.95}Nuevo	&\multirow{-2}{*}{\cellcolor[gray]{0.95}NPF}\\
%
\hline		\cellcolor[gray]{0.8}40	&  \begin{minipage}[l]{0.2cm}\small Esquema/\\ dibujo\end{minipage}	& \begin{minipage}[l]{3.5cm}\small\noindent\textbullet Cálculo literal\\ \noindent\textbullet Álgebra (Ec. 2ºG.)\\ \noindent\textbullet Área casquete \end{minipage}&\begin{minipage}[l]{1.21cm}\small\noindent TC (P)\end{minipage}&\begin{minipage}[l]{3.6cm}\small\noindent\textbullet Reconocer teorema\\ \noindent\textbullet Cálculos intermedios\\ \noindent\textbullet Etapas\end{minipage}\\ \hline		
		\end{tabular}
		}
		\caption{Tabla Sistema \{tarea,desarrollo\} ejercicio 40, p. 138}\label{ej40p138}
		\end{table}
		
%	\begin{table}[h!]
%	\scalebox{0.92}[1]{
%		\begin{tabular}{|c||l|l|l|l||l|l|l|l|}
%\hline	\cellcolor[gray]{0.8}& \multicolumn{4}{|>{\cellcolor[cmyk]{0.9,0.3,0.2,0.1}}c||}{\textbf{Tarea}} & \multicolumn{4}{|>{\cellcolor[cmyk]{0.9,0.3,0.2,0.1}}c|}{\textbf{Desarrollo}}\\ \cline{2-9}
%%
%	\cellcolor[gray]{0.8} & \cellcolor[gray]{0.95}	&\multicolumn{2}{|c|}{\cellcolor[gray]{0.95}Conocimiento}	& \cellcolor[gray]{0.95} &	\cellcolor[gray]{0.95} & \multicolumn{3}{|c|}{\cellcolor[gray]{0.95}Ayudas}	\\ \cline{3-4}\cline{7-9}
%%
%	\multirow{-3}{*}{\cellcolor[gray]{0.8}ej} &	\multirow{-2}{*}{\cellcolor[gray]{0.95}Configuración} &	\cellcolor[gray]{0.95}Antiguo	&\cellcolor[gray]{0.95}Nuevo	&\multirow{-2}{*}{\cellcolor[gray]{0.95}NPF}&	\multirow{-2}{*}{\cellcolor[gray]{0.95}\begin{minipage}[l]{0.2cm}\noindent Tiempo\\ Silencio\end{minipage}}	& \cellcolor[gray]{0.95}Naturaleza & \cellcolor[gray]{0.95}Momento & \cellcolor[gray]{0.95}Forma  \\
%%
%\hline		\cellcolor[gray]{0.8}40	&  \begin{minipage}[l]{0.2cm}\scriptsize Esquema/\\ dibujo\end{minipage}	& \begin{minipage}[l]{1.28cm}\scriptsize\noindent\textbullet Cálculo literal\\ \noindent\textbullet Ec. 2ºG.\\ \noindent\textbullet Área casquete \end{minipage}&\begin{minipage}[l]{0.2cm}\scriptsize\noindent TC (P)\end{minipage}&\begin{minipage}[l]{1.31cm}\scriptsize\noindent\textbullet Reconocer\\ teorema\\ \noindent\textbullet Cálculo\\ intermedio\\ \noindent\textbullet Etapas\end{minipage}&	10 mints	&	Método	&\begin{minipage}[l]{0.2cm}\scriptsize\noindent Tras\\ Investigación\end{minipage} &\scriptsize Colectivo	\\ \hline		
%		\end{tabular}
%		}
%		\caption{Tabla Sistema \{tarea,desarrollo\} ejercicio 40, p. 138}\label{ej40p138}
%		\end{table}
		
	Este ejercicio presenta una resolución más laboriosa que es necesario interpretar. En este caso, ante la necesidad de reconocer la situación, se presenta un dibujo esquemático representándola. Para su resolución será necesario el cálculo literal que se deriva de la aplicación de la propiedad P, y las fórmulas geométricas correspondientes al cono que se habrán de conocer. Debido a ello se deberán manejar expresiones algebraicas dando lugar a la manipulación de la ecuación de segundo grado y sus soluciones. Para ello será necesario reconocer el Teorema del Cateto que respalda la propiedad P, así como establecer las etapas que nos permitan calcular el área final con sus correspondientes cálculos intermedios.
	
	%Este ejercicio presenta una resolución más laboriosa que se ha de interpretar, por lo que el tiempo de trabajo individual de los alumnos habrá de ser mayor que en los casos anteriores: diez minutos. Trascurrido este tiempo y, de nuevo, a modo de corrección y de forma colectiva, se mostrarán algunas pautas para el reconocimiento de la situación de un triángulo rectángulo que nos permita aplicar el teorema.
		

	\paragraph{\underline{Teorema de la Altura.}} Para ver ahora un ejemplo de este caso atenderemos al ejercicio 38 de los propuestos en la página 138.
	
	\begin{quote}\small
	\begin{multicols}{2}
		\begin{minipage}{7cm}
			En una esfera de $15cm$ de radio hemos inscrito un cono de altura $12cm$.
Calcula su área lateral.
		\end{minipage}
		
	\columnbreak
	
	\begin{center}
		\includegraphics[scale=0.25]{Ej38p138.png}
	\end{center}
	\end{multicols}
	\end{quote}
	
\begin{proof}
	Siguiendo la dinámica de estos ejercicios, el realizar un pequeño esquema/dibujo ayudará a la comprensión del ejercicio y una mejor visualización de la posible aplicación de algún teorema, en este caso, el Teorema de la Altura gracias al triángulo rectángulo $ABC$ que se obtiene:

\newpage	
	
	\begin{multicols}{2}
		
		\begin{center}
		\includegraphics[scale=0.3]{Ej38p138s1.png}
%		\vspace{0.4cm}
%		\begin{align*}
%		&\text{Generatriz cono: }\\ &\quad g^2=12^2+216=360 \ \rightarrow \ g\approx18,98cm\\
%		%&\text{Área lateral: }\\ &\quad\pi rg=2\pi\cdot14,7\cdot18,98\approx279\pi cm^2
%	\end{align*}
	\end{center}
	
	
				
	\columnbreak
	
	\begin{minipage}{7cm}
			Observemos que el radio de la esfera es $r_{\text{esf}}=15cm$, y así $x=\overline{DC}=30-12=18cm$
			
			Ahora, aplicando el Teorema de la Altura (propiedad P) para el triángulo $ABC$, tenemos que $$r^2=12x=12\cdot18=216 \ \rightarrow \ r\approx 14,7cm$$	
			
%			De este modo, podemos calcular la generatriz del cono fácilmente y, con ella, su área lateral:
%			\begin{align*}
%				&\text{Generatriz cono: }\\ &\quad g^2=12^2+216=360 \ \rightarrow \ g\approx18,98cm\\
%				&\text{Área lateral: }\\ &\quad\pi rg=2\pi\cdot14,7\cdot18,98\approx279\pi cm^2
%			\end{align*}	 
		\end{minipage}
	\end{multicols}
	
	De este modo, podemos calcular la generatriz del cono fácilmente y, con ella, su área lateral:
	\vspace{-0.2cm}
		\begin{align*}
			&\text{Generatriz cono: } \quad g^2=12^2+216=360 \ \rightarrow \ g\approx18,98cm\\
			&\text{Área lateral: }\quad\pi rg=\pi\cdot14,7\cdot18,98\approx279\pi cm^2
		\end{align*}
	\vspace{-0.8cm}
\end{proof}
	
	Analizando, de nuevo, sus conocimientos elaboramos la Tabla \ref{ej38p138}.
	
	\begin{table}[h!]
	\centering
	\scalebox{1}[1]{
		\begin{tabular}{|c||l|c|l|c|}
\hline	\cellcolor[gray]{0.8}& \multicolumn{4}{|>{\cellcolor[cmyk]{0.9,0.3,0.2,0.1}}c|}{\textbf{Tarea}} \\ \cline{2-5}
%
	\cellcolor[gray]{0.8} & \cellcolor[gray]{0.95}	&\multicolumn{2}{|c|}{\cellcolor[gray]{0.95}Conocimiento}	& \cellcolor[gray]{0.95} 	\\ \cline{3-4}
%
	\multirow{-3}{*}{\cellcolor[gray]{0.8}ej} &	\multirow{-2}{*}{\cellcolor[gray]{0.95}Configuración} &	\cellcolor[gray]{0.95}Antiguo	&\cellcolor[gray]{0.95}Nuevo	&\multirow{-2}{*}{\cellcolor[gray]{0.95}NPF}\\
%
\hline		\cellcolor[gray]{0.8}38	&  \begin{minipage}[l]{0.2cm}\small Esquema/\\ dibujo\end{minipage}	& \begin{minipage}[l]{3.5cm}\small\noindent\textbullet Cálculo literal\\ \noindent\textbullet Álgebra (Ec. 2ºG.)\\ \noindent\textbullet Área lateral cono \end{minipage}&\begin{minipage}[l]{1.2cm}\small\noindent TA (P)\end{minipage}&\begin{minipage}[l]{3.6cm}\small\noindent\textbullet Reconocer teorema\\ \noindent\textbullet Cálculos intermedios\\ \noindent\textbullet Etapas\end{minipage}\\ \hline		
		\end{tabular}
		}
		\caption{Tabla Sistema \{tarea,desarrollo\} ejercicio 38, p. 138}\label{ej38p138}
		\end{table}
%	\begin{table}[h!]
%	\scalebox{0.92}[1]{
%		\begin{tabular}{|c||l|l|l|l||l|l|l|l|}
%\hline	\cellcolor[gray]{0.8}& \multicolumn{4}{|>{\cellcolor[cmyk]{0.9,0.3,0.2,0.1}}c||}{\textbf{Tarea}} & \multicolumn{4}{|>{\cellcolor[cmyk]{0.9,0.3,0.2,0.1}}c|}{\textbf{Desarrollo}}\\ \cline{2-9}
%%
%	\cellcolor[gray]{0.8} & \cellcolor[gray]{0.95}	&\multicolumn{2}{|c|}{\cellcolor[gray]{0.95}Conocimiento}	& \cellcolor[gray]{0.95} &	\cellcolor[gray]{0.95} & \multicolumn{3}{|c|}{\cellcolor[gray]{0.95}Ayudas}	\\ \cline{3-4}\cline{7-9}
%%
%	\multirow{-3}{*}{\cellcolor[gray]{0.8}ej} &	\multirow{-2}{*}{\cellcolor[gray]{0.95}Configuración} &	\cellcolor[gray]{0.95}Antiguo	&\cellcolor[gray]{0.95}Nuevo	&\multirow{-2}{*}{\cellcolor[gray]{0.95}NPF}&	\multirow{-2}{*}{\cellcolor[gray]{0.95}\begin{minipage}[l]{0.2cm}\noindent Tiempo\\ Silencio\end{minipage}}	& \cellcolor[gray]{0.95}Naturaleza & \cellcolor[gray]{0.95}Momento & \cellcolor[gray]{0.95}Forma  \\
%%
%\hline		\cellcolor[gray]{0.8}38	&  \begin{minipage}[l]{0.2cm}\scriptsize Esquema/\\ dibujo\end{minipage}	& \begin{minipage}[l]{1.28cm}\scriptsize\noindent\textbullet Cálculo literal\\ \noindent\textbullet Ec. 2ºG.\\ \noindent\textbullet Área lat.\\cono \end{minipage}&\begin{minipage}[l]{0.2cm}\scriptsize\noindent TA (P)\end{minipage}&\begin{minipage}[l]{1.31cm}\scriptsize\noindent\textbullet Reconocer\\ teorema\\ \noindent\textbullet Cálculo\\ intermedio\\ \noindent\textbullet Etapas\end{minipage}&	10 mints	&	Método	&\begin{minipage}[l]{0.2cm}\scriptsize\noindent Tras\\ Investigación\end{minipage} &\scriptsize Colectivo	\\ \hline		
%		\end{tabular}
%		}
%		\caption{Tabla Sistema \{tarea,desarrollo\} ejercicio 38, p. 138}\label{ej38p138}
%		\end{table}
		
		Esta tarea es muy semejante a la anterior. Se presenta con un dibujo esquemático para su comprensión y se necesita del manejo y manipulación del álgebra para el trato de ecuaciones de segundo grado y sus soluciones, así como del conocimiento del área lateral del cono. Para resolverlo será necesario reconocer el Teorema de la Altura, obtenido de la aplicación de la propiedad P, y dividir el ejercicio en etapas para llegar a la solución por medio de pequeños cálculos intermedios.
		
		%Su desarrollo será también similar a la anterior. Ante la complejidad se dejarán diez minutos de silencio para el trabajo del alumno. Tras éste se darán pautas colectivas a modo de método para la localización del triángulo rectángulo necesario para la aplicación del teorema. 
	
\subsubsection{Libro Francés}	

	Centrémonos ahora en el libro francés escogido. Como hemos podido comprobar en las estructuras de los libros desarrolladas anteriormente, en este caso no se dan razonamientos sobre otras técnicas basadas en la semejanza, como el teorema del cateto o la altura en \citet{spa}, sino que todos los ejercicios versan sobre las propiedades fundamentales de la semejanza de triángulos. Aún así, podemos destacar algunos ejercicios más específicos que, al igual que en el punto anterior, utilizan figuras geométricas para su desarrollo.

%\newpage
	
	\paragraph{\underline{9. En un paralelogramo.}} Para el primer ejemplo recurriremos al ejercicio 9 de la página 263 de \citet{fr}:
	
%	\begin{multicols}{2}
%		
%	\begin{center}
		\begin{quote}
%		\begin{description}
			$ABCD$ es un paralelogramo. $N$ un punto del segmento $\overline{DC}$ distinto de $D$ y de $C$. La recta $\overline{AN}$ corta a $\overline{BC}$ en $M$.
			\begin{enumerate}
				\item Demostrar que los triángulos $ADN$ y $ABM$ son semejantes.
				\item Deducir que $\overline{DN}\times\overline{BM}=\overline{AB}\times\overline{AD}$
			\end{enumerate}
%		\end{description}
		\end{quote}
%	\end{center}

\begin{proof}
	
	En primer lugar, dado el enunciado, conviene considerar un pequeño esquema/dibujo de la situación, basándonos en la solución de la página 268 del propio libro:
	\begin{center}
\definecolor{zzttqq}{rgb}{0.6,0.2,0.}
\begin{tikzpicture}[line cap=round,line join=round,>=triangle 45,x=1.0cm,y=1.0cm]
\clip(1.5131583491981573,-1.471018967377032) rectangle (11.217440428860222,3.2292289616418777);
\fill[color=zzttqq,fill=zzttqq,fill opacity=0.10000000149011612] (2.,0.) -- (4.991384009202945,0.) -- (6.138546840034175,1.6382971157757922) -- (3.14716283083123,1.6382971157757922) -- cycle;
\draw (2.,0.)-- (3.14716283083123,1.6382971157757922);
\draw [color=zzttqq] (2.,0.)-- (4.991384009202945,0.);
\draw [color=zzttqq] (4.991384009202945,0.)-- (6.138546840034175,1.6382971157757922);
\draw [color=zzttqq] (6.138546840034175,1.6382971157757922)-- (3.14716283083123,1.6382971157757922);
\draw [color=zzttqq] (3.14716283083123,1.6382971157757922)-- (2.,0.);
\draw (2.,0.)-- (10.931376460263756,1.6382971157757922);
\draw (3.14716283083123,1.6382971157757922)-- (10.931376460263756,1.6382971157757922);
\draw (1.605981916916664,0.1576127207749421) node[anchor=north west] {$\mathbf{A}$};
\draw (2.7958112849447954,2.115346149019802) node[anchor=north west] {$\mathbf{B}$};
\draw (6.0193206366238465,2.132223161332258) node[anchor=north west] {$\mathbf{C}$};
\draw (4.9307533424704495,0.10698168383757502) node[anchor=north west] {$\mathbf{D}$};
\draw (5.344240144125616,0.6639230901486128) node[anchor=north west] {$\mathbf{N}$};
\draw (10.702691553330322,2.165977185957169) node[anchor=north west] {$\mathbf{M}$};
\end{tikzpicture}
	\end{center}
	\begin{enumerate}
		\item Los ángulos $\hat{BMA}$ y $\hat{MAD}$ son alternos-internos y, por construcción, $\overline{BM}$ // $\overline{AD}$ por lo que tienen la misma  medida: $\hat{BMA}=\hat{MAD}$.
		
		Además, por la propiedad de los paralelogramos, $\hat{ADN}=\hat{ABM}$.
		
		Así, los triángulos $ADN$ y $ABM$ tienen dos pares de ángulos con la misma medida y, por tanto, los restantes $\hat{AND}$ y $\hat{BAM}$ también.
		
		Aplicando la propiedad C1, los triángulos son semejantes.
		
		\item De lo anterior podemos ahora aplicar la propiedad P dando lugar a: $$\dfrac{\overline{DN}}{\overline{AB}}=\dfrac{\overline{AD}}{\overline{BM}}$$ luego $$\overline{DN}\times\overline{BM}=\overline{AB}\times\overline{AD}$$ utilizando la igualdad del producto en cruz.
	\end{enumerate}
\end{proof}
	%\vspace{-0.75cm}
	
	Procediendo como hasta ahora, una vez resuelto el ejercicio, podemos concluir su análisis a partir de los conocimientos que se utilizan. Para ello atendamos a la Tabla \ref{ej9p263}.
	
	\begin{table}[h!]
	\centering
	\scalebox{1}[1]{
		\begin{tabular}{|c||l|c|l|c|}
\hline	\cellcolor[gray]{0.8}& \multicolumn{4}{|>{\cellcolor[cmyk]{0.9,0.3,0.2,0.1}}c|}{\textbf{Tarea}} \\ \cline{2-5}
%
	\cellcolor[gray]{0.8} & \cellcolor[gray]{0.95}	&\multicolumn{2}{|c|}{\cellcolor[gray]{0.95}Conocimiento}	& \cellcolor[gray]{0.95} 	\\ \cline{3-4}
%
	\multirow{-3}{*}{\cellcolor[gray]{0.8}ej} &	\multirow{-2}{*}{\cellcolor[gray]{0.95}Configuración} &	\cellcolor[gray]{0.95}Antiguo	&\cellcolor[gray]{0.95}Nuevo	&\multirow{-2}{*}{\cellcolor[gray]{0.95}NPF}\\
%
\hline		\cellcolor[gray]{0.8}\begin{minipage}[l]{0.2cm}9 \\ 1\end{minipage}	& Enunciado & \begin{minipage}[l]{3.5cm}\small \noindent Propiedades\\ paralelogramo \end{minipage}&\begin{minipage}[l]{1.2cm}\small C1\end{minipage}&\begin{minipage}[l]{3.5cm}\small\noindent\textbullet Necesidad\\ elecciones\\  \noindent\textbullet Etapas\end{minipage}\\ 
%
\hline	\cellcolor[gray]{0.8}\begin{minipage}[l]{0.2cm}9 \\ 2\end{minipage}	& Enunciado & \begin{minipage}[l]{3.5cm}\small\noindent Cálculo literal\end{minipage}&\begin{minipage}[l]{1.2cm}\small\noindent P\end{minipage}&\begin{minipage}[l]{3.5cm}\small\noindent Cálculo intermedio\end{minipage}\\ \hline		
		\end{tabular}
		}
		\caption{Tabla Sistema \{tarea,desarrollo\} ejercicio 9, p. 263}\label{ej9p263}
		\end{table}
%	\begin{table}[h!]
%	\scalebox{0.92}[1]{
%		\begin{tabular}{|c||l|l|l|l||l|l|l|l|}
%\hline	\cellcolor[gray]{0.8}& \multicolumn{4}{|>{\cellcolor[cmyk]{0.9,0.3,0.2,0.1}}c||}{\textbf{Tarea}} & \multicolumn{4}{|>{\cellcolor[cmyk]{0.9,0.3,0.2,0.1}}c|}{\textbf{Desarrollo}}\\ \cline{2-9}
%%
%	\cellcolor[gray]{0.8} & \cellcolor[gray]{0.95}	&\multicolumn{2}{|c|}{\cellcolor[gray]{0.95}Conocimiento}	& \cellcolor[gray]{0.95} &	\cellcolor[gray]{0.95} & \multicolumn{3}{|c|}{\cellcolor[gray]{0.95}Ayudas}	\\ \cline{3-4}\cline{7-9}
%%
%	\multirow{-3}{*}{\cellcolor[gray]{0.8}ej} &	\multirow{-2}{*}{\cellcolor[gray]{0.95}Configuración} &	\cellcolor[gray]{0.95}Antiguo	&\cellcolor[gray]{0.95}Nuevo	&\multirow{-2}{*}{\cellcolor[gray]{0.95}NPF}&	\multirow{-2}{*}{\cellcolor[gray]{0.95}\begin{minipage}[l]{0.2cm}\noindent Tiempo\\ Silencio\end{minipage}}	& \cellcolor[gray]{0.95}Naturaleza & \cellcolor[gray]{0.95}Momento & \cellcolor[gray]{0.95}Forma  \\
%%
%\hline		\cellcolor[gray]{0.8}9	&  \begin{minipage}[l]{0.2cm}\scriptsize Enunciado\end{minipage}	& \begin{minipage}[l]{1.7cm}\scriptsize\noindent\textbullet Cálculo literal\\ \noindent\textbullet Propiedades\\ paralelogramo \end{minipage}&\begin{minipage}[l]{0.2cm}\scriptsize\noindent\textbullet P\\ \noindent\textbullet C1\end{minipage}&\begin{minipage}[l]{1.31cm}\scriptsize\noindent\textbullet Necesidad\\ elecciones\\  \noindent\textbullet Etapas\end{minipage}&	5-10 mints	&	Aclaración	&\begin{minipage}[l]{0.2cm}\scriptsize\noindent Durante la\\ investigación\end{minipage} &\scriptsize Individual	\\ \hline		
%		\end{tabular}
%		}
%		\caption{Tabla Sistema \{tarea,desarrollo\} ejercicio 9, p. 248}\label{ej9p248}
%		\end{table}
		
		En este caso, a pesar de que se ofrece una situación geométrica no trivial, el ejercicio se presenta únicamente mediante un enunciado, lo cual dificulta su realización. Para comenzar, es necesario conocer las propiedades que presentan los ángulos de un paralelogramo, las cuales permiten poner en práctica el conocimiento C1 para resolver el primer apartado en varias etapas. Para el segundo apartado será necesario hacer una correcta elección de las dimensiones correspondientes de los triángulos resultantes. Tras esto, bastará un cálculo literal de las consecuencias del primer apartado, aplicando la propiedad P, para obtener la demostración que requería el ejercicio.
		
		%En cuanto al desarrollo, basándonos también en lo propuesto por el libro, se considera que un periodo de diez minutos será suficiente para resolverlo. Sin embargo, puesto que los alumnos pueden tener la mayor dificultad a la hora de estructurar la situación del problema, tras un periodo de cinco minutos de silencio que les permita razonar sobre él, se llevarán a cabo pequeñas aclaraciones-guía de forma individual a aquellos alumnos que estén atascados en algún punto del ejercicio. Se espera, por tanto, que salvo casos puntuales la clase sea capaz de resolver el ejercicio. %No obstante, este ejercicio está pensado para ser uno de los últimos a llevar a cabo en el tema a fin de que los alumnos tengan cierta soltura sobre los conocimientos.
	
	\paragraph{\underline{10. Un poco de investigación.}} En este segundo ejemplo que podemos encontrar en la página 263 del mismo libro, el ejercicio 10 se presenta a través de una circunferencia asumiendo, a la vez, una nueva propiedad con la que el alumno debe trabajar.
	
	\begin{center}
		\begin{quote}
			Consideramos un círculo de diámetro $\overline{CD}$ en el cual se inscribe un triángulo $ABC$ tal que $\hat{ACD}=\hat{ABD}=22º$.\\
			Sea $H$ la base de la altura trazada desde $C$ en el triángulo $ABC$.\\
			Admitamos la propiedad siguiente:\textit{``Si un triángulo está inscrito en un círculo, y uno de sus lados es un diámetro de ese círculo, entonces dicho triángulo es rectángulo.''}
			\begin{enumerate}
				\item Dibuja la figura
				\item Demostrar que los triángulos $ADC$ y $BHC$ son semejantes
				\item Deducir que $\overline{CA}\times\overline{CB}=\overline{CH}\times\overline{CD}$
			\end{enumerate}
		\end{quote}
	\end{center}
	
\begin{proof} Basémonos en la resolución de la página 269 de dicho libro:
	\begin{enumerate}
		\item Dibujando la figura queda:
		\begin{center}
			\includegraphics[scale=0.7]{Ej10p243.png}
		\end{center}
		\item  Veamos ahora que los ángulos de los triángulos $ADC$ y $BHC$ son iguales a fin de aplicar la propiedad C1:
		\begin{itemize}
			\item El triángulo $ADC$ está inscrito en el círculo de diámetro $\overline{CD}$, aplicando la propiedad del enunciado tenemos que es rectángulo en $A$. Además, $BHC$ es rectángulo en $H$ por definición. Así $\hat{BHC}=\hat{DAC}$.
			\item El triángulo $DBC$ está inscrito en el círculo de diámetro $\overline{DC}$, por lo que es rectángulo en $B$; además, $\hat{DBA}=22º$ luego $\hat{ABC}=90º-22º=68º$, de forma que $ADC$ es rectángulo en $A$. Así $\hat{ADC}=90º-\hat{ACD}=90º-22º=68º$. Así, $\hat{ADC}=\hat{ABC}$
			
			En conclusión, los triángulos $ACD$ y $ABC$ tienen los ángulos iguales dos a dos y son, por tanto, semejantes.
		\end{itemize}
		\item Los triángulos $ACD$ y $CHB$ son semejantes, donde $\overline{CH}$ y $\overline{CA}$ por un lado y $\overline{CB}$ y $\overline{CD}$ por otro, son homólogos (puesto que el ángulo recto se encuentra en $H$ y $A$). Así, aplicando la propiedad P: $$\dfrac{\overline{CH}}{\overline{CA}}=\dfrac{\overline{CB}}{\overline{CD}} \ \Longrightarrow \ \overline{CA}\times\overline{CB}=\overline{CH}\times\overline{CD}$$
	\end{enumerate}
\end{proof}
	 
	 Analizando este último ejercicio a partir de los conocimientos empleados, comprobemos en la Tabla \ref{ej10p263} los resultados.
	 
	 \begin{table}[h!]
	\centering
	\scalebox{1}[1]{
		\begin{tabular}{|c||l|c|c|c|}
\hline	\cellcolor[gray]{0.8}& \multicolumn{4}{|>{\cellcolor[cmyk]{0.9,0.3,0.2,0.1}}c|}{\textbf{Tarea}} \\ \cline{2-5}
%
	\cellcolor[gray]{0.8} & \cellcolor[gray]{0.95}	&\multicolumn{2}{|c|}{\cellcolor[gray]{0.95}Conocimiento}	& \cellcolor[gray]{0.95} 	\\ \cline{3-4}
%
	\multirow{-3}{*}{\cellcolor[gray]{0.8}ej} &	\multirow{-2}{*}{\cellcolor[gray]{0.95}Configuración} &	\cellcolor[gray]{0.95}Antiguo	&\cellcolor[gray]{0.95}Nuevo	&\multirow{-2}{*}{\cellcolor[gray]{0.95}NPF}\\
%
\hline		\cellcolor[gray]{0.8}\begin{minipage}[l]{0.2cm}10 \\ 1\end{minipage}	& Enunciado & \begin{minipage}[l]{3.5cm}\small \noindent Elementos\\ circunferencia \end{minipage}&\backslashbox{\:}{\:}&\begin{minipage}[l]{3.5cm}\small\noindent Necesidad elección\end{minipage}\\ 
%
\hline	\cellcolor[gray]{0.8}\begin{minipage}[l]{0.2cm}10 \\ 2\end{minipage}	& Enunciado & \begin{minipage}[l]{3.5cm}\small\noindent\textbullet Cálculo literal\\ \noindent\textbullet Suma ángulos\\ triángulo \end{minipage}&\begin{minipage}[l]{1.8cm}\small\noindent\textbullet C1\\ \noindent\textbullet Propiedad\\ enunciado\end{minipage}&\begin{minipage}[l]{3.5cm}\small\noindent Cálculos intermedios\end{minipage}\\ 
%
\hline	\cellcolor[gray]{0.8}\begin{minipage}[l]{0.2cm}10 \\ 3\end{minipage}	& Enunciado & \begin{minipage}[l]{3.5cm}\small\noindent Cálculo literal\end{minipage}&\begin{minipage}[l]{1.8cm}\small\noindent P\end{minipage}&\begin{minipage}[l]{3.5cm}\small\noindent\textbullet Necesidad elección\\ \noindent\textbullet Etapas\end{minipage}\\ \hline	
		\end{tabular}
		}
		\caption{Tabla Sistema \{tarea,desarrollo\} ejercicio 10, p. 263}\label{ej10p263}
		\end{table}
%	 \begin{table}[h!]
%	\scalebox{0.92}[1]{
%		\begin{tabular}{|c||l|l|l|l||l|l|l|l|}
%\hline	\cellcolor[gray]{0.8}& \multicolumn{4}{|>{\cellcolor[cmyk]{0.9,0.3,0.2,0.1}}c||}{\textbf{Tarea}} & \multicolumn{4}{|>{\cellcolor[cmyk]{0.9,0.3,0.2,0.1}}c|}{\textbf{Desarrollo}}\\ \cline{2-9}
%%
%	\cellcolor[gray]{0.8} & \cellcolor[gray]{0.95}	&\multicolumn{2}{|c|}{\cellcolor[gray]{0.95}Conocimiento}	& \cellcolor[gray]{0.95} &	\cellcolor[gray]{0.95} & \multicolumn{3}{|c|}{\cellcolor[gray]{0.95}Ayudas}	\\ \cline{3-4}\cline{7-9}
%%
%	\multirow{-3}{*}{\cellcolor[gray]{0.8}ej} &	\multirow{-2}{*}{\cellcolor[gray]{0.95}Configuración} &	\cellcolor[gray]{0.95}Antiguo	&\cellcolor[gray]{0.95}Nuevo	&\multirow{-2}{*}{\cellcolor[gray]{0.95}NPF}&	\multirow{-2}{*}{\cellcolor[gray]{0.95}\begin{minipage}[l]{0.2cm}\noindent Tiempo\\ Silencio\end{minipage}}	& \cellcolor[gray]{0.95}Naturaleza & \cellcolor[gray]{0.95}Momento & \cellcolor[gray]{0.95}Forma  \\
%%
%\hline		\cellcolor[gray]{0.8}10	&  \begin{minipage}[l]{0.2cm}\scriptsize Enunciado\end{minipage}	& \begin{minipage}[l]{1.2cm}\scriptsize\noindent\textbullet Cálculo\\ literal\\ \noindent\textbullet Suma\\ ángulos\\ triángulo \end{minipage}&\begin{minipage}[l]{1.3cm}\scriptsize\noindent\textbullet P\\ \noindent\textbullet C1\\ \noindent\textbullet Propiedad\\ enunciado\end{minipage}&\begin{minipage}[l]{1.31cm}\scriptsize\noindent\textbullet Cálculos\\ intermedios\\  \noindent\textbullet Etapas\end{minipage}&	10 mints	&	Método	&\begin{minipage}[l]{0.2cm}\scriptsize\noindent Tras\\ Investigación\end{minipage} &\scriptsize Colectivo	\\ \hline		
%		\end{tabular}
%		}
%		\caption{Tabla Sistema \{tarea,desarrollo\} ejercicio 10, p. 248}\label{ej10p248}
%		\end{table}
	 
	 Esta última tarea presenta una complejidad notablemente más alta que las anteriores. Partiendo de que se presenta una situación geométrica únicamente a partir de un enunciado, se afirma una propiedad que se ha de asumir y, más aún, ser utilizada en el propio ejercicio (cosa que no es nada usual para el alumno). La tarea se divide en tres apartados: en el primero sólo entran en juego las capacidades gráficas del alumno, que deberá de tener claros los elementos de la circunferencia a los que hace referencia el enunciado, así como hacer uso de la elección del punto $B$ que satisfaga las propiedades que se indican, ya que dicho punto no es único; en el segundo apartado se deberán utilizar las medidas de los ángulos de un triángulo para establecer la semejanza a través de la propiedad C1 mediante cálculos intermedios y diversas etapas; por último, en el tercer apartado, bastará aplicar la propiedad P a las conclusiones de semejanza deducidas del apartado anterior para obtener la solución.
	 
	 %En lo referente al desarrollo de este ejercicio, dada su estructura, es interesante que los alumnos tengan tiempo de razonar sobre él por lo que se dejarán 10 minutos de silencio para el trabajo autónomo del alumno. Una vez transcurrido dicho periodo se resolverá el ejercicio marcando pautas de forma colectiva explicando los métodos a llevar a cabo, de forma que el conjunto de la clase exponga sus dudas y se pueda aclarar la resolución al conjunto del grupo.
	 
	 
	 
	 

\section{Conclusiones}

	En primer lugar hemos de destacar que, para la realización de este trabajo, el marco de investigación ha tenido unos límites definidos por los recursos de los que se ha contado. Todo el análisis que se ha realizado ha sido a través de dos libros, uno español y otro francés, intentando, además, que fueran de uso lo más actual posible para que las conclusiones llevadas a cabo tuvieran sentido. Dado que se han utilizado únicamente dos libros, las conclusiones obtenidas no pueden, a priori, generalizarse. Sin embargo, se ha de tener en cuenta que estos libros representan a los programas oficiales de sus países, y estos sí son generales y de obligado cumplimiento. %Con todo, se tiene muy claro que una generalización de estas conclusiones quedan algo limitadas a lo que aquí podemos observar, sin embargo, se ha de tener en cuenta que tanto dichas conclusiones como el análisis expuesto quedan respaldados por los programas escolares, a los que representan los libros escogidos.

	Definido el punto de partida y basándonos en el análisis del trabajo llevado a cabo, podemos observar que hay una gran diferencia a la hora de enfocar y enseñar la semejanza de triángulos en ambos países:
	
	Como hemos podido ver apoyándonos en los ejercicios y el enfoque teórico que aporta el libro de \citet{spa}, en España, la semejanza de triángulos versa, sobre todo, sobre el Teorema de Thales. A partir de éste se estructura toda la dinámica de la semejanza. A pesar de que las propiedades C1, C2, C3 y Área se trabajan en el tema, no se requieren en los ejercicios salvo en alguna ocasión concreta, sino que se utilizan para la demostración del Teorema de la Altura y del Cateto. Más allá de este hecho, el uso de dichas propiedades queda implícito al colocar los triángulos en posición de Thales para comprobar su semejanza y aplicar el teorema. Podemos concluir entonces que, en España, la semejanza va completamente ligada al Teorema de Thales, otorgando la importancia y repercusión a éste, y quedando la propia semejanza, en ocasiones, en un segundo plano. 
	
	En Francia, al contrario que en España, es el teorema de Thales el que queda desplazado a un segundo plano, como un caso particular de las consecuencias de las propiedades de la semejanza de triángulos. Tanto en la teoría como en los ejercicios es más que evidente el énfasis que se da a las distintas propiedades C1, C3, P y Área, siendo éstas el centro total de estudio del tema.
	
	Podríamos decir, en este contexto, que la semejanza sufre un autismo temático dado que no se relaciona con otras áreas de la matemática como podría ser el campo de las funciones a través de las homotecias o las isometrías; ni tampoco con otras asignaturas, como el dibujo técnico donde la noción tiene un papel fundamental. No se marca, de este modo, ninguna utilidad ni objetivo para el aprendizaje de la noción más que los breves comentarios que, en el caso del libro español, se ofrecen, intentando contextualizar la noción pero sin ir más allá del propio capítulo.
	
	Cierto es, como ya se ha mencionado, que el enfoque del libro de \citet{spa} es mucho más genérico que el libro de \citet{fr}, enfocando la semejanza de manera global, no sólo para triángulos. Esto lleva a que se destaque más la importancia en otros conceptos, sin embargo, y centrándonos en uno de los problemas principales que señala la autora como es la localización de los vértices homólogos, esto no ayuda a superar los obstáculos y dificultades que plantea el artículo de Horoks sobre la semejanza de triángulos. Es más, al trabajar desde la perspectiva del teorema de Thales, la problemática de reconocer dichos vértices, cuando estos no están en una situación casi trivial, se acentúa ya que, para encontrar la semejanza, colocarlos en una posición de Thales puede ser una labor muy complicada. Además, cabe añadir la dificultad que analiza Horoks en su tesis de forma complementaria al resto, y es la que presenta el álgebra, ya que, en ocasiones, es necesaria para las resoluciones, obligando a un cambio de marco teórico que puede suponer un obstáculo para algunos alumnos.
	
	En conclusión, hemos podido ver que las formas de trabajo en ambos países son realmente diferentes. Sin embargo,  ni en un país con Thales (España) ni en el otro, basándonos en el artículo de Horoks (Francia), se enfrentan a las dificultades más importantes que destaca la propia autora en su investigación como son: las elecciones del profesor en la gestión de la clase cuando se basan en los programas escolares, junto con las dificultades u obstáculos en los que se pueden convertir; los conceptos con autismo temático marcados por los manuales escolares; o la que más critica Horoks en su trabajo, la localización de los vértices homólogos en situaciones no triviales.
	
%	\begin{itemize}
%		\item La localización de los vértices homólogos en situaciones no triviales
%		\item Los conceptos con autismo didáctico marcados por los manuales escolares, junto con las elecciones del profesor en la gestión de la clase en relación a éstos, y las dificultades en las que devienen
%		\item 
%	\end{itemize}
	
	

	
	
	
	
	
	
\include{Reflexiones}



%Indice Terminologico
%\def\indexname{Índice terminológico}
%\addcontentsline{toc}{chapter}{Índice Terminológico}
%\printindex


%Bibliografia:
\titleformat{\chapter}[hang]
    %{\titlerule[2pt]\Large\scshape\raggedright}{}{0em}{}[{\titlerule[2pt]}]
    {\titlerule\bf\Huge}{}{0em}{}[{\titlerule}]
    
\fancypagestyle{plain}{
	\fancyhead[L]{Trabajo Final de Máster}
	\fancyhead[C]{}
	\fancyhead[R]{Bibliografía}
	\fancyfoot[L]{}
	\fancyfoot[C]{}
	\fancyfoot[R]{\thepage}
	\renewcommand{\headrulewidth}{0.5pt}
	\renewcommand{\footrulewidth}{0.5pt}
}
\nocite{*}
%\renewcommand{\refname}{Bibliografía}
\bibliographystyle{flexbib}%amsplain plain    %apalike
\bibliography{Bi}


%Anexos
\titleformat{\chapter}[hang]
    %{\titlerule[2pt]\Large\scshape\raggedright}{}{0em}{}[{\titlerule[2pt]}]
    {\bf\Huge}{}{0em}{}[{}]
\include{Anexos}


\end{document}	
