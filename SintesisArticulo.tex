\chapter[Investigación sobre la semejanza de Triángulos]{Investigación sobre la \\semejanza de Triángulos}
%\markboth{\MakeUppercase{Resumen}}{\MakeUppercase{Resumen}}
%Encabezado
\lhead[Trabajo Final de Máster]{Investigación sobre la semejanza de Triángulos}
\rhead[Investigación sobre la semejanza de Triángulos]{Trabajo Final de Máster}

\begin{dedication} 
 ``La enseñanza de las matemáticas es mucho más complicada de lo que esperabas,\\ a pesar de que ya esperases que fuera más complicada de lo que esperabas.''
	\rightline{{\rm --- Edward Griffith Begle (1914-1978)}}
\end{dedication}
\pagestyle{fancy}

	El artículo de ``Los triángulos semejantes en clase de \textit{seconde}: la enseñanza de los aprendizajes'' de \citet{Horoks} se trabajan los contenidos y desarrollos de la enseñanza sobre la semejanza de triángulos en el curso de \textit{seconde} (15-16 años)\footnote{En el Anexo A se ha elaborado una tabla descriptiva de la estructura de los cursos en Francia para facilitar la compresión y asociación a la estructura de nuestro país.}, basado en la Tesis Doctoral de la misma autora autora \citep{TH}. En ella intenta estudiar el porqué de la gran cantidad de alumnos de matemáticas que asisten a clases particulares fuera de la escuela, y cómo se vincula este hecho a la forma en la que los diferentes profesores imparten sus clases, analizando las influencias que se ejerce en los alumnos\footnote{Se interpreta que esto es debido a que se considera que los aprendizajes se evalúan a través del éxito o el fracaso en los exámenes, cuestión que, en ocasiones, lleva al alumno con resultados negativos a acudir a las clases particulares. Razón por la cuál analiza la influencia que tiene el profesor y su práctica docente en dicho aprendizaje.}. A continuación comentaremos cada uno de los aspectos de dicho artículo. %Así, el estudio se centra en qué ocurre en clase, al tiempo que se trata de aprender las carencias o vacíos que tengan ciertos alumnos en términos de las tareas propuestas.  interpretación
	
\section*{Inicio del trabajo de investigación}
\phantomsection\addcontentsline{toc}{section}{Inicio del trabajo de investigación}
\lfoot[\thepage]{Inicio del trabajo de investigación}
\rfoot[Inicio del trabajo de investigación]{\thepage}
%Cuadro teórico
	En el artículo se adopta la perspectiva según la cual los aprendizajes se hacen a través de \textbf{actividades} propuestas en clase por el profesor por medio de \textbf{tareas}. Estos términos tienen significados concretos: \textit{actividades}\footnote{Ligadas a lo que el profesor manda, sus intervenciones y la propia metodología de la que haga uso -(Robert, 2003) citado por \citet{Horoks}.} va referido a lo que ``el alumno hace o no''; la \textit{tarea}, a la ``descripción del trabajo matemático'' (que puede dar lugar a \textit{sub-tareas} debidas a las modificaciones de la original); y las \textit{prácticas} a  ``todo aquello que el profesor hace antes, durante y después de la clase''. 
	
	Para analizar las actividades y valorar su complejidad se contemplan cuatro tipos de adaptaciones posibles sobre los procedimientos de resolución de las tareas, denotadas por \textit{nivel de puesta en funcionamiento} (en adelante NPF), que abarcan las distintas tácticas o técnicas que pueden seguir los alumnos a la hora de enfrentarse a sus ejercicios:
	\begin{itemize} \label{NPF}
		\item Reconocimiento de modalidades de aplicación de un teorema;
		\item Necesidad de hacer cálculos intermedios;
		\item Necesidad de hacer elecciones;
		\item Necesidad de introducir etapas.
	\end{itemize}
	
	Otras variables consideradas fueron los conocimientos antiguos que puedan intervenir y la configuración geométrica en la que se sitúa el ejercicio.		
	
	Se establece que el aprendizaje parte de lo que realiza el profesor y su colaboración en la clase, junto con el correspondiente desarrollo que el alumno llevará a cabo en casa. Se tiene en cuenta, por tanto, el análisis de las tareas atendiendo al tiempo en el que trascurre cada una, las formas de trabajo adoptadas y las ayudas aportadas por el profesor con las modificaciones que éste lleve a cabo. A partir de estos resultados se pretende prever qué rendimiento tendrán los alumnos fuera de clase así como una idea aproximada de sus actividades potenciales.

%Elección de la noción	
	Para poder realizar una comparación en varias clases la autora buscó una noción a observar que tuviera un número razonable de sesiones y que minimizara el aporte de conocimientos anteriores (a fin de distinguir los aprendizajes ligados a las nuevas propiedades). La noción elegida fue la ``Semejanza de triángulos''. Esta noción desaparece del currículo en 1970 y reaparece en el 2000, %(poco antes de la tesina) 
	lo que le llevó a pensar que esto daría lugar a una nueva gestión de las clases en las que sería interesante estudiar la evolución de las enseñanzas en los nuevos programas. Además, no exime de una posible generalización de los resultados a otras nociones.
	
%Problemática
	La autora defiende la diferencia entre el trabajo realizado en clase y las actividades que realmente lleva a cabo el alumno en casa. Sin embargo, el desarrollo de las tareas en clase aporta una aproximación bastante fiable de las posibles actividades potenciales del alumno. Para observar dicho desarrollo se atiende a las variables fijadas por el marco teórico antes definido, a través de las cuales se medirá el impacto sobre el resultado de los alumnos con respecto a la noción trabajada. Gracias a este análisis se pretende determinar, para cada conocimiento evaluado, la totalidad del trabajo previo propuesto así como sus condiciones.
	
%Límites del cuadro
	Se destaca que los aprendizajes no se adquieren únicamente de aquello que ocurre en clase sino que necesitan tiempo. Se recalca que, de haberse podido tener en cuenta, habría sido muy interesante poder analizar este factor pero que, dadas las características y circunstancias de la experiencia, no ha sido posible. Al igual que éste, muchos otros factores externos no se han podido medir ni analizar, sin embargo, la elección de la ``Semejanza de triángulos'' permite acotar al máximo los conocimientos enseñados en las clases observadas.
	

\section*{Un breve análisis de la noción de triángulos\\ semejantes}
\phantomsection\addcontentsline{toc}{section}{Un breve análisis de la noción de triángulos semejantes}
\lfoot[\thepage]{Un breve análisis de la noción de triángulos semejantes}
\rfoot[Un breve análisis de la noción de triángulos semejantes]{\thepage}
%Análisis del programa de matemáticas 2000 para la clase de \textit{seconde}
	El artículo señala que en Francia, en el escalón previo a la secundaria -el \textit{collège} (11-14 años)- dentro de la noción de ``Triángulos semejantes'', se tratan la simetría axial y central, y las rotaciones y  traslaciones; sin abordar otras trasformaciones entre figuras semejantes que podrían ser útiles más adelante. En el \textit{lycée} (15-17 años), en cambio, dicha noción se estudia junto al teorema de Thales y solamente en \textit{seconde}. Este trabajo tan aislado lleva a que algunos profesores no lo aborden debido a su reducida utilidad.
	
	Atendiendo a lo que indica el programa de 2.º, se destaca que la definición dada no es la euclídea que, según \citet[p. 55]{Elementos}, cita como sigue\footnote{O, según la autora en su artículo, modernizando la definición: ``\textit{Dos triángulos son semejantes si sus ángulos son respectivamente iguales y sus lados proporcionales}''.}:
	
	\begin{quote}\small
		``1. Figuras rectilíneas semejantes son las que tienen los ángulos iguales uno a uno y proporcionales los lados que comprenden los ángulos iguales.\\
		2. (Dos) figuras están inversamente relacionadas cuando en cada una de las figuras hay razones antecedentes y consecuentes\footnote{Los términos antecedente y consecuente aluden al dividendo y al divisor, respectivamente, de una determinada razón de semejanza.}.''
	\end{quote}
	
	 Sino que se utiliza una condición de semejanza, ya que sólo se necesita de la igualdad de dos ángulos para obtener la de sus correspondientes triángulos. Así mismo, el artículo refleja las siguientes tres caracterizaciones de los Triángulos Semejantes que se trabajan en la docencia de esta noción:
	\begin{itemize}
		\item[C1:] Dos triángulos que tienen sus ángulos iguales, respectivamente, son semejantes.\label{C1}
		\item[C2:] Dos triángulos que tienen un ángulo idéntico comprendido entre lados respectivamente proporcionales son semejantes.
		\item[C3:] Dos triángulos que tienen sus lados respectivamente proporcionales son semejantes.
	\end{itemize}

	A estas caracterizaciones se añaden: el recíproco de esta tercera propiedad (P) y la razón entre áreas de triángulos semejantes (Área) que, junto con las anteriores, conforman los aprendizajes que se analizan en el artículo.
	
%Algunos comentarios sobre los documentos que acompañan los programas
	Los documentos que acompañan los programas asocian la semejanza de triángulos con los triángulos isométricos o iguales. Se cuida el estatus del enunciado, los métodos lógicos y una continuidad de lo adquirido en el \textit{collège} (cuyos conocimientos \textit{movilizables} han de convertirse en \textit{disponibles}). Puesto que no se estudian las transformaciones no isométricas, esto lleva a que la única introducción posible de este tema sea mediante la semejanza euclídea. Sin embargo, dicha ausencia dará pie a problemas y obstáculos tales como la localización de los vértices homólogos.
	
%El problema de la localización de vértices homólogos
	Esta dificultad no se encuentra reflejada en los programas escolares, lo cual no es siempre evidente, sobre todo cuando los vértices no vienen dados en orden en el enunciado o cuando un triángulo viene encajado en otro (Figura \ref{Figura1Horoks}). Analizando los libros y manuales se puede observar que no se mencionan teoremas o propiedades, aparentemente triviales, como que ``el lado más largo siempre se sitúa de forma opuesta al ángulo más grande''; sino que se opta por clasificar lados y ángulos en función de su tamaño y, después, deducir los homólogos. Son cuestiones simples pero, si no son tratadas explícitamente, los estudiantes no tienen por qué ser conscientes de ellas no permitiendo, además, el uso de técnicas que permitan una explicación provechosa por parte de los profesores. Esto lleva a la autora a preguntarse sobre la relación entre el déficit que puede aportar este hecho, junto con la novedad de la noción, al aprendizaje del alumno.
	
	\begin{figure}[h!]
		\centering
		\includegraphics[scale=0.7]{Figura1Horoks.png}
		\caption{Ejemplo  de triángulos semejantes encajados, extraído de (Horoks, 2008, p. 392).}
		\label{Figura1Horoks}
	\end{figure}
	
%Análisis de los ejercicios
	Por otro lado, el artículo utiliza once libros ofertados a los alumnos con el fin de observar las propiedades más trabajadas, caracterizando la dificultad de los ejercicios propuestos en relación a ellas. Se reflexiona sobre si la falta de ciertas tareas sobre algunos conceptos, frente a otras más trabajadas, conlleva unas dificultades mayores en casa, encontrando, en esto, un objeto de estudio interesante. Con ello se ha podido detectar el problema de las transformaciones que, estando presentes en la noción de triángulos semejantes, se encuentran ausentes en los programas escolares desembocando, con ello, en un problema dentro las clases observadas.
	

\section*{Algunos elementos de metodología y ejemplos de\\ \mbox{análisis}}
\phantomsection\addcontentsline{toc}{section}{Algunos elementos de metodología y ejemplos de análisis}
\lfoot[\thepage]{Algunos elementos de metodología y ejemplos de análisis}
\rfoot[Algunos elementos de metodología y ejemplos de análisis]{\thepage}

%Los datos recogidos y su tratamiento
	Para proceder a este estudio, la autora se basó en la observación de tres profesoras: las señoras B., P. y F. durante sus clases en \textit{seconde}. No en todas las aulas ha estado presente un observador, sino que algunas se realizaron únicamente a través de grabaciones de vídeo. Aunque no se tuvo en cuenta el perfil de las profesoras como docentes, sí se tuvo en cuenta el ``nivel'' del centro.  Gracias a los vídeos se clasificaron las variables antes definidas (NPF), las tareas y las actividades potenciales. Con ello fue posible comparar los ejercicios realizados en clase frente a los propuestos en los controles.
	
%Análisis del sistema \{tarea, desarrollo\} a partir de un ejemplo
	Con el fin de sintetizar los datos recopilados Horoks elabora tablas de \{tarea, desarrollo\} que permiten analizar el sistema. Esto es debido a que, para analizar el trabajo que se realiza en una clase, se considera que no basta con conocer las tareas que se proponen sino que el comprobar cómo se desarrollan éstas en el aula también es fundamental para el estudio que se pretende llevar a cabo. De esta manera, indicando el \textbf{ejercicio}, la \textbf{tarea} y el \textbf{desarrollo}, es posible desglosar: la \textit{configuración}, los \textit{conocimientos} (antiguos y nuevos) y el \textit{NPF} para la tarea, así como los \textit{tiempos de ``silencio''} y las \textit{ayudas} para el desarrollo de la cuestión. Esta organización del trabajo por parte del profesor será clave, ya que determinará las iniciativas propuestas a los alumnos y las posibles modificaciones de los ejercicios que puedan ser de interés. A través de este método se han analizado las clases de las tres profesoras, obteniendo así un balance del conjunto de las sesiones en las que se trataba la semejanza de triángulos.
	
%Análisis de las tareas del control y puesta en paralelo con lo que ocurre en clase
	Del mismo modo, también se realiza un análisis de las tareas y ejercicios que se disponen en los controles y se comparan con los realizados en clase%\footnote{\textit{Posibilidad de escribir como anexo los ejemplos a los que hace referencia el artículo, con la resolución pertinente de cada uno de ellos: Figura6,7; y quizá algunos cuadros}}
	, ya que en ellos no se recibe ninguna ayuda por parte del profesor. A través de tablas %\footnote{\textit{como la 7 y 8, añadir si es preciso}}
	 se pueden observar las similitudes entre los ejercicios, en particular aquellos cuyas variables son más próximas y que pueden prepararles mejor para los del control. A continuación se muestra un ejemplo de cómo se realiza dicho análisis para un ejercicio concreto (obtenido de \citet[p. 398-401]{Horoks} y \citet[p. 110]{TH}):%\footnote{Se ha constatado, antes del estudio, que las prácticas de los profesores eran estables y coherentes y que el desarrollo durante el capítulo era el mismo para cada profesor}.
	 
	 \begin{quote}\small
		``Ejercicio 9:\\
		$(C)$ es un círculo de centro $O$ y de radio $r$, $[AB]$ es un diámetro de $(C)$ y $P$ es el punto de $[AB]$ tal que $AP=1/3r$.\\
		Una recta $d$ distinta de la recta $(AB)$ pasa por $P$ y corta al círculo en dos puntos $M$ y $N$.\\
		1) Demostrar que los triángulos $APM$ y $NPB$ son triángulos semejantes.\\
		2) Deducir que $PM\times PN=5/9r^2$\\
		\begin{center}
			\includegraphics[scale=0.35]{Exercice9.PNG}
		\end{center}
		Solución:\\
		1) Se demuestra que los dos triángulos tienen dos ángulos iguales con la ayuda de los ángulos opuestos por el vértice y el teorema del ángulo inscrito. Así son, por tanto, semejantes.\\
		2) Dos triángulos semejantes tienen sus lados respectivamente proporcionales, por lo que aquí, tras la localización de los vértices homólogos, se obtiene: $MP/BP=PA/PN$. Obteniendo la igualdad buscada.
		\begin{center}
		\scalebox{0.84}[1]{
		\begin{tabular}{|c||l|l|l|l||l|l|l|l|}
\hline	\cellcolor[gray]{0.8}& \multicolumn{4}{|>{\cellcolor[cmyk]{0.9,0.3,0.2,0.1}}c||}{\textbf{Tarea}} & \multicolumn{4}{|>{\cellcolor[cmyk]{0.9,0.3,0.2,0.1}}c|}{\textbf{Desarrollo}}\\ \cline{2-9}
%
	\cellcolor[gray]{0.8} & \cellcolor[gray]{0.95}	&\multicolumn{2}{|c|}{\cellcolor[gray]{0.95}Conocimiento}	& \cellcolor[gray]{0.95} &	\cellcolor[gray]{0.95} & \multicolumn{3}{|c|}{\cellcolor[gray]{0.95}Ayudas}	\\ \cline{3-4}\cline{7-9}
%
	\multirow{-3}{*}{\cellcolor[gray]{0.8}ej} &	\multirow{-2}{*}{\cellcolor[gray]{0.95}Configuración} &	\cellcolor[gray]{0.95}Antiguo	&\cellcolor[gray]{0.95}Nuevo	&\multirow{-2}{*}{\cellcolor[gray]{0.95}NPF}&	\multirow{-2}{*}{\cellcolor[gray]{0.95}\begin{minipage}[l]{0.2cm}\noindent Tiempo\\ Silencio\end{minipage}}	& \cellcolor[gray]{0.95}Naturaleza & \cellcolor[gray]{0.95}Momento & \cellcolor[gray]{0.95}Forma  \\
%
\hline		\cellcolor[gray]{0.8}\begin{minipage}[l]{0.2cm}9\\ 1)\end{minipage}&  círculo	& \begin{minipage}[l]{1.28cm}\scriptsize\noindent Cálculo literal\end{minipage}& C1 &\begin{minipage}[l]{1.31cm}\scriptsize\noindent Cálculos\\ Intermedios \end{minipage}&	\begin{minipage}[l]{1.31cm}\scriptsize\noindent 29 mint\\ al ppio. \end{minipage}	&	Método	&\begin{minipage}[l]{0.2cm}\scriptsize\noindent Tras\\ Investigación\end{minipage} &\scriptsize Individual	\\
%		
\hline		\cellcolor[gray]{0.8}\begin{minipage}[l]{0.2cm}9\\ 2)\end{minipage}&  círculo	& \begin{minipage}[l]{1.28cm}\scriptsize\noindent Ángulo inscrito\end{minipage}& P &\begin{minipage}[l]{1.31cm}\scriptsize\noindent Cálculos\\ Intermedios \end{minipage}&	\begin{minipage}[l]{1.31cm}\scriptsize\noindent 15 mint\\ al ppio. \end{minipage}	&	Método	&\begin{minipage}[l]{0.2cm}\scriptsize\noindent Tras\\ Investigación\end{minipage} &\scriptsize Individual	\\ \hline
		\end{tabular}''
		}
		\end{center}
	 \end{quote}

%Interpretación de los resultados de los alumnos
	A partir de las respuestas obtenidas en el control, sus resultados durante el año y las propias apreciaciones de los profesores se clasificó a los alumnos en dos categorías: ``buenos'' y ``malos'' con el fin de comparar las influencias previas que puede tener, en los resultados del control\footnote{Ante esta categorización, el artículo es prudente puesto que no se sabe si son los buenos alumnos los que se aprovechan de ciertas elecciones del profesor, o si son los alumnos que se aprovechan de las opciones del profesor los que se convierten en buenos alumnos.}, todo aquello que se realiza en clase y fuera de ella. Se obtuvo, lógicamente, que si se utiliza un \textit{NPF} más elevado en el control que el trabajado en clase se vuelve un obstáculo para el alumno. %Parece ser que un \textit{NPF} más elevado en el examen que en el trabajo de clase se vuelve un obstáculo para gran parte del alumnado (sólo diez alumnos de los ``buenos'' han tenido éxito en una de estas cuestiones requeridas), mientras que un \textit{NPF} más simple en el control desemboca en unos resultados notablemente superiores\footnote{\textit{tablas 9,10,11 y 12 interesantes como resumen visual}}. 
 Otra constatación importante es la dificultad de la conexión entre la relación de dos triángulos semejantes cuando uno se encuentra contenido en otro  en una posición diferente a la de Thales, junto con la proporcionalidad de sus lados \footnote{Ejemplo Figura \ref{ExerciceExamen} obtenido de \citet[p. 402]{Horoks} y \citet[p. 131]{TH} en la que se ha de comprobar la semejanza entre los triángulos MNE y END.}, ya que supone localizar los vértices homólogos en ambos triángulos. Esto ofrece la ocasión de comprobar si los alumnos han asimilado esta tarea fundamental, dado que algunas soluciones pueden haberse obtenido tras la aplicación de técnicas que el profesor utilizaba en clase, tomándolas como ``recetas'', no porque se haya comprendido la noción.
 
 \begin{figure}
 	\centering
 	\includegraphics[scale=0.35]{ExerciceExamen.PNG}
 	\caption{Figura asociada a uno de los ejercicios de los controles analizados en el artículo.}
 	\label{ExerciceExamen}
 \end{figure}

	De este modo, la distribución del \textit{NPF} es una variable relevante del análisis de las tareas para evaluar los aprendizajes de los alumnos: periodos amplios de tiempo de investigación personal, sin intervenciones del profesor ni puestas en común, beneficia únicamente a los ``buenos'' alumnos. 
	
\section*{Algunos resultados}
\phantomsection\addcontentsline{toc}{section}{Algunos resultados}
\lfoot[\thepage]{Algunos resultados}
\rfoot[Algunos resultados]{\thepage}

%Los aprendizajes: las estrechas relaciones con lo que pasa en clase
	Obtenidos los resultados tras los diferentes controles, en lugar de centrar la atención en la clase que mejor resolvió las pruebas, el artículo reconoce una mayor importancia en la resolución de los propios ejercicios: cuáles han sido superados por la mayor parte de los alumnos o, por el contrario, cuál ha sido el que ha supuesto un mayor fracaso. Esto permitirá deducir qué tipo de propuestas, aportadas por los profesores, beneficiará más a los alumnos. 
	
	El ``ejercicio mejor resuelto'' está asociado a una preparación previa y concienzuda en el aula, con ejercicios de mayor o igual dificultad tanto en clase como en el control, y para los cuales se ha tenido tiempo de indagación y trabajo personal. En todos los casos se trata de ejercicios que precisan la caraterización C1, y parece indicarse que los conocimientos de los alumnos no son transferibles a \textit{NPF} más elevados\footnote{Crahay(2000) citado por \citet{Horoks} - ``Los alumnos, finalmente, aprenden... lo que se les enseña. Pero no es tan sencillo.''.}.
	
	En lo referente al ``ejercicio peor resuelto'' hay mayor diferencia entre las clases observadas. Se trata de preguntas que requerían una localización de los vértices homólogos no trivial. Esta dificultad no está ligada únicamente a la organización del trabajo en clase, sino que es un elemento importante de las decisiones del profesor en el aula ya que, en los casos observados, es éste el que normalmente se encarga de dirigirla e incidir en las dificultades. De hecho, en el caso de las dos profesoras que en clase sólo habían hecho ejercicios con las letras de los vértices homólogos de forma ordenada, se detectó dificultades en la resolución del ejercicio en el que, por primera vez, se encontraban desordenadas. 
	
	Por último, lo que caracteriza a los ejercicios del control donde se obtuvo los ``resultados menos homogéneos'' concierne a aquellos en los que se utilizaba una situación más adidáctica, dejando la mayor autonomía al alumno sin intervención del profesor durante o después de la tarea. Esto parece indicar, como se señaló antes, que el trabajo autónomo no beneficia de la misma forma a todos los alumnos.
	
	De este modo el artículo concluye que las decisiones del profesor pueden tener ciertas influencias sobre los aprendizajes de los alumnos, aunque hay que señalar que, en muchas ocasiones, estas decisiones vienen impuestas por los programas escolares, factor que es necesario tener en cuenta. De hecho, Horoks se sorprende ante la falta de comentarios o instrucciones sobre esta dificultad en dichos programas.
	
%Algunas observaciones sobre las prácticas de las enseñanzas en este capítulo
	Con respecto a las metodologías adoptadas por cada una de las profesoras, la autora concluye que la Sra. B realiza una simplificación de los ejercicios generalmente demasiado pronto, de forma que los alumnos acaban por realizar tareas muy dirigidas y aisladas. La Sra. P interviene en los ejercicios después de un tiempo de estudio autónomo por parte de los alumnos, dejándoles, potencialmente, la carga de las tareas complejas y elevando así, notablemente, el \textit{NPF} respecto de las otras clases. En último lugar, la Sra F. simplifica rápidamente las tareas complejas que se presentan en clase, sin que los alumnos puedan beneficiarse de un tiempo de estudio de cara a las preguntas más difíciles que trabajarán más tarde en casa.
	
	\begin{table}[h!]
		\centering
		\scalebox{0.9}[1]{
		\begin{tabular}{|>{\columncolor[gray]{0.7}}c|c|c|c|}
	\hline \cellcolor[gray]{0.6}\backslashbox{metodología}{profesora}& 	\cellcolor[gray]{0.8}Sra. B	 &	\cellcolor[gray]{0.8}Sra. P	 &	 \cellcolor[gray]{0.8}Sra. F\\
	\hline nivel e preparación  &	bajo	 & muy bueno & muy bueno\\
	\hline tareas de clase		& simples	 & complejas & complejas\\
	\hline carga de los alumnos
			en clase			& tareas simples & tareas complejas & tareas simples\\
	\hline en casa				& simples	 & simples o complejas  & complejas\\
	\hline en el control 		& complejas  & complejas 			& simples\\%\footnote{Razón de buenos resultados}\\
	\hline resultados de los
			alumnos				& medios y heterogéneos	 & buenos	& buenos\\
	\hline
		\end{tabular}
		}
		\caption{Metodología y tareas de las diferentes profesoras}
		\label{metodologias}
	\end{table}
	
	Como resumen ante estas metodologías podemos observar la Tabla \ref{metodologias}, que nos permite comprender el porqué de los buenos resultados obtenidos en la clase de la Sra. F, a pesar de que su metodología de trabajo no sea eficaz, como es el caso de la Sra. P. que, aplicando una dinámica de trabajo que introduce más tareas complejas, obtiene también resultados positivos. Cabe destacar que la diferencia entre lo que se propone y lo que es potencialmente realizado por los alumnos en la clase tiene, ciertamente, una influencia sobre la aptitud para resolver tareas complejas. Sin embargo, el artículo señala que este hecho no puede ser evaluado por el control propuesto, de manera que conforma un límite en su estudio. Es interesante constatar, aún así, que las prácticas son estables y coherentes a lo largo del capítulo, que la novedad de esta noción en los programas es un elemento a tener en cuenta y que otras investigaciones, como la llevada a cabo por Roditi (2005) (citado por \citet{Horoks}), obtuvieron resultados similares. 
	
\section*{Conclusión}
\phantomsection\addcontentsline{toc}{section}{Conclusión}
\lfoot[\thepage]{Conclusión}
\rfoot[Conclusión]{\thepage}

	Para concluir el artículo se vuelve a la preocupación de partida: determinar los déficits que pudieran tener ciertos alumnos en clase en lo referente a la noción de triángulos semejantes.
	
%Límites y/o faltas en la enseñanza de la noción
	La primera y clara carencia, que ha quedado manifiesta a lo largo del artículo, es la referente a la localización de los vértices homólogos. Ésta queda ligada al vacío en los programas educativos que no facilitan al profesor el poder ofrecer un método sistemático o una justificación matemática de su situación. Como cita \citet[p. 341]{TH}: ``A pesar de la libertad de elección de sus estrategias de enseñanza - libertad que ejercen en este capítulo\footnote{Haciendo referencia al capítulo de triángulos semejantes.}, con estrategias muy diferentes - la restricción del programa y los horarios son muy fuertes''. %A coalición con ello, en la tesis se destacan la gran cantidad de errores en cuanto a las propiedades concernientes a las longitudes de los lados o la confusión entre triángulos isométricos y semejantes. Además destaca el uso del álgebra
	
	Otra de las limitaciones destacables es la poca variedad de tareas propuestas, trabajando realmente poco ciertas aplicaciones. En cuanto a las ``actividades posibles`'' de los alumnos cabe remarcar que no siempre reflejan las tareas propuestas y que, el trabajo en pequeños grupos, puede frenar el aprendizaje de aquellos que no son demasiado buenos. 
	
%Vuelta a las clases particulares: un medio para ir más lejos en esta investigación
	Finalmente, el artículo advierte de la importancia que suscita el fenómeno de las clases particulares. Las cifras de estudiantes que acuden a este tipo de cursos en Francia aumenta vertiginosamente. El ser capaces de poder estudiar y analizar este tipo de clases hubiera sido un valioso recurso para completar la investigación y las reflexiones del artículo pero, desgraciadamente, señala la gran dificultad de poder llevarlo a cabo. Es un tema delicado para ciertos profesores y también alumnos, que no siempre admiten tener este tipo de ayudas.
	













%	Esquema de resumen
%	\begin{itemize}
%		\item Introducción
%		\item Puesta en marcha del trabajo de búsqueda
%		\begin{enumerate}
%			\item Cuadro teórico
%			\item Elección de la noción
%			\item Problemática
%			\item Límites del cuadro
%		\end{enumerate}
%		\item Un breve análisis de la noción de triángulos semejantes
%		\begin{enumerate}
%			\item Análisis del programa de matemáticas 2000 para la clase de \textit{seconde}
%			\item Algunos comentarios sobre los documentos que acompañan los programas
%			\item El problema de la localización de los vértices homólogos
%			\item Análisis de los ejercicios de los manuales escolares
%		\end{enumerate}
%		\item Algunos elementos de metodología y ejemplos de análisis
%		\begin{enumerate}
%			\item Los datos recogidos y su tratamiento
%			\item Análisis del sistema \{tarea, desarrollo\} a partir de un ejemplo
%			\item Análisis de las tareas del control y puesta en paralelo con lo que ocurre en clase
%			\item Interpretación de los resultados de los alumnos
%		\end{enumerate}			
%		\item Algunos resultados
%		\begin{enumerate}
%			\item Los aprendizajes: las estrechas relaciones con lo que pasa en clase
%			\item Algunas observaciones sobre las prácticas de las enseñanzas en este capítulo
%		\end{enumerate}
%		\item Conclusiones
%		\begin{enumerate}
%			\item Límites y/o faltas en la enseñanza de la noción
%			\item Vuelta a las clases particulares: un medio para ir más lejos en esta investigación
%		\end{enumerate}
%	\end{itemize}
	
	
	
	
%	
%	
%\chapter*{Cuestiones Relevantes}
%
%	A continuación se destacan algunas de las cuestiones más relevantes y que me han llamado la atención a lo largo del capítulo:
%
%\begin{itemize}
%	\item Vértices homólogos, ejemplos y problemáticas de ejemplos propuestos
%	\item Ejercicios y análisis de los programas educativos (libros de texto español/francés)
%	\item Niveles de puesta en funcionamiento (página \pageref{NPF})
%	\item Ventajas de la noción de triángulos semejantes
%	\item Límites del artículo (enmarcación, duración, imposibilidad de poner ejercicios en controles, imposibilidad de comprobar las clases particulares)
%	\item Sistema \{tarea, desarrollo\} empleado para recoger los datos
%	\item Relación: aquello que ocurre en clase $\longleftrightarrow$ aprendizajes del alumno y superación del control
%	\item Sistemas metodológicos de las profesoras (\ref{metodologias})
%\end{itemize}




