\chapter{Análisis Francia-España}
%\phantomsection\addcontentsline{toc}{chapter}{Análisis Francia-España}
%Encabezado
\lhead[Trabajo Final de Máster]{Análisis Francia-España}
\rhead[Análisis Francia-España]{Trabajo Final de Máster}
\begin{dedication} 
\large ``Toma lo que hace falta, opera como debes\\ y obtendrás lo que deseas.''
	\rightline{{\rm --- Gottfried Leibniz (1646-1716)}}
\end{dedication}

	El artículo de Horoks realiza un buen análisis sobre el tema de la semejanza de triángulos y cuáles son sus características respecto a la enseñanza en Francia. El tema de la semejanza de triángulos es un concepto que se enseña tanto en Francia como en España y que, en ambos países, es una noción algo desubicada en los programas educativos. Parece interesante comprobar, pues, cuáles son las diferencias fundamentales del aprendizaje de esta noción en relación a ellos. De esta manera, se seleccionó un libro correspondiente al \textit{lycée} francés y otro al instituto español para poder compararlos.
	
\section{Libros y estructura}
\lfoot[\thepage]{Libros y estructura}
\rfoot[Libros y estructura]{\thepage}

	En este apartado realizaremos un análisis del tema de semejanza de cada uno de los libros que se han elegido para llevar a cabo la comparación:
	
	Por un lado tenemos el libro español: \citealp*{spa}\footnote{Desarrollo del tema en Anexo B}. Como se especifica en el título, este libro corresponde al cuarto curso de la ESO donde, según la actual Ley Orgánica 8/2013, de 9 de diciembre, para la mejora de la calidad educativa (LOMCE), se tiene estipulado que se estudien las nociones de semejanza pertinentes para nuestro estudio. Por otro lado, disponemos del libro francés: \citet{fr}\footnote{Desarrollo del tema en Anexo C}. Este libro, en cambio, pertenece al curso de $3^{\text{ème}}$ que se asemejaría al curso de $3.º$ de la ESO en España\footnote{Estructura de los cursos esquematizada en Anexo A.}. Aunque los libros no se ajustan al mismo nivel de enseñanza, hemos visto conveniente su comparación y estudio puesto que los niveles son los más cercanos posibles y el estudio de las nociones se corresponde. Hay que tener en cuenta que los desarrollos teóricos son diferentes según el país y que, a pesar de que las edades no sean las mismas, los contenidos que se abordan sí son lo más similares posible.
	
	Al mismo tiempo que se describen los temas, se señalarán las correspondencias con los cinco conocimientos de la semejanza de triángulos sobre los que versa el artículo de Horoks, así como alguna notación nueva que nos servirá para los análisis posteriores.
	
\subsection{Libro Español}\label{DefConN}

	El libro de \citet{spa} estructura el tema con puntos bien diferenciados: da una breve introducción histórica de la semejanza, explica el concepto de semejanza, trata la semejanza de triángulos, particulariza en los triángulos rectángulos, explica algunas aplicaciones interesantes y termina la teoría con la semejanza de rectángulos y sus peculiaridades, aludiendo al número áureo. Finalmente, propone una serie de ejercicios: primero tres problemas resueltos y, tras ellos, una lista de cincuenta y cuatro ejercicios para practicar divididos por sus nociones. A modo de cierre del capítulo, se proponen ejercicios para razonar a partir de lo estudiado y se termina con una autoevaluación de siete ejercicios. 
	
	El esquema del tema es el siguiente:
	\begin{itemize}
		\item[] \underline{6. Semejanza. Aplicaciones}
		\begin{itemize}
			\item[\ding{43}] \textbf{Breve contexto histórico}
			\begin{itemize}
				\item[\textbullet] Thales de Mileto y teorema de Thales
				\item[\textbullet]Eratóstenes y su predicción del radio de la Tierra
				\item[\textbullet] Semejanza en la antigua China
			\end{itemize}
			\item[1.] \textbf{Semejanza}
			\begin{itemize}
				\item[\textbullet] Definición de Semejanza%: ``misma forma'' - ángulos correspondientes iguales y longitudes proporcionales. Razón de semejanza.
%				\item[\ding{43}] Semejanza en la vida cotidiana
				\item[\textbullet] Definición de Escala%: cociente entre longitud de reproducción y correspondiente longitud de la realidad. Escala $=$ razón de semejanza.
%				\begin{itemize}
%					\item[-] Objetos grandes: 1:200 $\Rightarrow$ 1$cm$ reproducción $\rightarrow$ 200$cm$ reales.
%					\item[-] Objetos pequeños: 100:1 $\Rightarrow$ 100$cm$ reproducción $\rightarrow$ 1$cm$ real.
%				\end{itemize}
				\item[\textbullet] Relación entre áreas y volúmenes semejantes (conocimiento: Área)%: áreas $\rightarrow \ k^2$, volúmenes $\rightarrow \ k^3$ 
%				\item[\ding{43}] Ejercicios resueltos y propuestos
			\end{itemize}
			\item[2.] \textbf{Semejanza de Triángulos}
			\begin{itemize}
%				\item[\ding{43}] Estudio de triángulos semejantes $\Longrightarrow$ Fundamental el Teorema de \mbox{Thales}
				\item[\textbullet] Teorema de Thales%: Si $a$, $b$ y $c$ son rectas paralelas cortando a $r$ y $s$, entonces los segmentos que determinan son semejantes.
				\item[\textbullet] Triángulos semejantes
				\item[\textbullet] Triángulos en posición de Thales \textit{(conocimiento que denotaremos \mbox{por \textup{T})}}
				\item[\textbullet] Criterios de Semejanza de Triángulos
				\begin{itemize}
					\item Primer criterio: $\overset{\triangle}{ABC}\sim\overset{\triangle}{A'B'C'}$ \ si \ $\hat{A}=\hat{A'}$ y $\hat{B}=\hat{B'}$ \\(conocimiento: C1)
					\item Segundo criterio: $\overset{\triangle}{ABC}\sim\overset{\triangle}{A'B'C'}$ \ si \ $\dfrac{a'}{a}=\dfrac{b'}{b}=\dfrac{c'}{c}$ \\(conocimiento: C3)
					\item Tercer criterio: $\overset{\triangle}{ABC}\sim\overset{\triangle}{A'B'C'}$ \ si \ $\hat{A}=\hat{A'}$ y $\dfrac{b'}{b}=\dfrac{c'}{c}$ \\(conocimiento: C2)
				\end{itemize}
%				\item[\ding{43}] Ejercicios propuestos
			\end{itemize}
			\item[3.] \textbf{Semejanza de Triángulos Rectángulos}
			\begin{itemize}
				\item[\textbullet] Criterio de Semejanza de Triángulos Rectángulos y consecuencias
				\item[\textbullet] Teorema del Cateto \textit{(conocimiento que denotaremos \mbox{por \textup{TC})}}
				\item[\textbullet] Teorema de la Altura \textit{(conocimiento que denotaremos \mbox{por \textup{TA})}}
%				\item[\ding{43}] Ejercicios resueltos y propuestos
			\end{itemize}
			\item[4.] \textbf{Aplicación de la Semejanza}
			\begin{itemize}
				\item[\textsquare] Ejercicio resuelto: Tronco de Cono
				\item[\textsquare] Ejercicio resuelto: Superficie visible de la Tierra a una altura dada
%				\item[\ding{43}] Ejercicios propuestos
			\end{itemize}
			\item[5.] \textbf{Semejanza de Rectángulos. Aplicaciones}
			\begin{itemize}
				\item[\textbullet] Caracterización de la semejanza
				\item[\textbullet] Una hoja de papel A4 y el Rectángulo Áureo
%				\item[\ding{43}] Ejercicios propuestos
			\end{itemize}
			\item[\ding{43}] Ejercicios y problemas resueltos
			\item[\ding{43}] Ejercicios y problemas
			\item[\ding{43}] Aprende y reflexiona y autoevaluación
		\end{itemize}
	\end{itemize}
	
	Destacamos la definición de algunos conocimientos nuevos: T, TC y TA que añadimos a los descritos por Horoks en su artículo (C1, C2, C3, P y Área) ya que serán de utilidad más adelante en este trabajo.
	
\subsection{Libro Francés}
	
	El libro de \citet{fr} también organiza los puntos del capítulo de forma bien diferenciada, pero con una estructura distinta al modelo español: introduce el tema con un ejercicio tipo, define la semejanza y el vocabulario que se empleará, establece propiedades sobre los ángulos y las longitudes, relaciona las áreas de triángulos semejantes y, finalmente, establece la solución del ejercicio tipo de la introducción y propone diez ejercicios con sus soluciones.
	
	Esquemáticamente, el capítulo se estructura de la siguiente forma:
	\begin{itemize}
		\item[10.] \underline{Triángulos Semejantes}
		\begin{itemize}
			\item[\ding{43}] Introducción con ejercicio tipo
			\item[1.] Definición y vocabulario
			\begin{itemize}
				\item[\textbullet] Definición de Semejanza
				\item[\textbullet] Definición de lados, ángulos y vértices homólogos
			\end{itemize}
			\item[2.] Propiedades sobre los ángulos
			\begin{itemize}
				\item[\textbullet] Propiedad 1: $\overset{\triangle}{ABC}\sim\overset{\triangle}{A'B'C'}$ \ si \ $\hat{A}=\hat{A'}$, $\hat{B}=\hat{B'}$ y $\hat{C}=\hat{C'}$ \\ (conocimiento: C1)
				\item[\textbullet] Propiedad 2: recíproca de la Propiedad 1 
			\end{itemize}
			\item[3.] Propiedades sobre las longitudes
			\begin{itemize}
				\item[\textbullet] Propiedad 3: Si $\overset{\triangle}{ABC}\sim\overset{\triangle}{A'B'C'} \ \Longrightarrow \ \dfrac{a'}{a}=\dfrac{b'}{b}=\dfrac{c'}{c}$ \\(conocimiento: P)
				\item[\textbullet] Propiedad 4: recíproca de la propiedad 3 (conocimiento: C3)
			\end{itemize}
			\item[4.] Relación entre las áreas
			\begin{itemize}
				\item[\textbullet] Propiedad 5: Si $\overset{\triangle}{ABC}\sim\overset{\triangle}{A'B'C'}$ tal que $\dfrac{A'B'}{AB}=k \ \Longrightarrow \ \dfrac{\text{área}(A'B'C')}{\text{área}(ABC)}=k^2$ (conocimiento: Área)
				\item[\textbullet] Anécdota: Teorema de Thales
			\end{itemize}
			\item[\ding{43}] Solución del ejercicio tipo: Propiedades 1 y 2
			\item[\ding{43}] Ejercicios y soluciones
		\end{itemize}
	\end{itemize}
	
\section{Estudio}
	
	Analizados ambos libros hemos podido comprobar que hay grandes diferencias entre uno y otro, sobre todo en cuanto a contenido en ambos temas:
	
	En el libro español podemos observar que al tratar la semejanza no se centran únicamente en la semejanza de triángulos, sino que la definen en general a partir del concepto de ``escala''. Hecho esto, particularizan en la semejanza de triángulos y, más concretamente, en la de los triángulos rectángulos; y, finalmente, termina con la semejanza en rectángulos, aportando curiosas aplicaciones de su estudio. 
	
	El libro francés, por el contrario, únicamente trata la semejanza de triángulos centrándose en sus propiedades principales (las caracterizaciones en las que se centra el artículo de Horoks: C1, C2, C3, P y Área) para resolver los problemas ``tipo'' de esta clase de noción.
	
	Un detalle importante de este desarrollo en el libro español es que, al trabajar con la semejanza de triángulos rectángulos más detenidamente, utiliza este concepto como demostración del Teorema del Cateto y el Teorema de la Altura. Considero que la importancia de dar estas dos nociones es substancial ya que, con un procedimiento simple y sencillo, se es capaz de probar dos resultados altamente útiles y efectivos -como se puede comprobar en los ejercicios del capítulo- y que están al alcance de los alumnos. Les permite comenzar a ser conscientes de la importancia de una prueba, que no por ser eficaz ha de ser compleja.
	
	Otro detalle destacable es la gran diferencia que podemos encontrar en cuanto al número de ejercicios propuestos y su contenido:
	
	En el libro español hay un total de cincuenta y cuatro ejercicios propuestos al final del tema, a los que hay que sumar los ejercicios planteados durante el capítulo, más los ejercicios de la reflexión y la autoevaluación. Esto hace un total de ochenta y ocho ejercicios, entre propuestos y resueltos, para que el alumno pueda realizar. Dada la extensión del tema y el tiempo que se tiene para impartirlo, es difícil que todos esos ejercicios puedan llevarse a cabo por algún alumno. Por añadidura, si algún alumno llegara a intentarlo, acabaría aburrido y desmotivado pues los ejercicios son altamente repetitivos: de los cincuenta y cuatro ejercicios propuestos al final del tema, diecisiete son aplicaciones del Teorema de Thales; once, aplicaciones del Teorema del Cateto y de la Altura y trece, aplicación directa de la escala y la razón de semejanza. Es improbable que a un estudiante le pueda ser eficaz intentar, siquiera, realizar todos estos ejercicios. Sin embargo, y por otro lado, el no enfocar tal cantidad de ejercicios como ``obligatorios para poder aprobar el examen'' sino como un listado variado donde el estudiante pueda elegir, tenga opción al error y una diversidad de ejercicios que le permitan practicar y subsanar dichos errores, pueda ser positivo.
	
	Por su parte, el libro francés plantea únicamente diez ejercicios, más el ejercicio tipo que resuelve al final. Si nos centramos en los diez ejercicios propuestos tras la resolución anterior: los dos primeros tratan la teoría preguntando sobre las nociones de forma teórica (verdadero/falso y elección de respuesta) y, tras ellos, se piden comprobaciones de semejanza de triángulos en diferentes contextos geométricos (circunferencias, paralelogramos, a través de una bisectriz, por Thales...) terminando con un problema de investigación un poco más complejo sobre semejanza. Estructurando así los ejercicios sí que es cierto que los alumnos no tienen tanta variedad como en el caso del libro español pero, siendo pocos y pudiéndose trabajar incluso en clase, es más motivador para el alumnos el llevarlos a cabo. Pese a ello, considero que sería propicio plantear más ejercicios, no llegando quizás a la cantidad que se presentan en \citet{spa}, pero sí los suficientes como para que el alumno pueda poner en práctica lo aprendido y, como se ha especificado antes, tenga opción a equivocarse y poder rectificar con problemas diferentes. Citando a \citet[p. 243]{Chevallard}: ``Estudiar problemas es un medio que permite crear y poner en marcha una técnica relativa a los problemas del mismo tipo, técnica que será a continuación el medio para resolver de manera casi rutinaria los problemas de este tipo''.
	
\section{Sistemas \{tarea, desarrollo\}}

	Para poder observar qué tipos de conocimiento entran en juego de cara a la resolución de los distintos tipos de problemas propuestos en ambos libros, analizaremos algunos de los ejercicios que en ellos aparecen. Se llevará a cabo un estudio por medio del sistema \{tarea, desarrollo\} propuesto en \citet{Horoks} para dos clases de ejercicio: uno ``tipo'' y otro ``más geométrico'' destinado a presentar aplicaciones de la semejanza en otro tipo de figuras geométricas. Dicha distinción es interesante ya que, como determina Horoks en su estudio, la configuración geométrica en la que se presente el ejercicio, junto con los conocimientos previos asociados, es una variable destacable en el análisis que se realice.
	
	Para realizar el análisis nos basaremos en los NPF, los conocimientos que desarrolla el artículo (C1, C2, C3, P y Área) y los que se definieron en la sección \ref{DefConN} (T, TC y TA) en lo referente al apartado de ``tarea''. La parte de desarrollo, en cambio, ha conllevado un límite en nuestro marco de trabajo ya que no ha sido posible llevar dicho factor al aula. Por ello, el análisis que se expone a continuación se ocupará únicamente de la parte correspondiente a la tarea, tal y como lo lleva a cabo Horoks siguiendo el ejemplo que se especificó al comienzo de este trabajo: Enunciado, solución y tabla de análisis. % puesto que no hemos podido poner en práctica dichos ejercicios en el entorno de un aula se propondrá un desarrollo que se considere adecuado y eficiente, basándonos en la complejidad de la resolución de los ejercicios que llevemos a cabo, en las observaciones de Horoks sobre las prácticas de las tres profesoras que analiza ella en su artículo, así como de las guías de algunos de los profesores de matemáticas del IES Alfonso X ``El Sabio'' de Murcia, con los que he podido colaborar.
	
\subsection{Ejercicios tipo}

\subsubsection*{Libro Español}
	A continuación se muestra un ejercicio tipo del libro de texto \citet{spa}, así como una breve solución. En concreto, presentamos el ejercicio 26, que podemos encontrar en la página 137 de dicho libro:%, reflejado en la Tabla \ref{ej26p137}. El ejercicio se propone de la siguiente manera:
	
	\begin{quote}\small
	\begin{multicols}{2}
		\begin{minipage}{7cm}
		\begin{flushleft}
			Para medir la altura de la casa, Álvaro, de $165cm$ de altura, se situó a $1,5cm$ de la verja y tomó las medidas indicadas. ¿Cuánto mide la casa?
		\end{flushleft}
		\end{minipage}
		
	\columnbreak
	
	\begin{flushright}
		\includegraphics[scale=0.25]{Ej26p137.png}
	\end{flushright}
	\end{multicols}
	\end{quote}
	
\begin{proof} Para resolver este ejercicio será necesario realizar un pequeño dibujo esquemático de la situación, como el que observamos más abajo, de manera que tenemos dos triángulos en posición de Thales (conocimiento T). Así, podemos aplicar la igualdad $$\dfrac{25+1,5}{1,5}=\dfrac ha$$ donde, calculando $a$ queda:
	
	\begin{center}
\begin{tabular}{>{\centering\arraybackslash}c>{\centering\arraybackslash}c}
		\multirow{5}{*}{\includegraphics[scale=0.42]{Ej26p137s111.png}} & 
			$a=3,5-1,65=1,85m$\\&\\
			& $\dfrac{25+1,5}{1,5}=\dfrac{h}{1,85}\rightarrow h=\dfrac{26,5\cdot1,85}{1,5}=32,68m$\\ &\\
			&$\text{Altura de la casa: }32,68+1,65=34,33m$
	\end{tabular}
	\end{center}
	
\end{proof}
	
	Una vez resuelto el ejercicio podemos observar los resultados del análisis resumidos en la Tabla \ref{ej26p137}.
	
	\begin{table}[h!]
	\centering
	\scalebox{1}[1]{
		\begin{tabular}{|c||l|c|l|c|}
\hline	\cellcolor[gray]{0.8}& \multicolumn{4}{|>{\cellcolor[cmyk]{0.9,0.3,0.2,0.1}}c|}{\textbf{Tarea}} \\ \cline{2-5}
%
	\cellcolor[gray]{0.8} & \cellcolor[gray]{0.95}	&\multicolumn{2}{|c|}{\cellcolor[gray]{0.95}Conocimiento}	& \cellcolor[gray]{0.95} 	\\ \cline{3-4}
%
	\multirow{-3}{*}{\cellcolor[gray]{0.8}ej} &	\multirow{-2}{*}{\cellcolor[gray]{0.95}Configuración} &	\cellcolor[gray]{0.95}Antiguo	&\cellcolor[gray]{0.95}Nuevo	&\multirow{-2}{*}{\cellcolor[gray]{0.95}NPF}\\
%
\hline		\cellcolor[gray]{0.8}26	&  \begin{minipage}[l]{0.2cm}\small Dibujo\end{minipage}	& \begin{minipage}[l]{3cm}\small\noindent Descomposición triángulos rectángulos\end{minipage}&\begin{minipage}[l]{0.2cm}\small\noindent\textbullet T\\ \noindent\textbullet C3\end{minipage}&\begin{minipage}[l]{3.5cm}\small\noindent\textbullet Cálculos\\ Intermedios\\ \noindent\textbullet Etapas\end{minipage}\\ \hline		
		\end{tabular}
		}
		\caption{Tabla Sistema \{tarea,desarrollo\} ejercicio 26, p.137}\label{ej26p137}
		\end{table}
%	\begin{table}[h!]
%	\scalebox{0.92}[1]{
%		\begin{tabular}{|c||l|l|l|l||l|l|l|l|}
%\hline	\cellcolor[gray]{0.8}& \multicolumn{4}{|>{\cellcolor[cmyk]{0.9,0.3,0.2,0.1}}c||}{\textbf{Tarea}} & \multicolumn{4}{|>{\cellcolor[cmyk]{0.9,0.3,0.2,0.1}}c|}{\textbf{Desarrollo}}\\ \cline{2-9}
%%
%	\cellcolor[gray]{0.8} & \cellcolor[gray]{0.95}	&\multicolumn{2}{|c|}{\cellcolor[gray]{0.95}Conocimiento}	& \cellcolor[gray]{0.95} &	\cellcolor[gray]{0.95} & \multicolumn{3}{|c|}{\cellcolor[gray]{0.95}Ayudas}	\\ \cline{3-4}\cline{7-9}
%%
%	\multirow{-3}{*}{\cellcolor[gray]{0.8}ej} &	\multirow{-2}{*}{\cellcolor[gray]{0.95}Configuración} &	\cellcolor[gray]{0.95}Antiguo	&\cellcolor[gray]{0.95}Nuevo	&\multirow{-2}{*}{\cellcolor[gray]{0.95}NPF}&	\multirow{-2}{*}{\cellcolor[gray]{0.95}\begin{minipage}[l]{0.2cm}\noindent Tiempo\\ Silencio\end{minipage}}	& \cellcolor[gray]{0.95}Naturaleza & \cellcolor[gray]{0.95}Momento & \cellcolor[gray]{0.95}Forma  \\
%%
%\hline		\cellcolor[gray]{0.8}26	&  \begin{minipage}[l]{0.2cm}\scriptsize\noindent\textbullet Thales\\ \noindent\textbullet Esquema/\\ dibujo\end{minipage}	& \begin{minipage}[l]{1.5cm}\scriptsize\noindent Descomp.  triángulos rectángulos\end{minipage}&\begin{minipage}[l]{0.2cm}\scriptsize\noindent\textbullet T\\ \noindent\textbullet C3\end{minipage}&\begin{minipage}[l]{1.31cm}\scriptsize\noindent\textbullet Cálculos\\ Intermedios\\ \noindent\textbullet Etapas\end{minipage}&	5 mints		&	Método	&\begin{minipage}[l]{0.2cm}\scriptsize\noindent Tras\\ Investigación\end{minipage} & \scriptsize Individual	\\ \hline		
%		\end{tabular}
%		}
%		\caption{Tabla Sistema \{tarea,desarrollo\} ejercicio 26, p.137}\label{ej26p137}
%		\end{table}
		
		Como podemos observar, la tarea se muestra con una breve descripción de un dibujo con dos triángulos en posición de Thales, lo cual lleva a la resolución que hemos marcado como ``tipo''. Su resolución, por tanto, debería ser sencilla, sin embargo, la forma en la que está expresado el ejercicio, junto con el dibujo que aporta detalles demasiado superfluos, puede llevar al alumno a confusión ya que interpretar el ejercicio para llegar al esquema que se muestra en la solución no es sencillo y, además, se ha de tener un nivel elevado de manejo de este tipo de situaciones. Debido a ello, como conocimiento previo será necesario reconocer esta estructura, descomponiendo la visualización de la imagen en los diferentes triángulos de la solución. En este punto se podrán emplear los nuevos conocimientos - C1 y P - resolviendo el ejercicio por etapas y con pequeños cálculos intermedios.
		
		%En lo referente al desarrollo para este ejercicio, puesto que se trata de un ejercicio tipo y, por tanto, sencillo, su resolución no será compleja cuando los alumnos hayan realizado ya algunos parecidos. Por ello, bastará con un periodo de cinco minutos de trabajo del alumno, tras los cuales, si es necesario, se guiará a aquellos que tengan algún problema centrándonos en el método de resolución; pero de forma individualizada, pues se espera que el conjunto de la clase no tenga problemas.
	

\subsubsection*{Libro Francés}

	Veamos ahora un ejemplo de \citet{fr}. 
	
	\paragraph{\underline{5. Cálculos de longitudes.}}Centrémonos en el ejercicio 5 que podemos encontrar en la página 261 del citado libro. Se tiene lo siguiente:
	\vspace{0.5cm}
	\begin{multicols}{2}
		\begin{minipage}{9cm}
	
	Sobre la figura siguiente, los triángulos $ABC$ y $BA'C'$ son semejantes. Con la ayuda de los datos de esta figura, responde a las siguientes cuestiones:
	\begin{enumerate}
		\item Calcular la longitud de $BA'$
		\item Calcular la longitud de $BC$
	\end{enumerate}
	\end{minipage}
		
	\columnbreak
	
%	\begin{center}
%		\includegraphics[scale=1]{Ej5p261.png}
%	\end{center}
\definecolor{zzttqq}{rgb}{0.6,0.2,0.}
\begin{tikzpicture}[line cap=round,line join=round,>=triangle 45,x=0.9cm,y=0.9cm]
\clip(-0.4208460561053181,-0.6538734257527625) rectangle (12.562054335131556,5.634366155046311);
\fill[color=zzttqq,fill=zzttqq,fill opacity=0.10000000149011612] (3.,0.) -- (7.,0.) -- (3.9794875000000003,3.046408416044006) -- cycle;
\fill[color=zzttqq,fill=zzttqq,fill opacity=0.10000000149011612] (3.9794875000000003,3.046408416044006) -- (6.009334715321547,2.099138506012362) -- (6.35575378388172,4.877626193395177) -- cycle;
\draw [color=zzttqq] (3.,0.)-- (7.,0.);
\draw [color=zzttqq] (7.,0.)-- (3.9794875000000003,3.046408416044006);
\draw [color=zzttqq] (3.9794875000000003,3.046408416044006)-- (3.,0.);
\draw [color=zzttqq] (3.9794875000000003,3.046408416044006)-- (6.009334715321547,2.099138506012362);
\draw [color=zzttqq] (6.009334715321547,2.099138506012362)-- (6.35575378388172,4.877626193395177);
\draw [color=zzttqq] (6.35575378388172,4.877626193395177)-- (3.9794875000000003,3.046408416044006);
\draw [shift={(3.,0.)}] plot[domain=0.:1.259713317181554,variable=\t]({1.*0.47402406126574403*cos(\t r)+0.*0.47402406126574403*sin(\t r)},{0.*0.47402406126574403*cos(\t r)+1.*0.47402406126574403*sin(\t r)});
\draw [shift={(3.9794875000000003,3.046408416044006)}] plot[domain=4.401305970771347:5.493518790998627,variable=\t]({1.*0.30375904219655825*cos(\t r)+0.*0.30375904219655825*sin(\t r)},{0.*0.30375904219655825*cos(\t r)+1.*0.30375904219655825*sin(\t r)});
\draw [shift={(3.9794875000000003,3.046408416044006)}] plot[domain=4.401305970771347:5.4935187909986265,variable=\t]({1.*0.3859035249165582*cos(\t r)+0.*0.3859035249165582*sin(\t r)},{0.*0.3859035249165582*cos(\t r)+1.*0.3859035249165582*sin(\t r)});
\draw [shift={(3.9794875000000003,3.046408416044006)}] plot[domain=-0.43663034610513485:0.6565729649128689,variable=\t]({1.*0.25857945415581607*cos(\t r)+0.*0.25857945415581607*sin(\t r)},{0.*0.25857945415581607*cos(\t r)+1.*0.25857945415581607*sin(\t r)});
\draw [shift={(3.9794875000000003,3.046408416044006)}] plot[domain=-0.43663034610513485:0.6565729649128711,variable=\t]({1.*0.32288132591701085*cos(\t r)+0.*0.32288132591701085*sin(\t r)},{0.*0.32288132591701085*cos(\t r)+1.*0.32288132591701085*sin(\t r)});
\draw [shift={(6.009334715321547,2.099138506012362)}] plot[domain=1.4467574054890744:2.7049623074846583,variable=\t]({1.*0.42349128661882796*cos(\t r)+0.*0.42349128661882796*sin(\t r)},{0.*0.42349128661882796*cos(\t r)+1.*0.42349128661882796*sin(\t r)});
\draw (2.604734208982927,0.12510059772144944) node[anchor=north west] {$\mathbf{A}$};
\draw (3.428866146861442,3.658707399858092) node[anchor=north west] {$\mathbf{B}$};
\draw (7.064078256407766,0.19283746932790266) node[anchor=north west] {$\mathbf{C}$};
\draw (5.946419874901288,2.24752257472365) node[anchor=north west] {$\mathbf{A'}$};
\draw (6.307683190135705,5.340839711418346) node[anchor=north west] {$\mathbf{C'}$};
\draw (2.582155251780776,1.931417173893535) node[anchor=north west] {$3,2$};
\draw (4.896498365001262,0.03478476891284516) node[anchor=north west] {$4$};
\draw (4.896498365001262,4.539286730741984) node[anchor=north west] {$3$};
\draw (6.206077882726025,3.760312707267772) node[anchor=north west] {$2,8$};
\end{tikzpicture}
	\end{multicols}

\begin{proof} Como se puede comprobar en la página 266 del mismo libro, la solución a este ejercicio se basa en la aplicación de la propiedad P. Además, se especifica que la necesidad de conocer los vértices homólogos para poder aplicarla, resaltando la importancia de este hecho. Resolviendo el ejercicio tenemos que:

	Basándonos en los datos del enunciado, $\hat{BAC}=\hat{BA'C'}$ y $\hat{ABC}=\hat{A'BC'}$, por lo que los lados homólogos son: $AB$ y $A'B$, $AC$ y $A'C'$, y $BC$ y $BC'$. Podemos, por tanto, escribir $$\dfrac{A'B}{AB}=\dfrac{C'B}{BC}=\dfrac{A'C'}{AC} \ \Longrightarrow \ \dfrac{A'B}{3,2}=\dfrac{3}{BC}=\dfrac{2,8}{4}$$
	
	\begin{enumerate}
		\item Para determinar la longitud $BA'$ tomamos $\dfrac{A'B}{3,2}=\dfrac{2,8}{4}$ con lo que $$A'B=\dfrac{2,8\times3,2}{4}=2,24$$
		\item Para determinar la longitud $BA'$ tomamos $\dfrac{3}{BC}=\dfrac{2,8}{4}$ con lo que $$BC=\dfrac{3\times4}{2,8}=4,29$$
	\end{enumerate}	
\end{proof}	
	
	Aludiendo de nuevo a las variables que conforman la tarea propuesta, analizamos el ejercicio con la Tabla \ref{ej5p261}.
	
	\begin{table}[h!]
	\centering
	\scalebox{1}[1]{
		\begin{tabular}{|c||l|c|l|c|}
\hline	\cellcolor[gray]{0.8}& \multicolumn{4}{|>{\cellcolor[cmyk]{0.9,0.3,0.2,0.1}}c|}{\textbf{Tarea}} \\ \cline{2-5}
%
	\cellcolor[gray]{0.8} & \cellcolor[gray]{0.95}	&\multicolumn{2}{|c|}{\cellcolor[gray]{0.95}Conocimiento}	& \cellcolor[gray]{0.95} 	\\ \cline{3-4}
%
	\multirow{-3}{*}{\cellcolor[gray]{0.8}ej} &	\multirow{-2}{*}{\cellcolor[gray]{0.95}Configuración} &	\cellcolor[gray]{0.95}Antiguo	&\cellcolor[gray]{0.95}Nuevo	&\multirow{-2}{*}{\cellcolor[gray]{0.95}NPF}\\
%
\hline		\cellcolor[gray]{0.8}5	&  \begin{minipage}[l]{0.2cm}\small Esquema/\\ dibujo\end{minipage}	& \begin{minipage}[l]{3cm}\small\noindent Cálculo literal\end{minipage}&\begin{minipage}[l]{0.2cm}\small P\end{minipage}&\begin{minipage}[l]{3.5cm}\small\noindent\textbullet Reconocer teorema\\ \noindent\textbullet Necesidad elección\end{minipage}\\ \hline		
		\end{tabular}
		}
		\caption{Tabla Sistema \{tarea,desarrollo\} ejercicio 5, p. 261}\label{ej5p261}
		\end{table}
%	\begin{table}[h!]
%	\scalebox{0.92}[1]{
%		\begin{tabular}{|c||l|l|l|l||l|l|l|l|}
%\hline	\cellcolor[gray]{0.8}& \multicolumn{4}{|>{\cellcolor[cmyk]{0.9,0.3,0.2,0.1}}c||}{\textbf{Tarea}} & \multicolumn{4}{|>{\cellcolor[cmyk]{0.9,0.3,0.2,0.1}}c|}{\textbf{Desarrollo}}\\ \cline{2-9}
%%
%	\cellcolor[gray]{0.8} & \cellcolor[gray]{0.95}	&\multicolumn{2}{|c|}{\cellcolor[gray]{0.95}Conocimiento}	& \cellcolor[gray]{0.95} &	\cellcolor[gray]{0.95} & \multicolumn{3}{|c|}{\cellcolor[gray]{0.95}Ayudas}	\\ \cline{3-4}\cline{7-9}
%%
%	\multirow{-3}{*}{\cellcolor[gray]{0.8}ej} &	\multirow{-2}{*}{\cellcolor[gray]{0.95}Configuración} &	\cellcolor[gray]{0.95}Antiguo	&\cellcolor[gray]{0.95}Nuevo	&\multirow{-2}{*}{\cellcolor[gray]{0.95}NPF}&	\multirow{-2}{*}{\cellcolor[gray]{0.95}\begin{minipage}[l]{0.2cm}\noindent Tiempo\\ Silencio\end{minipage}}	& \cellcolor[gray]{0.95}Naturaleza & \cellcolor[gray]{0.95}Momento & \cellcolor[gray]{0.95}Forma  \\
%%
%\hline		\cellcolor[gray]{0.8}5	&  \begin{minipage}[l]{0.2cm}\scriptsize Esquema/\\ dibujo\end{minipage}	& \begin{minipage}[l]{1.28cm}\scriptsize\noindent Cálculo literal\end{minipage}&\begin{minipage}[l]{0.2cm}\scriptsize\noindent P\end{minipage}&\begin{minipage}[l]{1.31cm}\scriptsize\noindent\textbullet Reconocer\\ teorema\\ \noindent\textbullet Necesidad\\ elección\end{minipage}&	5 mints		&	Método	&\begin{minipage}[l]{0.2cm}\scriptsize\noindent Tras\\ Investigación\end{minipage} &\scriptsize Colectivo	\\ \hline		
%		\end{tabular}
%		}
%		\caption{Tabla Sistema \{tarea,desarrollo\} ejercicio 5, p. 266}\label{ej5p266}
%		\end{table}
		
	En este ejemplo, la tarea describe la situación mediante un dibujo esquemático que representa el enunciado. Para su resolución podemos ver que sólo es necesario realizar los cálculos literales consecuentes de la aplicación de la propiedad P, una vez se elijan adecuadamente los lados correspondientes y se reconozca el teorema subyacente a esta propiedad. El ejercicio corresponde, por tanto, a lo que hemos categorizado como ``tipo'' debido tanto a su presentación como a su solución, ya que la tarea se presenta de forma clara y alude a un nivel muy simple, con el fin de que el alumno ponga en práctica lo aprendido, sin interés de ir más allá de la propia aplicación del conocimiento señalado.
	
	%Con respecto al desarrollo, nos encontramos ante un nuevo ejercicio tipo, de forma que no se esperan grandes problemáticas por parte del alumnado y bastará con un tiempo de trabajo propio de cinco minutos. Una vez trascurrido este tiempo, a modo de corrección se darán pautas colectivas sobre el método de resolución de este tipo de ejercicios.
	
	
\subsection{Aplicaciones a otras figuras geométricas}

	Los dos ejercicios que acabamos de ver representan los ejercicios más típicos que se requieren a los alumnos. Sin embargo, estas nociones permiten desarrollar algunos más complejos en los que poder aplicar la semejanza de triángulos y que están contenidos dentro de un marco de figuras geométricas, donde la utilidad de la semejanza se hace indispensable. Como bien expresa la propia autora en su tesis:
	
	\begin{quote}\small
		``Los ejercicios donde los triángulos semejantes son un útil para demostrar propiedades - generalmente sobre las longitudes y las áreas - son bastante raros [...] podrían, eventualmente, presentar un sustituto válido para reemplazar las trasformaciones ausentes, y volver a dar una dimensión intuitiva de esta localización\footnote{Aludiendo a las dificultades de la localización de los vértices homólogos.}, abstracta para los alumnos.''\cite[p. 340]{TH}.
	\end{quote}
	
	Por ello, se ha analizado también este tipo de ejercicios a fin de poder comprobar qué conocimientos se requieren en cada país ya que, como veremos, en España se hace gran uso de las propiedades T, TC y TA en dichos contextos. Veamos un ejemplo de este supuesto en cuanto a lo que abarcan los contenidos de la semejanza.
	
\subsubsection{Libro Español}

	En el caso del libro Español hay dos ejercicios basados en los teoremas del Cateto y de la Altura cuya demostración se basa, precisamente, en la semejanza de los triángulos rectángulos construidos a partir de un triángulo rectángulo mayor que los contiene. Estos ejercicios muestran muy bien la aplicación de la semejanza en cuerpos geométricos de muy distinta índole.
	
	\paragraph{\underline{Teorema del Cateto}} Para ejemplificar un caso de este tipo de ejercicios analizaremos el ejercicio 40 de la página 138 de \citet{spa}.
	
	\begin{multicols}{2}
		
		\begin{minipage}{7cm}
			Sobre una esfera de $20cm$ de radio se encaja un cono de $30cm$ de altura.
Halla el área del casquete esférico que determina el cono.
		\end{minipage}
		
	\columnbreak
	
	\begin{center}
		\includegraphics[scale=0.3]{Ej40p138.png}
	\end{center}
	\end{multicols}
	
\begin{proof} En primer lugar realizamos un pequeño dibujo esquemático del ejercicio para determinar cómo proceder, observando que podemos construir un triángulo rectángulo $ABC$ dividido en otros dos, también rectángulos, donde poder aplicar el Teorema del Cateto:

\begin{multicols}{2}
		
		\begin{center}
		\includegraphics[scale=0.35]{Ej40p138s1.png}
		\vspace{-2cm}
		\begin{align*}
				\hspace{6.85cm}x^2&+30x-400=0 \ \rightarrow \ x=\dfrac{-30\pm50}{2}=\begin{cases}-40cm\\ 10cm\end{cases}\\
				%\hspace{0}\text{Altura casquete}&=20-10=10cm\\
				%\text{Área casquete}&=2\pi Rh=400\pi cm^2
			\end{align*}
	\end{center}
				
	\columnbreak
	
	\begin{minipage}{7cm}
			Para calcular $x$ bastará aplicar el Teorema del Cateto para el cateto del rectángulo $ABC$, el cual coincide, precisamente, con el radio de la esfera:
			\begin{align*}
				20^2&=(30+x)\cdot x \ \rightarrow \ 400=30x+x^2\\
				%x^2&+30x-400=0 \ \rightarrow \ x=\dfrac{-30\pm50}{2}=\begin{cases}-40cm\\ 10cm\end{cases}
			\end{align*}	
			
%			Como estamos tratando con distancias, la solución $-40cm$ no es válida, por lo que $x=10$. Así:
%			\begin{align*}
%				%\text{Altura casquete}&=20-10=10cm\\
%				\text{Área casquete}&=2\pi Rh=400\pi cm^2
%			\end{align*}	 
		\end{minipage}
	
	\end{multicols}
	Como estamos tratando con distancias, la solución $-40cm$ no es válida, por lo que $x=10$. Así:
			\begin{align*}
				\text{Altura casquete}&=20-10=10cm\\
				\text{Área casquete}&=2\pi Rh=400\pi cm^2
			\end{align*}	
	%\vspace{-1.2cm}
\end{proof}

	El análisis de este ejercicio, según los conocimientos que emplea, queda reflejado en la Tabla \ref{ej40p138}.
	
	\begin{table}[h!]
	\centering
	\scalebox{1}[1]{
		\begin{tabular}{|c||l|c|l|c|}
\hline	\cellcolor[gray]{0.8}& \multicolumn{4}{|>{\cellcolor[cmyk]{0.9,0.3,0.2,0.1}}c|}{\textbf{Tarea}} \\ \cline{2-5}
%
	\cellcolor[gray]{0.8} & \cellcolor[gray]{0.95}	&\multicolumn{2}{|c|}{\cellcolor[gray]{0.95}Conocimiento}	& \cellcolor[gray]{0.95} 	\\ \cline{3-4}
%
	\multirow{-3}{*}{\cellcolor[gray]{0.8}ej} &	\multirow{-2}{*}{\cellcolor[gray]{0.95}Configuración} &	\cellcolor[gray]{0.95}Antiguo	&\cellcolor[gray]{0.95}Nuevo	&\multirow{-2}{*}{\cellcolor[gray]{0.95}NPF}\\
%
\hline		\cellcolor[gray]{0.8}40	&  \begin{minipage}[l]{0.2cm}\small Esquema/\\ dibujo\end{minipage}	& \begin{minipage}[l]{3.5cm}\small\noindent\textbullet Cálculo literal\\ \noindent\textbullet Álgebra (Ec. 2ºG.)\\ \noindent\textbullet Área casquete \end{minipage}&\begin{minipage}[l]{1.21cm}\small\noindent TC (P)\end{minipage}&\begin{minipage}[l]{3.6cm}\small\noindent\textbullet Reconocer teorema\\ \noindent\textbullet Cálculos intermedios\\ \noindent\textbullet Etapas\end{minipage}\\ \hline		
		\end{tabular}
		}
		\caption{Tabla Sistema \{tarea,desarrollo\} ejercicio 40, p. 138}\label{ej40p138}
		\end{table}
		
%	\begin{table}[h!]
%	\scalebox{0.92}[1]{
%		\begin{tabular}{|c||l|l|l|l||l|l|l|l|}
%\hline	\cellcolor[gray]{0.8}& \multicolumn{4}{|>{\cellcolor[cmyk]{0.9,0.3,0.2,0.1}}c||}{\textbf{Tarea}} & \multicolumn{4}{|>{\cellcolor[cmyk]{0.9,0.3,0.2,0.1}}c|}{\textbf{Desarrollo}}\\ \cline{2-9}
%%
%	\cellcolor[gray]{0.8} & \cellcolor[gray]{0.95}	&\multicolumn{2}{|c|}{\cellcolor[gray]{0.95}Conocimiento}	& \cellcolor[gray]{0.95} &	\cellcolor[gray]{0.95} & \multicolumn{3}{|c|}{\cellcolor[gray]{0.95}Ayudas}	\\ \cline{3-4}\cline{7-9}
%%
%	\multirow{-3}{*}{\cellcolor[gray]{0.8}ej} &	\multirow{-2}{*}{\cellcolor[gray]{0.95}Configuración} &	\cellcolor[gray]{0.95}Antiguo	&\cellcolor[gray]{0.95}Nuevo	&\multirow{-2}{*}{\cellcolor[gray]{0.95}NPF}&	\multirow{-2}{*}{\cellcolor[gray]{0.95}\begin{minipage}[l]{0.2cm}\noindent Tiempo\\ Silencio\end{minipage}}	& \cellcolor[gray]{0.95}Naturaleza & \cellcolor[gray]{0.95}Momento & \cellcolor[gray]{0.95}Forma  \\
%%
%\hline		\cellcolor[gray]{0.8}40	&  \begin{minipage}[l]{0.2cm}\scriptsize Esquema/\\ dibujo\end{minipage}	& \begin{minipage}[l]{1.28cm}\scriptsize\noindent\textbullet Cálculo literal\\ \noindent\textbullet Ec. 2ºG.\\ \noindent\textbullet Área casquete \end{minipage}&\begin{minipage}[l]{0.2cm}\scriptsize\noindent TC (P)\end{minipage}&\begin{minipage}[l]{1.31cm}\scriptsize\noindent\textbullet Reconocer\\ teorema\\ \noindent\textbullet Cálculo\\ intermedio\\ \noindent\textbullet Etapas\end{minipage}&	10 mints	&	Método	&\begin{minipage}[l]{0.2cm}\scriptsize\noindent Tras\\ Investigación\end{minipage} &\scriptsize Colectivo	\\ \hline		
%		\end{tabular}
%		}
%		\caption{Tabla Sistema \{tarea,desarrollo\} ejercicio 40, p. 138}\label{ej40p138}
%		\end{table}
		
	Este ejercicio presenta una resolución más laboriosa que es necesario interpretar. En este caso, ante la necesidad de reconocer la situación, se presenta un dibujo esquemático representándola. Para su resolución será necesario el cálculo literal que se deriva de la aplicación de la propiedad P, y las fórmulas geométricas correspondientes al cono que se habrán de conocer. Debido a ello se deberán manejar expresiones algebraicas dando lugar a la manipulación de la ecuación de segundo grado y sus soluciones. Para ello será necesario reconocer el Teorema del Cateto que respalda la propiedad P, así como establecer las etapas que nos permitan calcular el área final con sus correspondientes cálculos intermedios.
	
	%Este ejercicio presenta una resolución más laboriosa que se ha de interpretar, por lo que el tiempo de trabajo individual de los alumnos habrá de ser mayor que en los casos anteriores: diez minutos. Trascurrido este tiempo y, de nuevo, a modo de corrección y de forma colectiva, se mostrarán algunas pautas para el reconocimiento de la situación de un triángulo rectángulo que nos permita aplicar el teorema.
		

	\paragraph{\underline{Teorema de la Altura.}} Para ver ahora un ejemplo de este caso atenderemos al ejercicio 38 de los propuestos en la página 138.
	
	\begin{quote}\small
	\begin{multicols}{2}
		\begin{minipage}{7cm}
			En una esfera de $15cm$ de radio hemos inscrito un cono de altura $12cm$.
Calcula su área lateral.
		\end{minipage}
		
	\columnbreak
	
	\begin{center}
		\includegraphics[scale=0.25]{Ej38p138.png}
	\end{center}
	\end{multicols}
	\end{quote}
	
\begin{proof}
	Siguiendo la dinámica de estos ejercicios, el realizar un pequeño esquema/dibujo ayudará a la comprensión del ejercicio y una mejor visualización de la posible aplicación de algún teorema, en este caso, el Teorema de la Altura gracias al triángulo rectángulo $ABC$ que se obtiene:

\newpage	
	
	\begin{multicols}{2}
		
		\begin{center}
		\includegraphics[scale=0.3]{Ej38p138s1.png}
%		\vspace{0.4cm}
%		\begin{align*}
%		&\text{Generatriz cono: }\\ &\quad g^2=12^2+216=360 \ \rightarrow \ g\approx18,98cm\\
%		%&\text{Área lateral: }\\ &\quad\pi rg=2\pi\cdot14,7\cdot18,98\approx279\pi cm^2
%	\end{align*}
	\end{center}
	
	
				
	\columnbreak
	
	\begin{minipage}{7cm}
			Observemos que el radio de la esfera es $r_{\text{esf}}=15cm$, y así $x=\overline{DC}=30-12=18cm$
			
			Ahora, aplicando el Teorema de la Altura (propiedad P) para el triángulo $ABC$, tenemos que $$r^2=12x=12\cdot18=216 \ \rightarrow \ r\approx 14,7cm$$	
			
%			De este modo, podemos calcular la generatriz del cono fácilmente y, con ella, su área lateral:
%			\begin{align*}
%				&\text{Generatriz cono: }\\ &\quad g^2=12^2+216=360 \ \rightarrow \ g\approx18,98cm\\
%				&\text{Área lateral: }\\ &\quad\pi rg=2\pi\cdot14,7\cdot18,98\approx279\pi cm^2
%			\end{align*}	 
		\end{minipage}
	\end{multicols}
	
	De este modo, podemos calcular la generatriz del cono fácilmente y, con ella, su área lateral:
	\vspace{-0.2cm}
		\begin{align*}
			&\text{Generatriz cono: } \quad g^2=12^2+216=360 \ \rightarrow \ g\approx18,98cm\\
			&\text{Área lateral: }\quad\pi rg=\pi\cdot14,7\cdot18,98\approx279\pi cm^2
		\end{align*}
	\vspace{-0.8cm}
\end{proof}
	
	Analizando, de nuevo, sus conocimientos elaboramos la Tabla \ref{ej38p138}.
	
	\begin{table}[h!]
	\centering
	\scalebox{1}[1]{
		\begin{tabular}{|c||l|c|l|c|}
\hline	\cellcolor[gray]{0.8}& \multicolumn{4}{|>{\cellcolor[cmyk]{0.9,0.3,0.2,0.1}}c|}{\textbf{Tarea}} \\ \cline{2-5}
%
	\cellcolor[gray]{0.8} & \cellcolor[gray]{0.95}	&\multicolumn{2}{|c|}{\cellcolor[gray]{0.95}Conocimiento}	& \cellcolor[gray]{0.95} 	\\ \cline{3-4}
%
	\multirow{-3}{*}{\cellcolor[gray]{0.8}ej} &	\multirow{-2}{*}{\cellcolor[gray]{0.95}Configuración} &	\cellcolor[gray]{0.95}Antiguo	&\cellcolor[gray]{0.95}Nuevo	&\multirow{-2}{*}{\cellcolor[gray]{0.95}NPF}\\
%
\hline		\cellcolor[gray]{0.8}38	&  \begin{minipage}[l]{0.2cm}\small Esquema/\\ dibujo\end{minipage}	& \begin{minipage}[l]{3.5cm}\small\noindent\textbullet Cálculo literal\\ \noindent\textbullet Álgebra (Ec. 2ºG.)\\ \noindent\textbullet Área lateral cono \end{minipage}&\begin{minipage}[l]{1.2cm}\small\noindent TA (P)\end{minipage}&\begin{minipage}[l]{3.6cm}\small\noindent\textbullet Reconocer teorema\\ \noindent\textbullet Cálculos intermedios\\ \noindent\textbullet Etapas\end{minipage}\\ \hline		
		\end{tabular}
		}
		\caption{Tabla Sistema \{tarea,desarrollo\} ejercicio 38, p. 138}\label{ej38p138}
		\end{table}
%	\begin{table}[h!]
%	\scalebox{0.92}[1]{
%		\begin{tabular}{|c||l|l|l|l||l|l|l|l|}
%\hline	\cellcolor[gray]{0.8}& \multicolumn{4}{|>{\cellcolor[cmyk]{0.9,0.3,0.2,0.1}}c||}{\textbf{Tarea}} & \multicolumn{4}{|>{\cellcolor[cmyk]{0.9,0.3,0.2,0.1}}c|}{\textbf{Desarrollo}}\\ \cline{2-9}
%%
%	\cellcolor[gray]{0.8} & \cellcolor[gray]{0.95}	&\multicolumn{2}{|c|}{\cellcolor[gray]{0.95}Conocimiento}	& \cellcolor[gray]{0.95} &	\cellcolor[gray]{0.95} & \multicolumn{3}{|c|}{\cellcolor[gray]{0.95}Ayudas}	\\ \cline{3-4}\cline{7-9}
%%
%	\multirow{-3}{*}{\cellcolor[gray]{0.8}ej} &	\multirow{-2}{*}{\cellcolor[gray]{0.95}Configuración} &	\cellcolor[gray]{0.95}Antiguo	&\cellcolor[gray]{0.95}Nuevo	&\multirow{-2}{*}{\cellcolor[gray]{0.95}NPF}&	\multirow{-2}{*}{\cellcolor[gray]{0.95}\begin{minipage}[l]{0.2cm}\noindent Tiempo\\ Silencio\end{minipage}}	& \cellcolor[gray]{0.95}Naturaleza & \cellcolor[gray]{0.95}Momento & \cellcolor[gray]{0.95}Forma  \\
%%
%\hline		\cellcolor[gray]{0.8}38	&  \begin{minipage}[l]{0.2cm}\scriptsize Esquema/\\ dibujo\end{minipage}	& \begin{minipage}[l]{1.28cm}\scriptsize\noindent\textbullet Cálculo literal\\ \noindent\textbullet Ec. 2ºG.\\ \noindent\textbullet Área lat.\\cono \end{minipage}&\begin{minipage}[l]{0.2cm}\scriptsize\noindent TA (P)\end{minipage}&\begin{minipage}[l]{1.31cm}\scriptsize\noindent\textbullet Reconocer\\ teorema\\ \noindent\textbullet Cálculo\\ intermedio\\ \noindent\textbullet Etapas\end{minipage}&	10 mints	&	Método	&\begin{minipage}[l]{0.2cm}\scriptsize\noindent Tras\\ Investigación\end{minipage} &\scriptsize Colectivo	\\ \hline		
%		\end{tabular}
%		}
%		\caption{Tabla Sistema \{tarea,desarrollo\} ejercicio 38, p. 138}\label{ej38p138}
%		\end{table}
		
		Esta tarea es muy semejante a la anterior. Se presenta con un dibujo esquemático para su comprensión y se necesita del manejo y manipulación del álgebra para el trato de ecuaciones de segundo grado y sus soluciones, así como del conocimiento del área lateral del cono. Para resolverlo será necesario reconocer el Teorema de la Altura, obtenido de la aplicación de la propiedad P, y dividir el ejercicio en etapas para llegar a la solución por medio de pequeños cálculos intermedios.
		
		%Su desarrollo será también similar a la anterior. Ante la complejidad se dejarán diez minutos de silencio para el trabajo del alumno. Tras éste se darán pautas colectivas a modo de método para la localización del triángulo rectángulo necesario para la aplicación del teorema. 
	
\subsubsection{Libro Francés}	

	Centrémonos ahora en el libro francés escogido. Como hemos podido comprobar en las estructuras de los libros desarrolladas anteriormente, en este caso no se dan razonamientos sobre otras técnicas basadas en la semejanza, como el teorema del cateto o la altura en \citet{spa}, sino que todos los ejercicios versan sobre las propiedades fundamentales de la semejanza de triángulos. Aún así, podemos destacar algunos ejercicios más específicos que, al igual que en el punto anterior, utilizan figuras geométricas para su desarrollo.

%\newpage
	
	\paragraph{\underline{9. En un paralelogramo.}} Para el primer ejemplo recurriremos al ejercicio 9 de la página 263 de \citet{fr}:
	
%	\begin{multicols}{2}
%		
%	\begin{center}
		\begin{quote}
%		\begin{description}
			$ABCD$ es un paralelogramo. $N$ un punto del segmento $\overline{DC}$ distinto de $D$ y de $C$. La recta $\overline{AN}$ corta a $\overline{BC}$ en $M$.
			\begin{enumerate}
				\item Demostrar que los triángulos $ADN$ y $ABM$ son semejantes.
				\item Deducir que $\overline{DN}\times\overline{BM}=\overline{AB}\times\overline{AD}$
			\end{enumerate}
%		\end{description}
		\end{quote}
%	\end{center}

\begin{proof}
	
	En primer lugar, dado el enunciado, conviene considerar un pequeño esquema/dibujo de la situación, basándonos en la solución de la página 268 del propio libro:
	\begin{center}
\definecolor{zzttqq}{rgb}{0.6,0.2,0.}
\begin{tikzpicture}[line cap=round,line join=round,>=triangle 45,x=1.0cm,y=1.0cm]
\clip(1.5131583491981573,-1.471018967377032) rectangle (11.217440428860222,3.2292289616418777);
\fill[color=zzttqq,fill=zzttqq,fill opacity=0.10000000149011612] (2.,0.) -- (4.991384009202945,0.) -- (6.138546840034175,1.6382971157757922) -- (3.14716283083123,1.6382971157757922) -- cycle;
\draw (2.,0.)-- (3.14716283083123,1.6382971157757922);
\draw [color=zzttqq] (2.,0.)-- (4.991384009202945,0.);
\draw [color=zzttqq] (4.991384009202945,0.)-- (6.138546840034175,1.6382971157757922);
\draw [color=zzttqq] (6.138546840034175,1.6382971157757922)-- (3.14716283083123,1.6382971157757922);
\draw [color=zzttqq] (3.14716283083123,1.6382971157757922)-- (2.,0.);
\draw (2.,0.)-- (10.931376460263756,1.6382971157757922);
\draw (3.14716283083123,1.6382971157757922)-- (10.931376460263756,1.6382971157757922);
\draw (1.605981916916664,0.1576127207749421) node[anchor=north west] {$\mathbf{A}$};
\draw (2.7958112849447954,2.115346149019802) node[anchor=north west] {$\mathbf{B}$};
\draw (6.0193206366238465,2.132223161332258) node[anchor=north west] {$\mathbf{C}$};
\draw (4.9307533424704495,0.10698168383757502) node[anchor=north west] {$\mathbf{D}$};
\draw (5.344240144125616,0.6639230901486128) node[anchor=north west] {$\mathbf{N}$};
\draw (10.702691553330322,2.165977185957169) node[anchor=north west] {$\mathbf{M}$};
\end{tikzpicture}
	\end{center}
	\begin{enumerate}
		\item Los ángulos $\hat{BMA}$ y $\hat{MAD}$ son alternos-internos y, por construcción, $\overline{BM}$ // $\overline{AD}$ por lo que tienen la misma  medida: $\hat{BMA}=\hat{MAD}$.
		
		Además, por la propiedad de los paralelogramos, $\hat{ADN}=\hat{ABM}$.
		
		Así, los triángulos $ADN$ y $ABM$ tienen dos pares de ángulos con la misma medida y, por tanto, los restantes $\hat{AND}$ y $\hat{BAM}$ también.
		
		Aplicando la propiedad C1, los triángulos son semejantes.
		
		\item De lo anterior podemos ahora aplicar la propiedad P dando lugar a: $$\dfrac{\overline{DN}}{\overline{AB}}=\dfrac{\overline{AD}}{\overline{BM}}$$ luego $$\overline{DN}\times\overline{BM}=\overline{AB}\times\overline{AD}$$ utilizando la igualdad del producto en cruz.
	\end{enumerate}
\end{proof}
	%\vspace{-0.75cm}
	
	Procediendo como hasta ahora, una vez resuelto el ejercicio, podemos concluir su análisis a partir de los conocimientos que se utilizan. Para ello atendamos a la Tabla \ref{ej9p263}.
	
	\begin{table}[h!]
	\centering
	\scalebox{1}[1]{
		\begin{tabular}{|c||l|c|l|c|}
\hline	\cellcolor[gray]{0.8}& \multicolumn{4}{|>{\cellcolor[cmyk]{0.9,0.3,0.2,0.1}}c|}{\textbf{Tarea}} \\ \cline{2-5}
%
	\cellcolor[gray]{0.8} & \cellcolor[gray]{0.95}	&\multicolumn{2}{|c|}{\cellcolor[gray]{0.95}Conocimiento}	& \cellcolor[gray]{0.95} 	\\ \cline{3-4}
%
	\multirow{-3}{*}{\cellcolor[gray]{0.8}ej} &	\multirow{-2}{*}{\cellcolor[gray]{0.95}Configuración} &	\cellcolor[gray]{0.95}Antiguo	&\cellcolor[gray]{0.95}Nuevo	&\multirow{-2}{*}{\cellcolor[gray]{0.95}NPF}\\
%
\hline		\cellcolor[gray]{0.8}\begin{minipage}[l]{0.2cm}9 \\ 1\end{minipage}	& Enunciado & \begin{minipage}[l]{3.5cm}\small \noindent Propiedades\\ paralelogramo \end{minipage}&\begin{minipage}[l]{1.2cm}\small C1\end{minipage}&\begin{minipage}[l]{3.5cm}\small\noindent\textbullet Necesidad\\ elecciones\\  \noindent\textbullet Etapas\end{minipage}\\ 
%
\hline	\cellcolor[gray]{0.8}\begin{minipage}[l]{0.2cm}9 \\ 2\end{minipage}	& Enunciado & \begin{minipage}[l]{3.5cm}\small\noindent Cálculo literal\end{minipage}&\begin{minipage}[l]{1.2cm}\small\noindent P\end{minipage}&\begin{minipage}[l]{3.5cm}\small\noindent Cálculo intermedio\end{minipage}\\ \hline		
		\end{tabular}
		}
		\caption{Tabla Sistema \{tarea,desarrollo\} ejercicio 9, p. 263}\label{ej9p263}
		\end{table}
%	\begin{table}[h!]
%	\scalebox{0.92}[1]{
%		\begin{tabular}{|c||l|l|l|l||l|l|l|l|}
%\hline	\cellcolor[gray]{0.8}& \multicolumn{4}{|>{\cellcolor[cmyk]{0.9,0.3,0.2,0.1}}c||}{\textbf{Tarea}} & \multicolumn{4}{|>{\cellcolor[cmyk]{0.9,0.3,0.2,0.1}}c|}{\textbf{Desarrollo}}\\ \cline{2-9}
%%
%	\cellcolor[gray]{0.8} & \cellcolor[gray]{0.95}	&\multicolumn{2}{|c|}{\cellcolor[gray]{0.95}Conocimiento}	& \cellcolor[gray]{0.95} &	\cellcolor[gray]{0.95} & \multicolumn{3}{|c|}{\cellcolor[gray]{0.95}Ayudas}	\\ \cline{3-4}\cline{7-9}
%%
%	\multirow{-3}{*}{\cellcolor[gray]{0.8}ej} &	\multirow{-2}{*}{\cellcolor[gray]{0.95}Configuración} &	\cellcolor[gray]{0.95}Antiguo	&\cellcolor[gray]{0.95}Nuevo	&\multirow{-2}{*}{\cellcolor[gray]{0.95}NPF}&	\multirow{-2}{*}{\cellcolor[gray]{0.95}\begin{minipage}[l]{0.2cm}\noindent Tiempo\\ Silencio\end{minipage}}	& \cellcolor[gray]{0.95}Naturaleza & \cellcolor[gray]{0.95}Momento & \cellcolor[gray]{0.95}Forma  \\
%%
%\hline		\cellcolor[gray]{0.8}9	&  \begin{minipage}[l]{0.2cm}\scriptsize Enunciado\end{minipage}	& \begin{minipage}[l]{1.7cm}\scriptsize\noindent\textbullet Cálculo literal\\ \noindent\textbullet Propiedades\\ paralelogramo \end{minipage}&\begin{minipage}[l]{0.2cm}\scriptsize\noindent\textbullet P\\ \noindent\textbullet C1\end{minipage}&\begin{minipage}[l]{1.31cm}\scriptsize\noindent\textbullet Necesidad\\ elecciones\\  \noindent\textbullet Etapas\end{minipage}&	5-10 mints	&	Aclaración	&\begin{minipage}[l]{0.2cm}\scriptsize\noindent Durante la\\ investigación\end{minipage} &\scriptsize Individual	\\ \hline		
%		\end{tabular}
%		}
%		\caption{Tabla Sistema \{tarea,desarrollo\} ejercicio 9, p. 248}\label{ej9p248}
%		\end{table}
		
		En este caso, a pesar de que se ofrece una situación geométrica no trivial, el ejercicio se presenta únicamente mediante un enunciado, lo cual dificulta su realización. Para comenzar, es necesario conocer las propiedades que presentan los ángulos de un paralelogramo, las cuales permiten poner en práctica el conocimiento C1 para resolver el primer apartado en varias etapas. Para el segundo apartado será necesario hacer una correcta elección de las dimensiones correspondientes de los triángulos resultantes. Tras esto, bastará un cálculo literal de las consecuencias del primer apartado, aplicando la propiedad P, para obtener la demostración que requería el ejercicio.
		
		%En cuanto al desarrollo, basándonos también en lo propuesto por el libro, se considera que un periodo de diez minutos será suficiente para resolverlo. Sin embargo, puesto que los alumnos pueden tener la mayor dificultad a la hora de estructurar la situación del problema, tras un periodo de cinco minutos de silencio que les permita razonar sobre él, se llevarán a cabo pequeñas aclaraciones-guía de forma individual a aquellos alumnos que estén atascados en algún punto del ejercicio. Se espera, por tanto, que salvo casos puntuales la clase sea capaz de resolver el ejercicio. %No obstante, este ejercicio está pensado para ser uno de los últimos a llevar a cabo en el tema a fin de que los alumnos tengan cierta soltura sobre los conocimientos.
	
	\paragraph{\underline{10. Un poco de investigación.}} En este segundo ejemplo que podemos encontrar en la página 263 del mismo libro, el ejercicio 10 se presenta a través de una circunferencia asumiendo, a la vez, una nueva propiedad con la que el alumno debe trabajar.
	
	\begin{center}
		\begin{quote}
			Consideramos un círculo de diámetro $\overline{CD}$ en el cual se inscribe un triángulo $ABC$ tal que $\hat{ACD}=\hat{ABD}=22º$.\\
			Sea $H$ la base de la altura trazada desde $C$ en el triángulo $ABC$.\\
			Admitamos la propiedad siguiente:\textit{``Si un triángulo está inscrito en un círculo, y uno de sus lados es un diámetro de ese círculo, entonces dicho triángulo es rectángulo.''}
			\begin{enumerate}
				\item Dibuja la figura
				\item Demostrar que los triángulos $ADC$ y $BHC$ son semejantes
				\item Deducir que $\overline{CA}\times\overline{CB}=\overline{CH}\times\overline{CD}$
			\end{enumerate}
		\end{quote}
	\end{center}
	
\begin{proof} Basémonos en la resolución de la página 269 de dicho libro:
	\begin{enumerate}
		\item Dibujando la figura queda:
		\begin{center}
			\includegraphics[scale=0.7]{Ej10p243.png}
		\end{center}
		\item  Veamos ahora que los ángulos de los triángulos $ADC$ y $BHC$ son iguales a fin de aplicar la propiedad C1:
		\begin{itemize}
			\item El triángulo $ADC$ está inscrito en el círculo de diámetro $\overline{CD}$, aplicando la propiedad del enunciado tenemos que es rectángulo en $A$. Además, $BHC$ es rectángulo en $H$ por definición. Así $\hat{BHC}=\hat{DAC}$.
			\item El triángulo $DBC$ está inscrito en el círculo de diámetro $\overline{DC}$, por lo que es rectángulo en $B$; además, $\hat{DBA}=22º$ luego $\hat{ABC}=90º-22º=68º$, de forma que $ADC$ es rectángulo en $A$. Así $\hat{ADC}=90º-\hat{ACD}=90º-22º=68º$. Así, $\hat{ADC}=\hat{ABC}$
			
			En conclusión, los triángulos $ACD$ y $ABC$ tienen los ángulos iguales dos a dos y son, por tanto, semejantes.
		\end{itemize}
		\item Los triángulos $ACD$ y $CHB$ son semejantes, donde $\overline{CH}$ y $\overline{CA}$ por un lado y $\overline{CB}$ y $\overline{CD}$ por otro, son homólogos (puesto que el ángulo recto se encuentra en $H$ y $A$). Así, aplicando la propiedad P: $$\dfrac{\overline{CH}}{\overline{CA}}=\dfrac{\overline{CB}}{\overline{CD}} \ \Longrightarrow \ \overline{CA}\times\overline{CB}=\overline{CH}\times\overline{CD}$$
	\end{enumerate}
\end{proof}
	 
	 Analizando este último ejercicio a partir de los conocimientos empleados, comprobemos en la Tabla \ref{ej10p263} los resultados.
	 
	 \begin{table}[h!]
	\centering
	\scalebox{1}[1]{
		\begin{tabular}{|c||l|c|c|c|}
\hline	\cellcolor[gray]{0.8}& \multicolumn{4}{|>{\cellcolor[cmyk]{0.9,0.3,0.2,0.1}}c|}{\textbf{Tarea}} \\ \cline{2-5}
%
	\cellcolor[gray]{0.8} & \cellcolor[gray]{0.95}	&\multicolumn{2}{|c|}{\cellcolor[gray]{0.95}Conocimiento}	& \cellcolor[gray]{0.95} 	\\ \cline{3-4}
%
	\multirow{-3}{*}{\cellcolor[gray]{0.8}ej} &	\multirow{-2}{*}{\cellcolor[gray]{0.95}Configuración} &	\cellcolor[gray]{0.95}Antiguo	&\cellcolor[gray]{0.95}Nuevo	&\multirow{-2}{*}{\cellcolor[gray]{0.95}NPF}\\
%
\hline		\cellcolor[gray]{0.8}\begin{minipage}[l]{0.2cm}10 \\ 1\end{minipage}	& Enunciado & \begin{minipage}[l]{3.5cm}\small \noindent Elementos\\ circunferencia \end{minipage}&\backslashbox{\:}{\:}&\begin{minipage}[l]{3.5cm}\small\noindent Necesidad elección\end{minipage}\\ 
%
\hline	\cellcolor[gray]{0.8}\begin{minipage}[l]{0.2cm}10 \\ 2\end{minipage}	& Enunciado & \begin{minipage}[l]{3.5cm}\small\noindent\textbullet Cálculo literal\\ \noindent\textbullet Suma ángulos\\ triángulo \end{minipage}&\begin{minipage}[l]{1.8cm}\small\noindent\textbullet C1\\ \noindent\textbullet Propiedad\\ enunciado\end{minipage}&\begin{minipage}[l]{3.5cm}\small\noindent Cálculos intermedios\end{minipage}\\ 
%
\hline	\cellcolor[gray]{0.8}\begin{minipage}[l]{0.2cm}10 \\ 3\end{minipage}	& Enunciado & \begin{minipage}[l]{3.5cm}\small\noindent Cálculo literal\end{minipage}&\begin{minipage}[l]{1.8cm}\small\noindent P\end{minipage}&\begin{minipage}[l]{3.5cm}\small\noindent\textbullet Necesidad elección\\ \noindent\textbullet Etapas\end{minipage}\\ \hline	
		\end{tabular}
		}
		\caption{Tabla Sistema \{tarea,desarrollo\} ejercicio 10, p. 263}\label{ej10p263}
		\end{table}
%	 \begin{table}[h!]
%	\scalebox{0.92}[1]{
%		\begin{tabular}{|c||l|l|l|l||l|l|l|l|}
%\hline	\cellcolor[gray]{0.8}& \multicolumn{4}{|>{\cellcolor[cmyk]{0.9,0.3,0.2,0.1}}c||}{\textbf{Tarea}} & \multicolumn{4}{|>{\cellcolor[cmyk]{0.9,0.3,0.2,0.1}}c|}{\textbf{Desarrollo}}\\ \cline{2-9}
%%
%	\cellcolor[gray]{0.8} & \cellcolor[gray]{0.95}	&\multicolumn{2}{|c|}{\cellcolor[gray]{0.95}Conocimiento}	& \cellcolor[gray]{0.95} &	\cellcolor[gray]{0.95} & \multicolumn{3}{|c|}{\cellcolor[gray]{0.95}Ayudas}	\\ \cline{3-4}\cline{7-9}
%%
%	\multirow{-3}{*}{\cellcolor[gray]{0.8}ej} &	\multirow{-2}{*}{\cellcolor[gray]{0.95}Configuración} &	\cellcolor[gray]{0.95}Antiguo	&\cellcolor[gray]{0.95}Nuevo	&\multirow{-2}{*}{\cellcolor[gray]{0.95}NPF}&	\multirow{-2}{*}{\cellcolor[gray]{0.95}\begin{minipage}[l]{0.2cm}\noindent Tiempo\\ Silencio\end{minipage}}	& \cellcolor[gray]{0.95}Naturaleza & \cellcolor[gray]{0.95}Momento & \cellcolor[gray]{0.95}Forma  \\
%%
%\hline		\cellcolor[gray]{0.8}10	&  \begin{minipage}[l]{0.2cm}\scriptsize Enunciado\end{minipage}	& \begin{minipage}[l]{1.2cm}\scriptsize\noindent\textbullet Cálculo\\ literal\\ \noindent\textbullet Suma\\ ángulos\\ triángulo \end{minipage}&\begin{minipage}[l]{1.3cm}\scriptsize\noindent\textbullet P\\ \noindent\textbullet C1\\ \noindent\textbullet Propiedad\\ enunciado\end{minipage}&\begin{minipage}[l]{1.31cm}\scriptsize\noindent\textbullet Cálculos\\ intermedios\\  \noindent\textbullet Etapas\end{minipage}&	10 mints	&	Método	&\begin{minipage}[l]{0.2cm}\scriptsize\noindent Tras\\ Investigación\end{minipage} &\scriptsize Colectivo	\\ \hline		
%		\end{tabular}
%		}
%		\caption{Tabla Sistema \{tarea,desarrollo\} ejercicio 10, p. 248}\label{ej10p248}
%		\end{table}
	 
	 Esta última tarea presenta una complejidad notablemente más alta que las anteriores. Partiendo de que se presenta una situación geométrica únicamente a partir de un enunciado, se afirma una propiedad que se ha de asumir y, más aún, ser utilizada en el propio ejercicio (cosa que no es nada usual para el alumno). La tarea se divide en tres apartados: en el primero sólo entran en juego las capacidades gráficas del alumno, que deberá de tener claros los elementos de la circunferencia a los que hace referencia el enunciado, así como hacer uso de la elección del punto $B$ que satisfaga las propiedades que se indican, ya que dicho punto no es único; en el segundo apartado se deberán utilizar las medidas de los ángulos de un triángulo para establecer la semejanza a través de la propiedad C1 mediante cálculos intermedios y diversas etapas; por último, en el tercer apartado, bastará aplicar la propiedad P a las conclusiones de semejanza deducidas del apartado anterior para obtener la solución.
	 
	 %En lo referente al desarrollo de este ejercicio, dada su estructura, es interesante que los alumnos tengan tiempo de razonar sobre él por lo que se dejarán 10 minutos de silencio para el trabajo autónomo del alumno. Una vez transcurrido dicho periodo se resolverá el ejercicio marcando pautas de forma colectiva explicando los métodos a llevar a cabo, de forma que el conjunto de la clase exponga sus dudas y se pueda aclarar la resolución al conjunto del grupo.
	 
	 
	 
	 

\section{Conclusiones}

	En primer lugar hemos de destacar que, para la realización de este trabajo, el marco de investigación ha tenido unos límites definidos por los recursos de los que se ha contado. Todo el análisis que se ha realizado ha sido a través de dos libros, uno español y otro francés, intentando, además, que fueran de uso lo más actual posible para que las conclusiones llevadas a cabo tuvieran sentido. Dado que se han utilizado únicamente dos libros, las conclusiones obtenidas no pueden, a priori, generalizarse. Sin embargo, se ha de tener en cuenta que estos libros representan a los programas oficiales de sus países, y estos sí son generales y de obligado cumplimiento. %Con todo, se tiene muy claro que una generalización de estas conclusiones quedan algo limitadas a lo que aquí podemos observar, sin embargo, se ha de tener en cuenta que tanto dichas conclusiones como el análisis expuesto quedan respaldados por los programas escolares, a los que representan los libros escogidos.

	Definido el punto de partida y basándonos en el análisis del trabajo llevado a cabo, podemos observar que hay una gran diferencia a la hora de enfocar y enseñar la semejanza de triángulos en ambos países:
	
	Como hemos podido ver apoyándonos en los ejercicios y el enfoque teórico que aporta el libro de \citet{spa}, en España, la semejanza de triángulos versa, sobre todo, sobre el Teorema de Thales. A partir de éste se estructura toda la dinámica de la semejanza. A pesar de que las propiedades C1, C2, C3 y Área se trabajan en el tema, no se requieren en los ejercicios salvo en alguna ocasión concreta, sino que se utilizan para la demostración del Teorema de la Altura y del Cateto. Más allá de este hecho, el uso de dichas propiedades queda implícito al colocar los triángulos en posición de Thales para comprobar su semejanza y aplicar el teorema. Podemos concluir entonces que, en España, la semejanza va completamente ligada al Teorema de Thales, otorgando la importancia y repercusión a éste, y quedando la propia semejanza, en ocasiones, en un segundo plano. 
	
	En Francia, al contrario que en España, es el teorema de Thales el que queda desplazado a un segundo plano, como un caso particular de las consecuencias de las propiedades de la semejanza de triángulos. Tanto en la teoría como en los ejercicios es más que evidente el énfasis que se da a las distintas propiedades C1, C3, P y Área, siendo éstas el centro total de estudio del tema.
	
	Podríamos decir, en este contexto, que la semejanza sufre un autismo temático dado que no se relaciona con otras áreas de la matemática como podría ser el campo de las funciones a través de las homotecias o las isometrías; ni tampoco con otras asignaturas, como el dibujo técnico donde la noción tiene un papel fundamental. No se marca, de este modo, ninguna utilidad ni objetivo para el aprendizaje de la noción más que los breves comentarios que, en el caso del libro español, se ofrecen, intentando contextualizar la noción pero sin ir más allá del propio capítulo.
	
	Cierto es, como ya se ha mencionado, que el enfoque del libro de \citet{spa} es mucho más genérico que el libro de \citet{fr}, enfocando la semejanza de manera global, no sólo para triángulos. Esto lleva a que se destaque más la importancia en otros conceptos, sin embargo, y centrándonos en uno de los problemas principales que señala la autora como es la localización de los vértices homólogos, esto no ayuda a superar los obstáculos y dificultades que plantea el artículo de Horoks sobre la semejanza de triángulos. Es más, al trabajar desde la perspectiva del teorema de Thales, la problemática de reconocer dichos vértices, cuando estos no están en una situación casi trivial, se acentúa ya que, para encontrar la semejanza, colocarlos en una posición de Thales puede ser una labor muy complicada. Además, cabe añadir la dificultad que analiza Horoks en su tesis de forma complementaria al resto, y es la que presenta el álgebra, ya que, en ocasiones, es necesaria para las resoluciones, obligando a un cambio de marco teórico que puede suponer un obstáculo para algunos alumnos.
	
	En conclusión, hemos podido ver que las formas de trabajo en ambos países son realmente diferentes. Sin embargo,  ni en un país con Thales (España) ni en el otro, basándonos en el artículo de Horoks (Francia), se enfrentan a las dificultades más importantes que destaca la propia autora en su investigación como son: las elecciones del profesor en la gestión de la clase cuando se basan en los programas escolares, junto con las dificultades u obstáculos en los que se pueden convertir; los conceptos con autismo temático marcados por los manuales escolares; o la que más critica Horoks en su trabajo, la localización de los vértices homólogos en situaciones no triviales.
	
%	\begin{itemize}
%		\item La localización de los vértices homólogos en situaciones no triviales
%		\item Los conceptos con autismo didáctico marcados por los manuales escolares, junto con las elecciones del profesor en la gestión de la clase en relación a éstos, y las dificultades en las que devienen
%		\item 
%	\end{itemize}
	
	

	
	
	
	
	
	