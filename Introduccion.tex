\chapter*{Introducción}
\phantomsection\addcontentsline{toc}{chapter}{Introducción}
\pagenumbering{arabic} 
\setcounter{page}{1}
\lhead[Trabajo Final de Máster]{Introducción}
\rhead[Introducción]{Trabajo Final de Máster}
\lfoot[\thepage]{José Andrés Muñoz}
\rfoot[José Andrés Muñoz]{\thepage}
\begin{dedication} 
\large ``La agudeza consiste en saber la semejanza de las cosas diferentes,\\ y la diferencia de las cosas semejantes.''
	\rightline{{\rm --- Germaine De Staël (1766-1817)}}
\end{dedication}
\pagestyle{fancy}
%\begin{dedication} 
%\large ``Las matemáticas, bien vistas,\\ no sólo poseen la verdad.''
%	\rightline{{\rm --- Bertrand Russell (1872-1970)}}
%\end{dedication}

	La Geometría, vista desde los ojos de un matemático, no es únicamente una de las ramas básicas del estudio de esta maravillosa ciencia, sino que es la forma que tenemos de estructurar, analizar y explicar todo aquello que nos rodea. Categorizamos la realidad desde las figuras geométricas más simples como los cuadrados, los triángulos o los polígonos regulares; hasta los más enrevesados fractales que son capaces de componer un copo de nieve, estructurar el crecimiento de algunas plantas o describir comportamientos en el universo.
	
	En este contexto de armonías y figuras que nos permite desarrollar la geometría, se sitúa el objeto de estudio en el que se centrará nuestro tema: la Semejanza. Más concretamente nos centraremos en la semejanza de triángulos y las propiedades que de ella se derivan. Aludiendo a las palabras del famoso matemático y escritor Eric Temple Bell (1883-1960): ``Ningún tema pierde tanto cuando se le divorcia de su historia como las Matemáticas'' \cite[p. 17]{Suma45}, y haciendo honor a ellas merece la pena comentar brevemente el contexto histórico en el que se desarrolla esta noción.
	
	Como se especifica en \citet{Suma58}, se tiene constancia de que ya en la antigua mesopotamia se precisaba de la resolución de problemas geométricos para el cálculo de figuras y superficies agrarias en los que la situación se describía mediante triángulos semejantes. Claro está, no con la notación actual, pero sí dando lugar a las primeras concepciones de esta noción. Más tarde, en el siglo VII a.C., Thales de Mileto dio los primeros avances en la proporción y la semejanza de triángulos mediante los teoremas de semejanza que se le atribuyen. Sin embargo, no fue hasta el 300 a.C. con la obra de Euclides, \textit{Los Elementos}, más concretamente el libro VI, cuando se establecen las definiciones y proposiciones asociadas a esta noción.
	
	Visto ahora que matemáticos de la talla de Thales o Euclides estudiaron esta noción y que, además, es uno de los temas que se tratan en los programas escolares, cabe preguntarse cómo se enfocará la didáctica de este tema para que pueda adaptarse a los distintos niveles de enseñanza. Con este objetivo, la doctora en matemáticas Julie Horoks observó que, en Francia, el número de alumnos que acuden a clases particulares para poder superar la asignatura de matemáticas está creciendo a medida que pasan los años. Ello suscitó el desarrollo de su tesis y de un artículo sobre este tema, particularizando en los contenidos y prácticas de enseñanza que se aplican en \textit{seconde}, concretamente, en la semejanza de triángulos.
	
	Así, en este trabajo nos dedicaremos a intentar analizar la investigación que Horoks llevó a cabo. Para ello se estructurará en tres partes bien diferenciadas:
	
	En el primer capítulo nos centraremos en el análisis del propio artículo de Horoks, que complementaremos en ciertas ocasiones con algunas alusiones hacia su tesis, donde se definirán los parámetros principales del análisis: los niveles de puesta en funcionamiento de los alumnos a la hora de enfrentarse a los ejercicios propuestos, y los conocimientos básicos que se habrán de adquirir una vez superada la noción. También se comentará algún ejemplo de cómo la autora lleva a cabo su investigación en relación al análisis de los ejercicios, terminando con las conclusiones a las que le llevó su estudio.
	
	El segundo capítulo tiene el objetivo de adaptar la investigación de Horoks a dos casos particulares de la docencia de la semejanza: uno en España y otro en Francia, a través de dos libros de texto de uso actual en los centros. Se analizarán, de este modo, algunos ejercicios dispuestos en ellos a partir del sistema \{tarea, desarrollo\} que propone la autora. Tras ello se dedica una sección para tratar las conclusiones a las que se ha llegado una vez desarrollado el grueso del trabajo, comparando con los resultados obtenidos por la propia autora en su investigación.
	
	Para terminar, se concluirá el trabajo con un último capítulo destinado a unas reflexiones más globales que versarán sobre el propio artículo, el trabajo llevado a cabo y algunas reflexiones personales que ha suscitado el trabajo a lo largo de su desarrollo y que se ha considerado relevante destacar.
	
	En definitiva, con este trabajo se pretende dar una visión reflexionada sobre la importancia de las prácticas docentes y el impacto que pueden llegar a ejercer sobre los alumnos. La trascendencia del papel del profesor que, sujeto en cierto modo a unas limitaciones impuestas por los tiempos y los programas académicos, puede ser notable sobre ellos.
	
	
	
	
	
	
	
	